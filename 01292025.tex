\documentclass{assets/fieldnotes}

\title{Kono (Sierra Leone)}
\author{LING3020/5020}
\date{University of Pennsylvania, Spring 2025\\01/29/2025 Phonetics and Phonology I }

\setcounter{secnumdepth}{4} %enable \paragraph -- for subsubsubsections

\begin{document}

\maketitle
\tableofcontents

\newpage 
%(Daniel absent today)



%dipthongs / vowel sequences 
%vowel nasalization - CVN/CVVN (compared to CV/CVV) 
\section{Consonants - stop voicing and place contrast -- Alex}



\section{/gb/, /kp/  versus /b/, /p/ --- Wesley}

\wml{Planned minimal pairs -- I will try at least a few pairs for each consonantal contrast, so that we can get tokens for a few different phonological environments.}\\
\\
\wml{\textipa{/\t{gb}/ vs. /b/}}\\

\exg. \textipa{b\'{e}k\`{i}}\\
bag\\
`bag'

\exg. \textipa{\texttoptiebar{gb}\'{e}k\`{i}}\\
Kono.surname\\
`(a Kono surname)'

\exg. \textipa{\t{gb}\^{i}}\\
pumpkin\\
`pumpkin'

\exg. \textipa{b\^{i}}\\
termite\\
`termite'

\exg. \textipa{b\'{i}}\\
today\\
`today'

\exg. \textipa{\t{gb}o\textlengthmark}\\
body\\
`body'

\exg. \textipa{\t{gb}\`o\textlengthmark}\\
padlock\\
`padlock'

\exg. \textipa{\t{gb}\^o\textlengthmark}\\
curse\\
`curse'

\exg. \textipa{\t{gb}\'{o}\textlengthmark}\\
book\\
`book'

\exg. \textipa{bo\textlengthmark}\\
hand\\
`hand'

\exg.\textipa{\t{gb}\'{e}}\\
all\\
`all'

\exg. \textipa{b\`{e}}\\
give\\
`to give'

\exg. \textipa{\t{gb}a}\\
dry\\
`dry'

\exg. \textipa{ba}\\
big\\
`big'

\exg. \textipa{sam\t{gb}a}\\
drum\\
`drum'

\exg. \textipa{samba}\\
basket\\
`basket'

\exg. \textipa{jamba}\\
leaf\\
`leaf'

\wml{\textipa{/\t{gb}/} vs. \textipa{/\t{kp}/}}\\



\wml{I had one token of \textipa{\t{kp}} in my January 27 notes and wasn't sure if this was truly a voiceless phoneme. Foday-Ngongou reports interspeaker phonological variation in voiced and voiceless counterparts.}

\exg. \textipa{\t{kp}ajne}\\
fish\\
`to fish'

\wml{There's also...:}\\

\exg. \textipa{\t{kp}am\t{gb}a}\\
oven\\
`oven'

\wml{(Near)-minimal pairs:}

\exg. \textipa{\t{kp}e\t{kp}e}\\
frog\\
`frog'

\exg. \textipa{\t{kp}\textepsilon\t{kp}\textepsilon}\\
comb\\
`comb'

\exg. \textipa{\t{gb}\textepsilon\t{gb}\textepsilon}\\
stubborn\\
`stubborn'

\exg. \textipa{\t{gb}e\ng\t{gb}\textepsilon}\\
boat\\
`boat'

\wml{\textipa{/p/} vs. \textipa{/b/}}

\exg. \textipa{p\textsuperscript h u}\\
overseas\\
`overseas'

\exg. \textipa{b\'{u}}\\
stomach\\
`stomach'

\exg. \textipa{\t{gb}u}\\
swollen\\
`swollen'

\section{lateral, approximates -- Chun-Hung}

\chs{/j/ vs. /w/}

\exg. ju \\ 
rope \\
`rope' (m83)

\exg. wu \\
you.PL \\
`you (pl)' (m83)

\exg. wuu \\
dog \\
`dog'

\exg. ja \\
eye \\
eye 

\exg. jaa \\
peanut \\
`peanut'

\exg. jaa \\
lion \\
`lion' (m83)

\exg. wa \\
big \\
`big'

\exg. jai \\
name of a person \\
name of a person (n85)

\exg. wai \\
resting.place \\
resting place (n85)

\exg. ji \\
water \\
`water'

\exg. wi \\
blood \\
`blood' (F85)

\chs{/l/ and /\textipa{\*r}/}

\chs{/l/ is restricted to the word-medial position according to F85, unless loanwords.} 

\chs{/\textipa{\*r}/ is not documented}

\exg. dulu \\
five \\
`five'

\exg. dusu \\
fiber \\
`fiber from bamboo tree' (m83)

\exg. w\textipa{O}l\textipa{O} \\
six \\
`six'

\exg. d\textipa{O}f\textipa{O} \\
insignificant \\
`insignificant'

\exg. \textipa{\*r}umu \\
room \\
`room'

\exg. mumun\textipa{E} \\
dumb.person \\
`a dumb person' (n85)

\section{Back vowels - height contrast --- Joey}

\jf{Short Vowels:}

\ex. tù \\
'to pound/hit/touch'

\ex. tù \\
'oil'

\ex. tó \\
'to stay/remain' \\
\jf{This works better in the context of 'stay here/don't follow me.' Anthony said that it usually needs more context tó ne = 'stay'}

\ex. siné \\
'to stay/remain (seated here)'

\ex. nɔ́ \\
'there' ('not here') \\
\jf{This can't stand alone. It works better in the context of 'we are there.' - bé nɔ́}

\ex. pandɛ \\
'there' ('not here')

\ex. bù̆ \\
'Stomach'

\ex. bɔ́ \\
'to go out' (used if you're telling someone to go as well it is more common in the form of bɔ́wa)

\ex. dú \\
'to share/distribute'

\ex. dòmâ \\
'shirt'

\ex. dɔ̀ndɔ̀ \\
'one' (when ennumerating things, not when counting)

\ex. dòmâdɔ̀ndɔ̀wɛ \\
'one shirt'

\ex. t͡ʃɛlɛ \\
'one' (when counting, not when ennumerating things)


\jf{Long Vowels:}


\ex. kôː \\
'To fatten (an animal)'

\ex. túː \\
'To become fat' \\
\jf{Anthony mentioned that this actually means the fat that is inside of an animal}

\ex. tóː \\
'To envy/hate/be jealous of'

\ex. daŋba\\
'Dirty'

\ex. bóː\\
'Body'

\ex. gbôː\\
'Hand'





\section{Affricates and fricatives -- Jan}

\exg.
\textipa{s\'EI} \\
gloss \\
`to climb' % originally [se]

\exg. \textipa{\texttoptiebar{tS}\'En\'E} \\
gloss \\
`leg' 

\exg. \textipa{\texttoptiebar{pf}uu} \\ 
gloss \\
`overseas, abroad' 

\exg. \textipa{mako\`a} \\
gloss \\
`brush'

\exg. \textipa{\texttoptiebar{mb}ENamak\^of\'en\`e} \\
gloss \\
`thing used to clean spoon'

\exg. \textipa{k\`o\`o} \\
gloss \\
`wash'

\exg. \textipa{cE} \\
gloss \\
`to do'

\exg. \textipa{c\`E\`E} \\
gloss \\
`to call'

\exg. \textipa{\texttoptiebar{mb}\^acen\`e} \\
gloss \\
`neighbor'

\exg. \textipa{cim\'a\`a} \\
gloss \\
`cold'

\exg. \textipa{\texttoptiebar{gb}ak\`uo} \\
gloss \\
`adult', `older person'

\exg. \textipa{musukaku\`o} \\
gloss \\
`adult man'

\exg. \textipa{\texttoptiebar{gb}akokaj} \\
gloss \\
`adult man'

\exg. \textipa{mukuM\texttoptiebar{gb}abuku} \\
gloss \\
`adult man'

\exg. \textipa{\texttoptiebar{gb}\'e\'e} \\
gloss \\
`meet'

\exg. \textipa{\texttoptiebar{gb}\'en\`e} \\
gloss \\
`meet'

\exg. \textipa{sw\'e\`e} \\
gloss \\
`meat'

\exg. \textipa{sw\`E\`E} \\
gloss \\
`bean'

\exg. \textipa{eab\'asi} \\
gloss \\
`onion'

\exg. \textipa{cEnJEnE} \\
gloss \\
`football'

\exg. \textipa{sw\'in\`e} \\
gloss \\
`rest'

\exg. \textipa{cEne} \\
gloss \\
`house'

\jmt{everything else below was not elicited}

\exg. \textipa{p\'Esa} \\
gloss \\
`sitting area outside house' 

\exg. \textipa{cE} \\
gloss \\
`to do' (climb)

\exg. \textipa{\texttoptiebar{tS}i:ma} \\
gloss \\
`cold' 

\exg. \textipa{a-kOa\texttoptiebar{tS}i} \\
3SG.POSS wear \\
`(s)he's wearing' 

\exg. \textipa{pa\texttoptiebar{tS}EnE} \\
gloss \\
`neighbor'

\exg. \textipa{\texttoptiebar{tS}\'en\`e} \\
gloss \\
`house', `home' 

\exg. \textipa{p\'Esa} \\
gloss \\
`morning' 

\exg. \textipa{\texttoptiebar{tS}EnanE} \\
gloss \\
`adult' 

\exg. \textipa{\texttoptiebar{tS}Enank\super hai} \\
gloss \\
`adult man' 

\exg. \textipa{\texttoptiebar{tS}Enamusu} \\
gloss \\
`adult woman' 

\exg. \textipa{swei} \\
gloss \\
`meat' (bean)

\exg. \textipa{f\super{j}awa} \\
gloss \\
`forest' 

\exg. \textipa{k\'i\texttoptiebar{tS}i} \\
gloss \\
`kitchen' 

\exg. \textipa{eab\'asi} \\
gloss \\
`onion' 

\exg. \textipa{\texttoptiebar{tS}ena} \\
gloss \\
`morning' 

\exg. \textipa{s\^amb\super{H}a} \\
gloss \\
`drum' 

\exg. \textipa{\texttoptiebar{tS}\super{j}En\texttoptiebar{\textrtaild\textrtailz}EnE} \\
gloss \\
`football' 

\exg. \textipa{s\super{w}winE} \\
gloss \\
`rest' 

\section{Nasal consonants - place contrasts Lex} 
\jal{minimal pairs for place contrasts?}

\ex. koŋg wa or kon tʃenama or kom ba \\
` Big tree'\\

\ex.koné \\
mountain'\\

\ex. koˈŋgʷé tʃenama, koˈŋgʷé wa %check IPA for this\\
`mountain'\\

\ex. ɲinba\\
` big tooth'\\

\ex. ɲinɛ\\
`tooth'\\

\ex. nin-ba\\
`big tooth'\\

\ex. n-ɲinɛ\\
`my tooth'\\

\ex. my fish

\ex. musú\\
`woman'\\

\ex. mɔɛ̃ \\
`human'\\

\ex.  mɔɛ̃-ɡ͡bu \\
`human being'\\

\ex. mɔɛ̃kama\\
` human man'\\

\ex. wiinE
`deer'\\

\ex. senɛ\\
`stone'\\

\ex. sɛnɛ\\
`rice farm'\\

\ex. bùːbùː\\
`vegetable farm'\\

\ex. yadí\\
`farm of cash crops'\\

\ex. nɛːnɛnɛ\\
`tongue'\\

\exg. n-nɛːnɛnɛ\\
` my tongue'\\


\section{Front vowels - height contrast -- Giang}
Minimal pairs we got from last time:
%also I tried to get the formats from Praat and plot the vowel but it looks... wrong. -- check with Jianjing
\ex. \textipa{tS\'en\^E}\\
house\\
\g{The falling tone on the second syllable might just be an end-of-word thing...}\\

\ex. \textipa{tS\`en\'E}\\
leg\\

\ex. \textipa{f\'in\`e}\\
black\\
\g{Foday-Ngongou has this as \textipa{f\'in\^E}.}\\

\ex. \textipa{f\'en\`e}\\
thing\\
\g{I think the contrast between black and thing is just the first vowel: i vs e.}\\

Other words for front vowels:
\ex. \textipa{b\`i}\\
today\\ %this might have been mentioned last time??
%it was mentioned, but im getting it again for contrast anyway.

\ex. \textipa{b\`e:}\\
redemption fee (in a secret society)\\

\ex. \textipa{b\`e}\\
to give\\
\g{Near minimal set...}\\

\ex. \textipa{d\'i}\\
sweet\\

\ex. \textipa{d\1e}\\
mother\\

\ex. \textipa{d\`i:}\\
to get used to\\
\g{Near minimal set}\\

\ex. \textipa{n\`E}\\
here\\

\ex. \textipa{m\'E}\\
to stay longer, live longer, overstay\\

\ex. \textipa{f\`i:}\\
to miss, mistake, e.g: a person; a type of rice\\

\ex. \textipa{f\'E}\\
to blow, inflate; seize\\

\ex. \textipa{\t{kp}\`E}\\
to fry; difficult, hard\\

\ex. \textipa{\t{kp}\'e:\t{kp}\^e}\\
curse\\
\g{Near minimal pair with ``to fry; difficult''}\\

\ex. \textipa{s\`E}\\
to climb\\

\ex. \textipa{s\`ic\`i}\\
an inner feeling, suspicion\\

\ex. \textipa{s\`Ec\`E}\\
a type of musical instrument\\
\g{This is a minimal pair but the meaning is so weird...}\\
\jal{he said this meant ``this s\textipa{EE}, it's bimorphemic"}

\ex. \textipa{c\'ec\^e}\\
ringworm\\

\ex. \textipa{c\'ic\^i:}\\
a round hut\\
\g{This is again a minimal pair with a weird meaning...}\\
\jal{what's weird about the meaning?}

\ex. \textipa{f\'if\^i:}\\
mud, sand\\

\ex. \textipa{f\'Ef\^E:}\\
a spray of rain\\
\g{This is again a minimal pair with a weird meaning...}

\section{Vowel Length Contrast - Mingyang}
\begin{itemize}
    \item \textbf{i vs. i:}
    \ex. si:\\
        `music'

    \ex. si\\
        `to sit'
    
    \ex. donbi:\\
        `orange'

    \ex. bi\\
        `today'

    \item \textbf{a vs. a:}
    \ex. banfa:\\
        `pants'

    \ex. banfa\\
        `disrespect'

    \ex. fa\\
        `father'

    \ex. k$^w$a\\
        `monkey'
    
    \ex. ka:\\
        `snake'

    \ex. ta:\\
        `calabash'

    \ex. ta\\
        `fire'

    \item \textbf{ɛ vs. ɛ:}
    \ex. t͡ʃɛ\\
        `this'

    \ex. t͡ʃɛ:\\
        `war'

    \ex. tɛtɛ\\
        `spider'
    
    \ex. tɛ:\\
        `chicken'

    \ex. dɛ\\
        `mother'

    %\mb{The pair of `chicken' and `mother' is covered by Alex.}

    \ex. m$^w$ɛ\\
        `person'

    \ex. mɛ:\\
        `knife'

    \ex. ɲɛː\\
        `fish'

    \item \textbf{ɔ vs. ɔ:}
    \ex. bɔ\\
        `to comb'

    \ex. bɔ:\\
        `to curse'
    
    \item \textbf{e vs. e:}
    \ex. t$^h$e:\\
        `sun'

    \ex. te\\
        `to break'

    \item \textbf{u vs. u:}
    \ex. bu:\\
        `garden'

    \ex. bu\\
        `stomach'
    
    \ex. du:\\
        `to bend'

    \ex. du\\
        `town'

    \item \textbf{o vs. o:}
    \ex. bo:\\
        `hand'
        
    \ex. bo\\
        `to cut'
        
    \ex. to:\\
        `ear'

\end{itemize}

\end{document}
