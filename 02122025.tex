\documentclass{assets/fieldnotes}

\title{Kono (Sierra Leone)}
\author{LING3020/5020}
\date{University of Pennsylvania, Spring 2025\\02/12/2025 Morphology I}

\setcounter{secnumdepth}{4} %enable \paragraph -- for subsubsubsections

\newcommand{\mb}[1]{\textcolor{Dandelion}{#1}}


\begin{document}


\maketitle
\tableofcontents

%\jal{Note to everyone:  please pay attention to what others are preparing so that we don't make him repeat from 10 minutes before and give him the impression that we aren't learning from what he's telling us.  You can however build from what previous people have asked; e.g. so a few minutes ago / a couple of weeks ago you were telling us the word for X, it was /X/?  Right, so now I'm wondering how you would say ``Saa's X?"}

\section{Inalienable Possession -- Chun-Hung} 

%\jal{you'll note from previous notes and the dissertation that there's a further contrast in ``our''; remember these may end up tricky to elicit, you need to have thought it through}\\
%\jal{do also try a nominal possessor (name and/or very simple noun}

%\chs{I'm using the ones that are included in the field notes two years ago but were not elicited at that time.}

\chs{disyllabic starting with a H tone}

\ex. m\'{a}m\'{a}-k\'{a}\'{a}\`{i} \\
grandparent-man \\
`grandfather'

\ex. m\'{a}m\'{a}-m\'{u}s\`{u} \\
grandparent-woman \\
`grandmother'

\ex. \'{m}-m\'{a}m\`{a}-k\`{a}\`{a}\`{i} \\
1SG-grandparent-man \\
`my grandfather'

\ex. \'{i}-m\'{a}m\`{a}-k\`{a}\`{a}\`{i} \\
2SG-grandparent-man \\
`your (sg.) grandfather'

\ex. \`{a}-m\`{a}m\`{a}-k\`{a}\`{a}\`{i} \\
3SG-grandparent-man \\
`his/her grandfather'

\ex. \`{m}-m\'{a}m\`{a}-k\`{a}\`{a}\`{i} \\
1EXCL-grandparent-man \\
`our (exclusive) grandfather' \chs{didn't succeed in eliciting the number distinction}

\ex. m\'{o}-m\`{a}m\`{a}-k\`{a}\`{a}\`{i} \\
1INCL-grandparent-man \\
`our (inclusive) grandfather' \chs{didn't succeed in eliciting the number distinction}


\ex. w\'{o}-m\'{a}m\`{a}-k\`{a}\`{a}\`{i} \\
2PL-grandparent-man \\
`your (pl.) grandfather'

\ex. \`{a}m-m\'{a}m\`{a}-k\`{a}\`{a}\`{i} \chs{not sure whether it's a /am/ or a nasalized vowel for 3PL.GEN}\\
3PL-grandparent-man \\
`their grandfather'


\chs{disyllabic starting with a L tone}

\ex. k\`{a}m\'{a}n\`{e} \\
`shoulder'

\ex. \'{\textipa{N}}-k\'{a}m\`{a}n\`{e} \\
1SG-shoulder \\
`my shoulder'

\ex. \'{i}-k\'{a}m\`{a}n\`{e} \\
2SG shoulder \\
`your (sg.) shoulder'

\ex. \`{a}-k\`{a}m\'{a}n\`{e} \\
3SG shoulder \\
`his/her shoulder'

\ex. \`{\textipa{N}}\`{\textipa{N}}-k\`{a}\'{a}m\`{a}n\`{e} \\
1EXCL shoulder \\
`our (exclusive) shoulder'

\ex. m\'{o}-k\`{a}m\'{a}n\`{e} \\
1INCL shoulder \\
`our (inclusive) shoulder'

\ex. w\'{o}-k\'{a}m\`{a}n\`{e} \\
2PL shoulder \\
`your (pl.) shoulder'

\ex. \'{a}\textipa{N}-k\'{a}m\`{a}n\`{e} \\
3PL shoulder \\
`their shoulder'


\chs{others: POSS + beard vs. POSS + (fat?) belly or POSS + nose vs. POSS+ head}


\section{Alienable Possession -- Mingyang}
%\jal{did he say dogs were kept as pets? if not, it's not an ideal noun.  Also not ideal in that sometimes languages characterize pets along with humans/family}\\
%\mb{Got it! Using `medicine' instead.}\\
%\jal{also note that time permitting you'll want to do nouns of different tonal patterns, and a noun that starts with a clear consonant}\\
\mb{Noun with H H tones}
\ex. búí\\
    `medicine'

\exg. ná-búí\\
    1SG.POSS-medicine\\
    my medicine

\exg. já-búí\\
    2SG.POSS-medicine\\
    your(sg.) medicine

\exg. wá-búí\\
    2PL.POSS-medicine\\
    your(pl.) medicine

\ex. à-búí\\
    3SG.POSS-medicine\\
    his/her medicine

\ex. àná-búí\\
    3PL.POSS-medicine\\
    their(pl.) medicine

\ex. mùà-búí\\
    1DU.INCL.POSS-medicine\\
    our (dual, inclusive) medicine

\exg. mùâ-búí\\
    1PL.INCL.POSS-medicine\\
    our (plural, inclusive) medicine

\exg. nǎ-búí\\
    1PL.EXCL.POSS-medicine\\
    our (dual/plural, exclusive) medicine

%\jal{you'll note from previous notes and the dissertation that there's a further contrast here; remember these may end up tricky to elicit, you need to be ready}

\exg. jàì-à-búí\\
    Jai-3SG.POSS-medicine\\
    Jai's medicine

\exg. sà-à-búí\\
    Sa-3SG.POSS-medicine\\
    Sa's medicine

%\jal{use a Kono name! He's given us multiple girls and boys names}
%\mb{Got it! Using `Jai' instead.}

Alienable possession morphemes (prefixes, preceding HH):
\begin{center}
    \begin{tabular}{|c|c|c|c|}
    \hline
    Person/Number & SG & DU & PL \\ \hline
    1 & ná & m\mb{ùà}(incl.) & m\mb{ùâ}(incl.)/nǎ(excl.) \\ \hline
    2 & já & N/A & wá \\ \hline
    3 & à & N/A & àná \\ \hline
    \end{tabular}
\end{center}


%\jal{good idea to use a simple noun, but this one is itself preferably/obligatorily inalienably possessed; pick another one}
%\mb{Got it! Using `teacher' instead.}


\section{Noun plus Adjective -- Daniel} % 
%\ds{Going to come up with pairs for tone combinations; how much should I focus purely on that and how much can I go into adj. order?}

%\jal{good, start with the tone combinations, I don't expect you to get to multiple adjectives at all, but definitely not ordering, which can be quite difficult}


Adjectives:
\exg. nàmà (nàmâ)\\
new\\

\exg. wâ\\
big\\

\ex. fáá\\
heart\\

\exg. fáà wa\\
heart big\\
`big heart'

\exg. \textipa{tS\'E}nà-ma\\
big-?\\

\exg. fáà \textipa{tSE}na-ma\\
heart big-(?)\\
`a big heart'

\exg. i fáà kô\\
2SG heart big\\
`(you are?) hot-tempered/proud'

\exg. d\textipa{\=o}\\
small\\

\exg. fáà dòò-ma\\
heart small-(?)\\
`small heart'

\exg. \textipa{feN} dòò-ma\\
thing small ?\\
`small object'

\exg. (n)dé\textipa{\`E}\\
small\\

\exg. \textipa{f\`eN} dé\\
thing small\\
`small object' (short from fene dé)

\exg. fáà e\textipa{N}dê\\
heart small\\
`small heart'

\ds{Quality/subjective}
\ex. jáo\\
bad (character/personality)

\exg. i-jào\\
2SG-bad\\
`you are bad (personality)

\exg. a-má-nyí\\
3SG-\Neg{}-good\\
not nice (appearance)

\exg. i-má-nyí\\
2SG-\Neg{}-good\\
`he is ugly'

\exg. m-má-nyí\\
1SG-\Neg{}-good\\
`I am ugly'

\ex. nyí\\
good/pretty (appearance)\\

\exg. a-nyí\\
3SG-nice\\
`(something) is nice'

\ex. sòné\\
character (personality)\\

\exg. a sònè nyi\\
3SG character good\\
good personality

\exg. a sònè ma-nyi\\
3SG character \Neg{}-good\\
bad/`ugly' personality

\ex. \textipa{tS\'End\`E}\\
good (object)

\exg. fe\textipa{N} \textipa{tS\`E}ndè\\
object good\\
a good item


\ds{Colour}
\exg. gbè\\
white\\

\exg. tecìì gbe\\
egg white\\
`white egg'

\exg. j\textipa{\=a}wá\\
red\\

\exg. \textipa{feN} jàwa\\
red thing\\

\exg. finɛ\\
black\\


\ex. gray\\
\ds{Commented that does not exist}

\ex. {\textltailn}e{\textltailn}ene\\
any different colour\\

\ex. {\textltailn}e{\textltailn}ende\\
any different colour\\


\section{Demonstratives -- Lex} 
\exg. dà: tʃɛ̀\\
 pot this\\
`this pot'

\exg. dà: tʃɛ̀nù\\
 pots these\\
`these pots'

\g{Anthony says same words as "This pot" just need to point directly at the other object }

\exg. dà: tʃɛ̀ \\
 pot that\\
`that pot'

\g{Anthony says same words as "These pots" just need to point directly at the other objects }

\exg. dà: tʃɛ̀nù \\
 pots those\\
`those pots'

\g{Anthony says you can express "that pot" or "those pots" by color, location, or time}

\exg. dà: fínɛ̀\\
 pot black\\
`black pot'

\exg. dà: jáwâ\\
 pot red\\
`red pot'

\exg. dà: tʃɛ̀ kònέkɔ̀\\
  pot this tree under\\
` the pot under the tree'

\exg. dà: tʃɛ̀ n-kɔ̀\\
 pot this under me\\
` give me that pot under me'

\exg. káŋɡánɛ̂\\
door\\
`door'

\exg. tʃɛ̀nέdâ\\
door\\
`door'

\exg. dà: tʃɛ̀ káŋɡánɛ̂ m-bέma\\
 pot this door close by  \\
` the pot close by the door'

\exg. dà: tʃɛ̀ tʃɛ̀nέdâ bέmá\\
 pot this door close by \\
` the pot close by the door'

\exg. dà: tʃɛ̀  ʃopidʒɔ̀ \\
 pot this store\\
`the pot at the store'

\g{Anthony says just refers to location of pot at store }

\exg. ʃopidʒɔ̀\\
store\\
`store'

\exg. ʃopijɔ̀\\
store\\
`store'

\exg. mwâ dà: miŋdʒέ ʃopidʒɔ̀\\ 
the pot we saw at the store\\ %Don't know how to structure these sentences syntactically in second line
`The pot we saw at the store'

\exg. mwâ dà: miŋdʒέ kùnù\\
the pot we saw yesterday\\
`the pot we saw yesterday'

\exg.mwâ dà: miŋdʒέ bi\\
the pot we saw today\\
`the pot we saw today'

\exg. m-bátʃɛ̀nέ\\
neighbor\\
 `neigbor'
 
\exg. m-bátʃɛ̀nέtʃɛ̀\\
this neighbor\\
 `this neigbor'

 

\section{Counting -- Giang} 

\ex. tʃélê\\
one

\ex. féà\\
two

\ex. sáwâ\\
three

\ex. náání\\
four

\ex. dúú\\
five

\ex. ɔ́ɔ́rlɔ́\\
six\\
\g{Really uncertain of this. Tony spells it as or-law.}
\wml{I don't think there's really a rhotic there. Tony probably spells it with an <r> because his English is non-rhotic, and <or> is his way of clarifying that it's [ɔ] and not [o].}

\ex. ómféà\\
seven\\
\g{Tony says the féà part is like in `two.'}

\ex. séêŋ\\
eight\\
\g{Tony: As if you are "saying" something}

\ex. kɔ̀nɔ́ŋtʰô\\
nine

\ex. tâŋ\\
ten

\g{Tony: When you're up to 10, you start with ten first and then you add the number.}

\ex. táŋ-kùmá dòndò\\
eleven\\
\g{Tony: kùmá is `on top of,' so this means `on top of ten.' so this is `one on top of ten.' If I'm not mistaken, dòndò is also used for `one' but not in counting contexts.}

\ex. táŋ-kùmá féà\\
twelve

\ex. táŋ-kùmá sàwâ\\
thirteen

\ex. táŋ-kùmá nààní\\
fourteen

\ex. táŋ-kùmá dùù\\
fifteen\\
\g{Pretty sure I heard a low tone on dùù here.}

\ex. táŋ-kùmá ɔ̀ɔ̀rlɔ̀\\
sixteen

\ex. táŋ-kùmá ómféà\\
seventeen

\ex. táŋ-kùmá séêŋ\\
eighteen 

\ex. táŋ-kùmá kɔ̀nɔ́ŋtʰô\\
nineteen

\ex. bí-féà\\
twenty

\ex. bí-féà-kùmá dòndò\\
twenty-one

\ex. bí-féà-kùmá féà\\
twenty-two

\ex. bí-féà-kùmá sàwâ\\
twenty-three

\ex. bí-féà-kùmá nààní\\
twenty-four

\ex. bí-féà-kùmá dùù\\
twenty-five

\ex. bí-féà-kùmá ɔ̀ɔ̀rlɔ̀\\
twenty-six

\ex. bí-féà-kùmá ómféà\\
twenty-seven

\ex. bí-féà-kùmá séêŋ\\
twenty-eight

\ex. bí-féà-kùmá kɔ̀nɔ́ŋtʰô\\
twenty-nine

\ex. bí-sáwâ\\
thirty\\
\g{Then you repeat with kùmá + number.\\
For 30, 40, etc., say bí+number.}

\ex. bí-nààní\\
forty

\ex. bí-dùù\\
fifty

\ex. bí-ɔ̀ɔ̀rlɔ̀\\
sixty

\ex. bí-ómféà\\
seventy

\ex. bí-séêŋ\\
eighty

\ex. ní-kɔ̀nɔ́ŋtʰô\\
ninety

\ex. tʃɛ̀mɛ́ dòndò\\
one hundred\\
\g{dòndò is one, tʃɛ̀mɛ́ is hundred. }


\ex. tʃɛ̀mɛ́ féà\\
two hundred

\ex. tʃɛ̀mɛ́ sáwâ\\
three hundred


\section{Noun plus Numeral -- Joey} 

%\jf{Based on previous elicitation data (specifically, that 'one shirt' is dòmâdɔ̀ndɔ̀wɛ), it appears that numbers are affixed to the ends of the nouns they modify, so I will try to find nouns that end in a variety of ways (re: vowels, nasals, tones, lengths, height, frontness, etc.). Additionally, as Tony mentioned that numbers used for counting and numbers used to modify nouns are different, I wonder if there is a kind of counter system that has a number of distinct counters; with this in mind, I will also try to include nouns that fall into different semantic categories, in case the counter changes depending on the category of the noun that it modifies. Finally, I will use a number of different numerals (generally one, two, and something else) to try to get some variety and to use as a basis for future elicitations if we want to see if there are any differences in agreement between singular, plural, dual}\\

\jf{Length Contrast:}

\ex. dà\\
`an opening/mouth'

\ex. kà\\
`Opening'\\

\ex. dàdɔ̀ndɔ̀\\
mouth-one\\
`one mouth' \\

\ex. dàféà\\
mouth-two\\
`two /mouths' \\

\ex. dǎsáwà\\
mouth-three\\
`three mouths' \\

\ex. dàː\\
`pot'\\

\ex. dàːdɔ̀ndɔ̀\\
pot-one\\
`one pot'\\

\ex. dàːféà\\
pot-two\\
`two pots'\\

\ex. dàːsáwà\\
pot-three\\
`three pots'\\

\ex. dáːɔ́lô\\
pot-six\\
`six pots'\\

\ex. dàːdúː\\
pot-five\\
`five pots'\\


\ex. tɛ̂tɛ̂\\
    `spider'\\

\ex. tɛ̂tɛ̂dɔ́ndɔ̀̆wɛ̂\\
spider-one\\
`one spider' \jf{(Per Tony, this is the full way of saying "one \_\_\_\_")}\\

\ex. tɛ̂tɛ̂dɔ́ndɔ̀\\
spider-one\\
`one spider' \\

\ex. tɛ̂tɛ̂féà\\
spider-two\\
`two spiders'\\
    
\ex. tɛ̂ː\\
    `chicken'\\

 \ex. tɛ̂ːdɔ́ndɔ̀\\
 chicken-one\\
`one chicken'\\

\ex. tɛ̂ːdú\\
chicken-five\\
`five chickens'\\

\ex. tɛ̂ːɔ́nféà\\
chicken-seven\\
`seven chickens'\\

\ex. dɛ̂\\
`mother' \jf{(This also provides a different semantic category from the previous two)}\\

 \ex. dɛ́dɔ́ndɔ̂\\
  mother-one\\
`one mother'\\

 \ex. dɛ̂féà\\
  mother-two\\
`two mothers'\\

 \ex. ndɛ̂\\
  my-mother\\
`my mother'\\

 \ex. ídɛ̂\\
  your-mother\\
`your mother'\\

 \ex. îdɛ̀nû\\
  your-mother-PL\\
`your mothers'\\


\jf{Tone Contrast:}

\exg. kònɛ̄\\
tree\\

\exg. kôndɔ́ndɔ̀\\
  tree-one\\
one tree\\



[\section{Plural nouns, plural N possessors -- Jan} 

\ex. \textipa{tEtS\'i\'i} \\
`egg' 

\ex. \textipa{tEtS\'i\'in\`u} \\
`eggs' 

\ex. \textipa{n\'a\`a tEtS\'i\'in\`u} \\
`my eggs' 

\ex. \textipa{kae a tEtS\'i\'in\`u} \\
`Kai's eggs' 

\ex. \textipa{ana t\`etS\'i\'i} $\sim$ \textipa{anat\^etS\'e\`en\`u}\\
`Their (Kai and Sahr's) eggs' 

\jmt{This is curious - the plural morpheme may be optional?}

\ex. \textipa{k\`u\textltailn\`e} \\
`taste' (of a dish; smell) 

\ex. \textipa{k\'a\`e a tE tS\'i\`i ku\textltailn\'e\`ene} \\
`Kai's eggs' smell'

\ex. \textipa{ana tE tS\'i\`i ku\textltailn\'e\`ene} \\
`Their eggs' smell' 

\ex. màà \\
`banana'

\ex. màà nu \\
`bananas'

\ex. na máà nù \\
`my bananas'

\ex. kae a máà nù \\
`Kai's bananas'

\exg. ana maa nu $\sim$ amaanu \\
3-PL-POSS banana PL
`Their bananas'

\jmt{There seems to be some  additonal variability in the first person marker?}

\ex. ana maa ku\textipa{\textltailn}éèle \\
3-PL-POSS banana smell
`Their bananas' smell'

\ex. táà nù \\
`calabashes' 

\ex. ná táá nù \\
1SG-POSS calabash PL
`my calabashes'

\ex. táà ku\textipa{\textltailn}éènè nu \\
`my calabashes' smell'

\ex. \textipa{s\'E\'E} \\
instrument

\ex. \textipa{s\'E\'E} nù \\
instrument (pl)

\ex. \textipa{a s\'E\'E n\`u} \\
3SG-POSS see PL
`His instrument'

\exg. \textipa{ka\'e a  s\'E\'E eawa n\`u} \\
Kai ? instrument red PL \\
`Kai's red instruments'
\jmt{The intended meaning was not elicited properly by me.}

\jmt{Takeaways: it seems [nù] is the plural marker for nouns, and I could not hear a change in tone based on its presence. }



\section{Complex NPs (combining multiple elements) -- Alex} 

\exg.
dà    \\
mouth \\
`mouth'

\exg.
dà      tʃɛ̀  \\
mouth   this \\%
`this mouth'

\exg.
dà      nàmà \\
mouth   new  \\%
`new mouth'

\exg.
dà      nàmà   tʃɛ̀  \\
mouth   new    this \\%
`this new mouth'

\exg.
dà      wà  \\
mouth   big \\%
`big mouth'

\exg.
dà      wà    tʃɛ̀  \\
mouth   big   this \\%
`this big mouth'

\exg.
dà      tʃɛ́nà   mà \\
mouth   big     ?  \\%
`big mouth'

\exg.
dà      tʃɛ́nà   mà   tʃɛ̀  \\
mouth   big     ?    this \\%
`this big mouth'

\exg.
dà      dùù  \\
mouth   five \\%
`five mouths'

\exg.
dà      dùù    tʃɛ́nàmà \\
mouth   five   big     \\%
`five big mouths'

\exg.
dà      dùù    tʃɛ́nàmà-nù \\
mouth   five   big-PL     \\%
`five big mouths'

\exg.
dà      dùù    wa  \\
mouth   five   big \\%
`five big mouths'

\exg.
dà      dùù    wa-nù  \\
mouth   five   big-PL \\%
`five big mouths'

\exg.
dà      dùù    tʃɛ́nàmà   tʃɛ̀-nù  \\
mouth   five   big       this-PL \\%
`these five big mouths'

\exg.
dà      dùù    wa    tʃɛ̀    nù \\
mouth   five   big   this   PL \\%
`these five big mouths'

\alex{plural marker appears necessary here. At first, Tony says ``it doesn't sound natural without -nu''}

\exg.
dà      dùù    wa    tʃɛ̀  \\
mouth   five   big   this \\%
`these five big mouths' \label{proximal det}

\alex{Tony says that without the plural marker /-nu/, as in \ref{proximal det} the context needs to be such that the referent is near to the speaker;
  perhaps this is a proximal/distal distinction with respect to the form of determiners?}

\exg.
dàà   tʃɛ̀  \\
pot   this \\%
`this pot'

\exg.
dàà   dùù  \\
pot   five \\%
`five pots'

\exg.
dàà   duu    tʃɛ̀  \\
pot   five   this \\%
`these five pots'

\exg.
dàà   duu    tʃɛ̀-nu  \\
pot   five   this-PL \\%
`these five pots'

\exg.
dàà   wà  \\
pot   big \\%
`big pot'

\exg.
dàà   dúú    wà  \\
pot   five   big \\%
`five big pots'

\exg.
dàà   dúú    wá-nù  \\
pot   five   big-PL \\%
`five big pots'

\exg.
dàà   dúú    wá    tʃɛ̀    nù \\
pot   five   big   this   PL \\%
`these five big pots'

\exg.
tú  \\
oil \\%
`oil'

\exg.
tú    tʃɛ̀  \\
oil   this \\%
`this oil'

\exg.
tú    wá  \\
oil   big \\%
`big oil'

\exg.
tú    wá    tʃɛ̀  \\
oil   big   this \\%
`this big oil'

\exg.
tú    jáwà \\
oil   red  \\%
`red oil'

\exg.
tú    jáwà   tʃɛ̀  \\
oil   red    this \\%
`this red oil'

Ordering of elements within a complex NP, PSR style:

\ex.
\a. NP -> N 
\b. NP -> N Det
\c. NP -> N Adj
\d. NP -> N Adj Det
\e. NP -> N Num
\f. NP -> N Num Adj
\f. NP -> N Num Adj PL
\f. NP -> N Num Adj Det
\f. NP -> N (Num) (Adj) (Det) (PL)

Unattested: whether you can get a pronoun with the determiner:

\ex. Pron N (X) Det (unattested)



\section{Complex Possessors -- Wesley} 

\exg. ǹ̩-dɛ́\\
\textsc{1sg}-mother\\
`my mother'

\exg. ǹ̩-dɛ́ à g͡bòò\\
\textsc{1sg}-mother \textsc{3sg.poss} book\\
`my mother’s book'

\exg. ǹ̩-dɛ́ g͡bòò\\
\textsc{1sg}-mother arm\\
`my mother’s arm'

\wml{It seems like à is an alienable possession marker. It is \textbf{not} used when the possessum is a family member or body part. Per Anthony's description, you don't need à when ``it's not detached'' from the possessor.}

\wml{For now, we seem to only have evidence of à used with \textsc{3sg} possessors, but the other possessive determiners seem to combine with à, e.g., \textsc{2pl.poss} is \'{o} for inalienable possessa but w\'{a} for alienable ones; could w\'{a} be \'{o}+\`{a}?}

\exg. ǹ̩-dé à t͡ʃènɛ́\\
\textsc{1sg}-mother \textsc{3sg.poss} house\\
`my mother’s house'

\exg. ǹ̩-dɛ́ t͡ʃènɛ́\\
\textsc{1sg}-mother leg\\
`my mother’s leg'

\exg. màmà (mùsù)\\
grandparent woman\\
`grandmother'

\exg. ǹ̩-dé dè\\
\textsc{1sg}-mother mother\\
`my mother's mother'

\exg. \textsc{1sg}-mother beautiful\\
ǹ̩-dé ɲìmà\\
`my beautiful mother'

\exg. ǹ̩-dé ɲímà à t͡ʃénɛ̀\\
\textsc{1sg}-mother beautiful \textsc{3sg.poss} house\\
my beautiful mother’s house

\exg. ǹ̩-dé à t͡ʃé ɲìmà\\
\textsc{1sg}-mother \textsc{3sg.poss} house beautiful\\
my mother’s beautiful house

\wml{Prompt was `my beautiful mother’s house’ but Anthony clarified that `beautiful’ in this phrase describes the house, not the mother.}
\wml{Given the elicitation of `that tall woman' vs. `that woman is tall', I want to confirm whether the adjective in  [ǹ̩-dé à t͡ʃé ɲìmà] is attributive or predicative.}

\exg. mùsù t͡ʃɛ̀\\
woman this\\
`this woman’

\ex. jã̀sã̀\\
`tall'

\wml{I think I hear nasalised vowels here---but in other forms, there's an oral vowel + nasal consonant.}

\exg. mùsù jã̀sàmá t͡ʃɛ̀\\
woman tall this\\
`this tall woman’

\exg. mùsù t͡ʃɛ̀ jã̀sã̀\\
woman this tall\\
`This woman is tall.’

\exg. mùsù jánsámá t͡ʃɛ̀ à t͡ʃénɛ̀\\
woman tall this \textsc{3sg.poss} house\\
`this tall woman’s house’

\exg. mùsù jánsámá t͡ʃɛ̀ já\\
woman tall this eye\\
`this tall woman’s eye’

\exg. mwà úú\\
\textsc{1INCL.DU-PST} dog\\
`our (\textsc{2du.incl}) dog’

\exg. mwà úú à dɔ̃ fènɛ́\\
\textsc{1INCL.DU-PST} dog \textsc{3sg.poss} eat thing\\
`our (\textsc{2du.incl}) dog’s food'

\ex. kánɛ́\\
`to learn’, `to teach’

\ex. mwɛ́ɛ́\\
`person’

\exg. kà-mó\\
teach-person\\
`teacher’\\
\wml{Anthony says you don’t say *kà-mwɛ́ɛ́.}

\exg. j-á kàmó\\
\textsc{2SG-POSS} teacher\\
`your (\textsc{sg.}) teacher’

\exg. j-á kámó pádì\\
\textsc{2SG-POSS} teacher friend\\
`your (\textsc{sg.}) teacher’s friend’

\exg. j-á kámó pádì à bwíí\\
\textsc{2SG.POSS} teacher friend \textsc{3sg.poss} medicine\\
`your teacher’s friend’s medicine’

\newpage 

\end{document}
