\documentclass{assets/fieldnotes}

\title{Kono (Sierra Leone)}
\author{LING3020/5020}
\date{University of Pennsylvania, Spring 2025\\02/26/2025 Syntax I}

\setcounter{secnumdepth}{4} %enable \paragraph -- for subsubsubsections

\begin{document}

\maketitle
\tableofcontents


\section{Adjectival Predicates (Wesley)}

\wml{Adjectival predicates like `tall' take Series 1 agreement marking.}

\exg. ŋ̩́-jã́sã̀\\
\textsc{1SG}-tall\\
`I am tall.’

\exg. mɔ̀-jã́sã̀\\
\textsc{1INCL.DU}-tall\\
`We (\textsc{du.incl}) are tall.’

\exg. mɔ́-jã́sã̀\\
\textsc{1INCL}-tall\\
`We (\textsc{pl.incl}) are tall.’

\exg. ŋ̩̀-jã́sã̀\\
\textsc{1EXCL}-tall\\
`We (\textsc{pl.excl}) are tall.’

\exg. í-jã́sã̀\\
\textsc{2SG}-tall\\
`You (\textsc{sg}) are tall.’

\exg. wó-jã́sã̀\\
\textsc{2PL}-tall\\
`You (\textsc{pl}) are tall.’

\exg. à-jã̀sã̀\\
\textsc{3SG}-tall\\
`He/She is tall.’

\exg. ã̀-jã̌sã̀\\
\textsc{3PL}-tall\\
`They are tall.’

\exg. ã̀-jã̌sã̀ ka-ka\\
\textsc{3PL}-tall very-\textsc{red}\\
`They are very tall.’

\wml{

In the following sentences, I elicited NP predicates with adjectival modification, i.e., `X is/are tall women'. These prompts elicited what seem to be two patterns using either Series 1 or Series 2 marking in different positions:
\begin{itemize}
    \item `Series 1 pattern': Series 1 + \textit{mu} + NP + \textit{\`{a}n\`{a}}
    \item `Series 2 pattern': NP + \textit{mu} + Series 2
\end{itemize}
}

\wml{Daniel's sentences (43--44) from February 19 seem to represent the same alternation with nominal subjects. His (45--51) also show the Series 1 pattern, although the \textit{ana} was obscured due to a preceding [a] in some of them.\\}

\wml{I wonder if the two structures represent differences in information structure. I've rearranged the following data to show the two patterns, so these don't follow the elicitation sequence.\\}

\wml{Series 1 pattern: Series 1 + \textit{mu} + NP + \textit{\`{a}n\`{a}}}

\exg. à-m mú mùsù jã́sá-má ànà \\
\textsc{3-PL} \textsc{mu} woman tall-\textsc{attr} \textsc{ana}\\
`They are tall women.’

\exg. à-ŋ g͡bɛ́ mú mùsù jã́sá-má ànà\\
\textsc{3-PL} all \textsc{mu} woman tall-\textsc{attr} \textsc{ana}\\
`They are all tall women.’

\exg. wó mú mùsù jã́sá-má ànà\\
\textsc{2PL} \textsc{mu} woman tall-\textsc{attr} \textsc{ana}\\
`You (pl.) are tall women.’

\exg. í mú mùsù jã́sá-má ànà\\
\textsc{2SG} \textsc{mu} woman tall-\textsc{attr} \textsc{ana}\\
`You (sg.) are a tall woman.’

\exg. mú mùsù jã́sá-má àná\\
\textsc{mu} woman tall-\textsc{attr} \textsc{ana}\\
`I am a tall woman.’\\
\wml{The Series 1 \textsc{1sg} form is N̩, so this could possibly be m̩mú.}

\wml{Series 2 pattern': NP + \textit{mu} + Series 2}

\exg. mùsù jã́sá-má mù à-n-á\\
woman tall-\textsc{attr} \textsc{mu} \textsc{3-PL-POSS}\\
`They are tall women.’

\exg. mùsù jã́sá-má mù à-ŋ g͡bɛ́ á\\
woman tall-\textsc{attr} \textsc{mu} \textsc{3-PL} all ?\\
`They are all tall women.’\\
\wml{The form [àŋ g͡bɛ́ á] here for `they all', compared to àná for `they', supports that àná may have more complex internal structure like [ã̀ + á] such that g͡bɛ́ can attach to ã̀ first before á does. Or it could be infixation, but I have no idea if infixation is available elsewhere in Kono.}

\exg. mùsù jã́sá-má mw-à\\
woman tall-\textsc{attr} \textsc{mu}-\textsc{3sg}\\
`She is a tall woman.’

\exg. mùsù jã́sá-má mɔ̀-wá\\
woman tall-\textsc{attr} \textsc{mu}-\textsc{2pl}\\
`You (pl.) are tall women.’

\exg. mùsù jã́sá-má mù-já\\
\textsc{2sg} ? woman tall-\textsc{attr} \textsc{mu}-\textsc{2SG.PST}\\
`You (sg.) are a tall woman.’

\exg. mùsù jã́sá-má mù-ná\\
woman tall-\textsc{attr} \textsc{mu}-\textsc{1sg}\\
`I am a tall woman.’

\wml{With demonstrative pronoun subjects, we get the pattern NP + \textit{ane} + \textsc{dem}. I wonder if the \textit{ane} morpheme is related to \textit{ana} in the `Series 1' \textit{mu} structure. Interestingly no \textit{mu} in these sentences...}

\exg. mùsù jã́sá-má ànè tʃà \\
woman tall-\textsc{attr} \textsc{ane} this\\
`This is a tall woman.’

\exg. mùsù jã́sá-má ànè tʃɛ̀-ná\\
woman tall-\textsc{attr} \textsc{ane} this-\textsc{pl}\\
`These are tall women.’

\wml{More stage-level adjectives that ended up being body-part idioms:}

\exg. ɱ̩́-fàà té-à\\
\textsc{1SG}-heart ?-?\\
`I am angry.’\\
\wml{I wonder if this might mean something like, `My heart is breaking,’ but used to express anger rather than sadness as in English.}

\exg. fàà té\\
heart ?\\
`(be) angry’
    
\exg. tʃí ándá\\
? ?\\
`(be) afraid’, `fear’\\
\wml{Later on we were told that tʃí means mind; I wonder if `afraid’ is an idiomatic predicate involving `mind’.}

\exg. tʃí ánè\\
? ?\\
`(be) afraid’, `fear’

\exg. ń̩-tʃí ándá\\
\textsc{1SG}-? ?\\
`I am afraid.’

\exg. ń̩-tʃí ándéɛ̀\\
\textsc{1SG}-? ?\\
`I am afraid.’

\exg. ń̩-tʃí ándéɛ̀ mù\\
\textsc{1SG}-? ? \textsc{cop?}\\
`I am afraid.’

\ex. pàndí\\
`sick’

\exg. ḿ̩-pándíné mù\\
\textsc{1SG}-sick \textsc{cop?}\\
`I am sick.’

\section{Transitive verbal predicates: ɔ-series objects (Giang)}  \jal{notice this exists from dissertation and previous notes}

\g{Previous findings from Foday-Ngongou: Many verbs have \`a as their 3rd person singular object pronoun, but some have \`ɔ. These verbs retain \`ɔ even when a noun object is used (pp. 228-229). (I wonder if it retains the \`ɔ when the object is a name.)}


\begin{table}
    \centering
    \begin{tabular}{ccccc}
        \toprule
        & Singular & \multicolumn{3}{c}{Plural} \\
        \toprule 
        & & 1, 2 & 1, 2, 3 & 1, 3 \\  
        \cmidrule(r){3-3}
        \cmidrule(r){4-4} 
        \cmidrule(r){5-5}
        1 & Ń̩dɔ́ & mɔ̀ & mɔ̂ & Ǹdɔ̀ \\
        \midrule
        2 & jɔ́ & \multicolumn{3}{c}{wɔ́} \\
        \midrule
        3 & ɔ́ & \multicolumn{3}{c}{Ń̩dɔ́}\\
        \bottomrule
    \end{tabular}
    \caption{ɔ-series object pronouns}
    \label{tab:fut}
\end{table} 

\wml{I've tabulated the ɔ-series object pronouns above. The elicitation using \textit{tisa} `ask' went great since we could use objects of all persons and numbers.\\}

\wml{I noticed that the citation form for `break' was \textit{t\'{e}-\`{a}} but in transitive sentences, the verb surfaced as \textit{t\'{e}}. Only in unaccusatives like \textit{the sɛɛ broke} did we get \textit{t\'{e}-\`{a}}. I don't think we saw this alternation with Alex's fall/drop sentences last week (that could be a intransitive-causative alternation anyway).}

\ex. tɛ̀ɛ̀\\
`to cut up’

\ex. káj\\
`to break (into two pieces)’

\exg. té-à\\
break-?\\
`to break (into small pieces)’

\ex. sɛ̀ɛ̀\\
`a Kono musical instrument’

\exg. n-á sɛ̀ɛ̀ ɔ̀ té\\
\textsc{1SG.PST} instrument \textsc{obj} break\\
‘I broke the sɛɛ.’

\exg. n-á sɛ̀ɛ̀ ǹ̩dɔ́ té\\
\textsc{1SG.PST} instrument \textsc{3pl.obj} break\\
‘I broke the sɛɛs.’

\exg. sɛ̀ɛ̀ ɔ̀ té-à\\
instrument \textsc{3sg.obj} break-?\\
‘The sɛɛ broke.’

\exg. sɛ̀ɛ̀ ǹ̩dɔ́ té-à\\
instrument \textsc{3pl.obj} break-?\\
‘The sɛɛs broke.’

\ex. tísá\\
`ask’

\exg. n-á kai ɔ̀ tísá\\
\textsc{1SG.PST} Kai \textsc{3sg.obj} ask\\
`I asked Kai.’

\exg. n-ɔ́ ɔ̀ tísá\\
\textsc{1SG-AUX.PST} \textsc{3sg.obj} ask\\
`I asked him.’\\
\wml{[nɔ́] seems to be the Series 2 \textsc{1sg} form but undergoing some vowel harmony when adjacent to [ɔ̀]. Anthony says it’s not possible to say [*ná nɔ́ ɔ̀ tísá] but I should really have asked about [ná ɔ̀ tísá].}

\exg. n-ɔ́ ɔ̀ tísá\\
\textsc{1SG-AUX.PST} \textsc{3sg.obj} ask\\
`I asked him.’

\exg. n-á ǹ̩dɔ̀ tísá\\
\textsc{1SG.PST} \textsc{3pl.obj} ask\\
`I asked them.’

\exg. n-á jɔ́ tísá\\
\textsc{1SG.PST} \textsc{2sg.obj} ask\\
`I asked you (\textsc{sg}).’

\exg. n-á wɔ̀ tísá\\
\textsc{1SG.PST} \textsc{2pl.obj} ask\\
`I asked you (\textsc{pl}).’

\exg. ǎ dɔ́/nɔ́/ń̩dɔ́ tísá\\
\textsc{3SG.PST} \textsc{1sg} ask\\
`He asked me.’\\
\wml{On different occasions, Anthony gave us dɔ́, nɔ́, and ndɔ́ for the \textsc{1sg} object marker.}

\exg. ǎ mɔ̀ tísá\\
\textsc{3SG.PST} \textsc{2du.incl.obj} ask\\
`He asked us (\textsc{du.incl}).’

\wml{With a proper name as the subject, Anthony gave us [já] which looks like the \textsc{2sg} Series 2 form.}

\exg. Alex j-á mɔ̀ tísá\\
Alex 2SG-PST \textsc{2du.incl.obj} ask\\
`Alex asked us (\textsc{du.incl}).’

\exg. Alex j-á mɔ̂ tísá\\
Alex 2SG-PST \textsc{2pl.obj} ask\\
`Alex asked us (\textsc{pl}).’

\exg. à ǹ̩dɔ́ tísá\\
3SG.PST \textsc{2du.excl.obj} ask\\
`He asked us (\textsc{du.excl}).'

\exg. á mùsù ǹ̩dɔ́ tísá\\
\textsc{3SG.PST} woman \textsc{3pl.obj} ask\\
`He asked the women.’

\exg. j-á ń̩dɔ́ tísá\\
\textsc{2SG-PST} \textsc{1sg.obj} ask\\
`You asked me.’



\section{Ditransitives: give (Lex)}


\exg. swéè\\
meat\\
`meat'


\exg. à swéè bè Bondu má\\
3SG.PST meat give Bondu P\\
`He gave the meat to Bondu' 

\exg.  à swéè bè Bondu má\\
3SG.PST meat give Bondu P\\
`He gave Bondu the meat'

\exg. à swéè bè à mà\\
3SG.PST meat give 3SG.OBJ P\\
`He gave him the meat.'

\exg. à swéè bè ǹ màà\\
3SG.PST meat give 1PL.INCL.DU.OBJ P\\
`He gave us (incl.dual) the meat.'

\exg. à swéè bè ǹ mɔ̀ mà/ à swéè bè mɔ̀ mà\\
3SG.PST meat give 1PL.INCL.OBJ P\\
`He gave us (incl.pl) the meat.'

\wml{ Anthony varied between including the nasal [ǹ] and dropping it}


\exg. à swéè bè màà\\
3SG.PST meat give 1.PL.EXCL.OBJ P\\
`He gave us (excl.pl) the meat.'

\wml{ Not sure but the vowel sounds like there is vowel lengthening here on [mà] as well}

\exg. à swéè bè í má\\
3SG.PST meat give 2SG.OBJ P\\
He gave you the meat.

\exg. à swéè bè àn mà\\
3SG.PST meat give 3PL.OBJ P\\
`He gave them the meat.'

\wml{ Believe this [mà]  has a low tone for the 3PL. and a high tone for the 2PL. in contrast to last year's data}
\wml{ In recording, Anythony repeats something like [nú] when explaining difference between 2Pl. and 3Pl., think it was just an example of difference}
\wml{ When Anthony breaks up sentences, sometimes give the [ma] a high tone when saying it isolated with the indirect object pronoun, when in the sentence it has a low tone}


\exg. à swéè bè m-má \\
3SG.PST meat give 1Sg.OBJ P\\
He gave me the meat. 

\exg. à swéè bè wó má\\
3SG.PST meat give 2Pl.OBJ P\\
`He gave you all (pl.) the meat.'


\exg. à swéè másɔ̀n Bandu ǹ téjá / à swéè másɔ̀n Bondu am boo\\
3SG.PST meat get Bandu P {} / 3SG.PST meat get Bondu P/FOC? hand\\
`He got the meat from Bandu.'

\wml{Is the tone on hand low with a lengthened vowel or no tone? Also, should it be transcribed with an open o? Conflicting data}



\section{Ditransitives: put (Jan)} %we have this from last time in past

\jal{remember we're going to use Bondu as our name now, since it's easier to hear any following /a/ morpheme}

\ex. dà \\
`pot'

\ex. kw\textipa{\'E} \\
`rice'

\exg. bóndù á-à kwɛ́ɛ́ tʃɛ̀ dà-ò \\
bondu AUX.PST rice put pot-inside \\
`Bondu put the rice in the pot.'

\ex. jògbá \\
`potato leaf'

\exg. bóndù á-à jògbá tʃɛ̀ɛ̀ dà-ò \\
bondu AUX.PST potato.leaf put pot-inside \\
`Bondu put the potato leaf in the pot.'

\exg. bóndù á-à fènè tʃɛ̀ɛ̀ dà-ò \\
bondu AUX.PST thing put pot-inside \\
`Bondu put that in the pot.'

\exg. bóndù á-à jògbá tʃɛ̀ɛ̀ nè \\
bondu AUX.PST potato.leaf put here \\
`Bondu put the potato here.'

\exg. n-á jògbá tʃè dá-ò \\
1SG.PST potato.leaf put pot \\
`I put the potato in the pot.'

\exg. j-á jògbá tʃè dá-ò \\
2SG-PST potato.leaf put pot-inside \\
`You (sg) put the potato in the pot.'

\exg. á-á jògbá tʃè dá-ò \\
3SG.PST potato.leaf put pot-inside \\
`He/She put the potato in the pot.'

\exg. ná-à jògbá tʃè dá-ò \\
1EXCL-PST potato.leaf put pot-inside \\
`We (ex, pl) put the potato in the pot.'

\exg. mwã̀ jògbá tʃè dá-ò \\
1INCl.PST potato.leaf put pot-inside \\
`We (inc, pl) put the potato in the pot.'

\exg. mwã́ jògbá tʃè dá-ò \\
1INCL.DU-PST potato.leaf put pot-inside \\
`We (inc, dual) put the potato in the pot.'

\exg. w-á jògbá tʃè dá-ò \\
2PL-PST potato.leaf put pot-inside \\
`You (pl) put the potato in the pot.'

\exg. à-n-á jògbá tʃè dá-ò \\
3-PL-PST potato.leaf put pot-inside \\
`They put the potato in the pot.'

\jmt{The elicitation notes from before suggest there are two distinct verbs for putting something \textit{on} and \textit{in} something else. }

\exg. gbó éámbà \\
book leaf \\
`paper'

\ex. gbò \\
`book'

\ex. féɲéènfénè \\
`pen'

\ex. pɛ̀ɛ̀nɛ̀ \\
`pen'

\exg. bóndù à pɛ̀ɛ̀nɛ̀ sà bóó mà \\
bondu AUX.PST pen put book on.top \\
`Bondu put the pen on the book.'

\exg. bóndù à pɛ̀ɛ̀nɛ̀ sà bù ó \\
bondu AUX.PST pen put book inside \\
`Bondu put the pen inside the book'

\exg. bóndù à gboo sá pɛɛnɛ ma \\
bondu AUX.PST book put pen on.top \\
`Bondu put the book on top of the pen.'

\ex. dúúmáá \\
`floor'

\exg. n-á pɛ̀ɛ̀nɛ̀ sà dúú mà \\
1SG.PST pen put floor on.top \\
`I put the book on the floor.'

\exg. bóndù à pɛ̀ɛ̀nɛ̀ sà dúú mà \\
Bondu AUX.PST pen put floor on.top \\
`Bondu put the book on the floor.'

\ex. dúú \\
`steam yourself'

\ex. dùú \\
`town'



\section{Ditransitives: beneficiary, etc (Mingyang)} \label{Ditransitives: beneficiary, etc (Mingyang)} %we have this from last time in past
%\mb{Building on the 2023 dataset, I'm trying benefactive `cook' in future.} \jal{perhaps get a couple past too to verify}
%\mb{OK, got it!}
\ex. tàŋgùmbà\\
    `cassava leaf'

\ex. táwá\\
    `cook'

\ex. síná\\
    `tomorrow'

\ex. kúnú\\
    `yesterday'

\exg. ḿ-bé tàŋgùmbà táwá wã síná.\\
    1SG-NPST cassava.leaves cook FUT tomorrow\\
    I will cook cassava leaves tomorrow.

\exg. ḿ-bé tàŋgùmbà táwá wã í jé síná.\\
    1SG-NPST cassava.leaves cook FUT 2.SG for tomorrow\\
    I will cook cassava leaves for you(sg.) tomorrow.

\exg. ḿ-bé tàŋgùmbà táwá wã wó jé síná.\\
    1SG-NPST cassava.leaves cook FUT 2.PL for tomorrow\\
    I will cook cassava leaves for you(pl.) tomorrow.

\exg. n-á tàŋgùmbà táwá wó jé kúnú.\\
    1SG-AUX.PST cassava.leaves cook 2.PL for yesterday\\
    I cooked cassava leaves for you(pl.) yesterday.

\exg. ḿ-bé tàŋgùmbà táwá wã à jé síná.\\
    1SG-NPST cassava.leaves cook FUT 3.SG for tomorrow\\
    I will cook cassava leaves for her/him tomorrow.

\exg. n-á tàŋgùmbà táwá à jé kúnú.\\
    1SG-AUX.PST cassava.leaves cook 3.SG for yesterday\\
    I cooked cassava leaves for her/him yesterday.

\exg. ḿ-bé tàŋgùmbà táwá wã à̃ gé síná.\\
    1SG-NPST cassava.leaves cook FUT 3.PL for tomorrow\\
    I will cook cassava leaves for them tomorrow. \label{3PL beneficiary}

\exg. ḿ-bé tàŋgùmbà táwá wã Ǹ gé síná.\\
    1SG-NPST cassava.leaves cook FUT 2.PL.EXCL for tomorrow\\
    I will cook cassava leaves for us(excl.) tomorrow.

\exg. ḿ-bé tàŋgùmbà táwá wã mɔ̀ jé síná.\\
    1SG-NPST cassava.leaves cook FUT 2.DU.INCL for tomorrow\\
    I will cook cassava leaves for us(incl., du.) tomorrow.

\exg. ḿ-bé tàŋgùmbà táwá wã mɔ́ jé síná.\\
    1SG-NPST cassava.leaves cook FUT 1.PL.INCL for tomorrow\\
    I will cook cassava leaves for us(incl., pl.) tomorrow.

\ex. kàmwɛ̂\\
    `teacher'

\exg. ḿ-bé tàŋgùmbà táwá wã kàmwɛ́ jè síná.\\
    1SG-NPST cassava.leaves cook FUT teacher for tomorrow\\
    I will cook cassava leaves for the teacher tomorrow.

\exg. ḿ-bé tàŋgùmbà táwá wã Bòndú jè síná.\\
    1SG-NPST cassava.leaves cook FUT Bondu for tomorrow\\
    I will cook cassava leaves for Bondu tomorrow.

\mb{Mingyang:
\begin{itemize}
    \item For subjects, we have Series 3 pronouns for Future and Series 2 pronouns for Past. This follows our previous generalizations.
    \item For indirect objects, we seem to have Series 1 pronouns.
    \item The Future tense marker \textit{wã} sounds like having a nasalized vowel when produced naturally in sentences but sounds more like \textit{wan} when produced separately.
    \item When the preceding segment is [+nasal] (e.g., recording ~49:23), the consonant in the postposition \textit{jé} turns into something velar. I put [g] there, but it could be something else. 
    \item When the indirect object is a common noun or a name, the post-position \textit{je} seems to have a low tone instead of a high tone.
\end{itemize}
}


\section{Causatives - unacc/transitive pairs (Alex)} % Alex

\exg.
n-á    taa        o     te    \\
1SG.PST   calabash   OBJ   break \\
`I broke the calabash.'

\exg.
n-á    taa        o     te      (*-a) \\
1SG.PST   calabash   OBJ   break   (*-A) \\
`I broke the calabash.'

\exg.
n-á    taa        a     te    \\
1SG.PST   calabash   OBJ   break \\
`I broke the calabash.' \label{I broke the calabash}

\alex{Tony gave another variant shown in \ref{I broke the calabash}. Here, I suppose that either the o- object marker is optional, or the a-series object marker is used. Unsure which is the case here. Need to investigate vowel length of /taa/ `calabash' in these contexts.}

\exg.
n-á    taa        ndo      te    \\
1SG.PST   calabash   PL.OBJ   break \\
`I broke the calabashes.'

\exg.
m-bé   taa        o     te-a    \\
1SG-NPST  calabash   OBJ   break-A \\
`I am breaking the calabash.'

\exg.
m-bé   taa        ndo      te-a    \\
1SG-NPST   calabash   PL.OBJ   break-A \\
`I am breaking the calabashes.'

\exg.
taa        o     te-a    \\
calabash   OBJ   break-A \\
`The calabash broke.' \label{The calabash broke}

\exg.
taa        ndo      te-a    \\
calabash   PL.OBJ   break-A \\
`The calabashes broke.' \label{The calabashes broke}

\exg.
taa        oo    te-a    \\
calabash   OBJ   break-A \\
`The calabash is breaking.' \label{The calabash is breaking}

\exg.
taa        ndoo     te-a    \\
calabash   PL.OBJ   break-A \\
`The calabashes are breaking.' \label{The calabashes are breaking}

\alex{In the anticausative constructions, there is a difference in what appears to be vowel length of the object marker. In the ``past'' anticausative, we see a short /o-/ object marker with both singular objects, as in \ref{The calabash broke}, and plural objects, as in \ref{The calabashes broke}. In the ``non-past'' anticausative, we see a long /o-/ object marker with both singular objects, as in \ref{The calabash is breaking}, and plural objects, as in \ref{The calabashes are breaking}. I suspect this vowel lengthening might be due to (the presence of) the AUX on the subject?}

\alex{Did not check whether the /-o/ object marker is optional/can be swapped with /-a/ in the anticausative constructions.}

\exg.
n-á    gbéŋgbɛ̀ɛ   a     tune \\
1SG.PST   boat       OBJ   sink \\
`I sank the boat.'

\exg.
n-á    gbéŋgbɛ̀ɛ   n    a     tune \\
1SG.PST   boat       PL   OBJ   sink \\
`I sank the boats.'

\exg.
mbé   gbéŋgbɛ̀ɛ   a     tune-a \\
1SG.NPST  boat       OBJ   sink-A \\
`I am sinking the boat.'

\exg.
mbé   gbéŋgbɛ̀ɛ   -n    -a     tune-a \\
1SG.NPST   boat       -PL   -OBJ   sink-A \\
`I am sinking the boats.'

\exg.
gbéŋgbɛ̀ɛ   -a     tune-a \\
boat       -OBJ   sink-A \\
`The boat sank.'

\exg.
gbéŋgbɛ̀ɛ   -n    -a     tune-a \\
boat       -PL   -OBJ   sink-A \\
`The boats sank.'

\exg.
gbéŋgbɛ̀ɛ   (-e)     -a     tune-a \\
boat       (-AUX)   -OBJ   sink-A \\
`The boat is sinking.' \label{The boat is sinking}

\exg.
gbéŋgbɛ̀ɛ   -n    -e     -a     tune-a \\
boat       -PL   -AUX   -OBJ   sink-A \\
`The boats are sinking.' \label{The boats are sinking}

\alex{In the anticausative construction in the non-past, it seems there is an additional morpheme after the grammatical subject and before the object marker (i.e., our auxiliary). It appears that with a singular grammatical subject (i.e., no plural morpheme), as in \ref{The boat is sinking}, it is difficult to detect, even for Tony. However, note the clear distinction observed in the plural constructions, as in \ref{The boats are sinking}. My hypothesis is that this /-e/ is the same AUX morpheme we see in the non-past pronominal forms (the AUX fuses with the pronominal, or maybe there's agreement or sth). To determine whether there is indeed a /-e/ AUX morpheme/vowel present in a construction like \ref{The boat is sinking}, we can change the surface subject to a noun that does end with /ɛ/ or /e/ (see examples below with `door').}

% # [CLOSE]

\exg.
n-á    tʃɛna   da      a     tõ    \\
1SG.PST   house   mouth   OBJ   close \\%
`I closed the door.'

\exg.
n-á    tʃɛna   a     tõ    \\
1SG.PST   house   OBJ   close \\%
`I closed the door.' \label{I closed the door}

\exg.
mbé       tʃɛna   da      a     tond-a  \\
1SG.NPST   house   mouth   OBJ   close-A \\
`I am closing the door.' \label{I am closing the door}

\alex{There seems to be a generalization regarding word-final nasals:
  if they occur word finally, they get realized as nasalization on the preceding vowel;
  if a vowel occurs after (e.g., a suffix), not only is the nasal realized as [n] but it appears that a [d] is (epenthetically?) inserted.
  We see this pattern in verbs like `close' (cf. /t\textbf{õ}/ in \ref{I closed the door} with /to\textbf{nd}-a/ in \ref{I am closing the door}). We also see this in object markers following a plural noun, shown in \ref{The calabash broke} and \ref{The calabashes broke}.}

\exg.
tʃɛna   da      a     tond-a  \\
house   mouth   OBJ   close-A \\%
`The door closed.'

\exg.
tʃɛna   da      -e     -a     tond-a  \\
house   mouth   -AUX   -OBJ   close-A \\%
`The door is closing.' \label{The door is closing}

\exg.
tʃɛna   daa (ɛ)    -n    -e     -a     tond-a  \\
house   mouth (?)  -PL   -AUX   -OBJ   close-A \\%
`The doors are closing.' \label{The doors are closing}

\alex{Note that the auxiliary /-e/ surfaces much more clearly with the singular grammatical subject in the anticausative non-past form in \ref{The door is closing}.}

\alex{In \ref{The doors are closing}, it might sound like there is an additional /ɛ/ vowel after \textit{daa} `mouth' and before the plural marker \texit{-n}. However, perhaps this is lengthening of the vowel in \texit{da} `mouth' + the soronant /n/ following it? If this is the case, this is the expected pattern, witht he AUX /e/ following the grammatical subject in non-past constructions.}

\begin{table}[!htb]
  \centering
  \caption{Past}
  \begin{tabular}{llllll}
    \midrule
    {\prag \small {Transitive:}}   & \textsc{agent}.\textbf{S2} & \textsc{theme}             & -\textsc{obj} & verb      \\
    {\prag \small {Intransitive:}} &                            & \textsc{theme}.\textbf{S1} & -\textsc{obj} & verb & -A \\
    \midrule
  \end{tabular}
\end{table}

\begin{table}[!htb]
  \centering
  \caption{Non-past}
  \begin{tabular}{llllll}
    \midrule
    {{Transitive:}}   & \textsc{agent}.\textbf{S3} & \textsc{theme}             & -\textsc{obj} & verb & -A \\
    {{Intransitive:}} &                            & \textsc{theme}.\textbf{S3} & -\textsc{obj} & verb & -A \\
    \midrule
  \end{tabular}
\end{table}

\subsection*{Generalizing patterns}

\begin{itemize}[label=$\bullet$, left=0mm, labelsep=2mm,itemsep=0pt,topsep=3pt, rightmargin=0cm]
  \item If ``Series 2'' pronouns are really decomposed into [pronoun + /-a/], then a question worth asking is whether the verb final /-a/ (glossed as -A) in past intrans and non-past trans/intrans is the same -A on the subject of past transitives

  \item Additionally, it's worth considering that the /-e/ vowel on the (pronominal) subjects of non-past sentences is actually an auxiliary form (that is, the final vowel in the ``Series 3'' pronouns).

  \item Putting these two ideas together, we get a pattern like in \ref{tab:trans-intrans-organized}:
\end{itemize}

\begin{table}[!htb]
  \centering
  \caption{Past}
  \label{tab:trans-intrans-organized}
  \begin{tabular}{llllll}
    \midrule
    { {Transitive:}}   & \textsc{agent}-\textbf{A} & \textsc{theme} & -\textsc{obj} & verb           \\
    { {Intransitive:}} &                       & \textsc{theme} & -\textsc{obj} & verb & -\textbf{A} \\
    \midrule
  \end{tabular}
\end{table}

\begin{table}[!htb]
  \centering
  \caption{Non-past}
  \begin{tabular}{llllll}
    \midrule
    {{Transitive:}}   & \textsc{agent}-\textbf{AUX} & \textsc{theme}          & -\textsc{obj} & verb & -\textbf{A}     \\
    {{Intransitive:}} &                         & \textsc{theme}-\textbf{AUX} & -\textsc{obj} & verb & -\textbf{A} \\
    \midrule
  \end{tabular}
\end{table}

\ex. \textit{Past}
\ag.
n     -\textbf{á}   gbéŋgbɛ̀ɛ-n   -a     tune \\
1SG.PST   -\textbf{A}   boat-PL      -OBJ   sink \\
`I sank the boats.'
\bg.
\phantom{n}     \phantom{-á}            gbéŋgbɛ̀ɛ-n   -a     tune   -\textbf{a} \\
\phantom{1SG}   \phantom{-\textbf{A}}   boat-PL      -OBJ   sink   -\textbf{A} \\
`The boats sank.'

\ex. \texit{Non-past}
\ag.
m-bé    gbéŋgbɛ̀ɛ-n   {}                        -a     tune   -\textbf{a} \\
1SG-NPST   -\textbf{AUX}   boat-PL      \phantom{-\textbf{AUX}}   -OBJ   sink   -\textbf{A} \\
`I am sinking the boats.'
\bg.
{}              {}                        gbéŋgbɛ̀ɛ-n   -\textbf{e}     -a     tune   -\textbf{a} \\
\phantom{1SG}   \phantom{-\textbf{AUX}}   boat-PL      -\textbf{AUX}   -OBJ   sink   -\textbf{A} \\
`The boats are sinking.'



%\section{Causatives - periphrastic}
\section{Embedded clauses: finite (Joey)} 

\jal{remember we're using two vowels in a row for length, not :}

\jf{Think:}

\ex. inat͡ʃii \\
    `think' \jf{(Originally said nd͡ʒinat͡ʃii before changing to above version.)}\\

\exg. ná nd͡ʒ-inat͡ʃii bondu à swee dao kunu\\
1SG.? ?-think bondu AUX.PST meat eat yesterday\\
    `I think that Bandu ate meat yesterday'\\


\jf{Tony notes that inat͡ʃii is rarely used to mean "think" in this sort of context. mwajɔ ("it looks like") is more common}

\jf{The role of the morpheme (nd͡ʒ) at the front of inat͡ʃii is not entirely clear. Perhaps it's analogous to the an- morpheme in front of dao below?}


\exg. m-wajɔ bondu swee an-dao\\
1SG-it.seems.like Bandu meat ?-eat\\
    `I think that Bandu eats meat'\\

\jf{It should be noted that in the above, Tony corrected himself a few times from saying mwajɔ bondu à swee an-dao to mwajɔ bondu swee an-dao. If we are saying that the pronominal system in Kono has auxiliaries built in, it's possible that series 2 carries past tense meaning (for transitive and unergative verbs)}

\jf{Say:}

\ex. fɔ́\\
    `Say'\\

\exg. kai a-n-a fɔ́ bondu à swee dao kunu\\
Kai 3-PL-PST say Bandu AUX.PST meat eat yesterday\\
    `He says that Bandu ate meat yesterday'\\

\jf{The above seems to have an erroneous 3PL marker. An alternative interpretation is the following, with the na- morpheme being similar to the nd͡ʒ morpheme above and the an- morpheme below. }

\exg. kai à na-fɔ́ bondu à swee dao kunu\\
Kai 3SG ?-say Bandu AUX.PST meat eat yesterday\\
    `He says that Bandu ate meat yesterday'\\

\ex. te\\
    `Say' \jf{(This appears to take ɔ-series as subject - need more variety re: subjects with this to see whether it is actually an ɔ-series subject)}\\

\exg. kai ɔ́ te bondu à swee an-dao kunu\\
Kai 3SG say Bandu AUX.PST meat ?-eat yesterday\\
    `Kai says that Bandu ate meat yesterday'\\



\exg. ɔ́ te bondu à swee an-dao kunu\\
3SG say Bandu AUX.PST meat ?-eat yesterday\\
    `He says that Bandu ate meat yesterday'\\

\exg.  ɔ́ te n-á swee an-dao kunu\\
3SG say 1SG.PST meat ?-eat yesterday\\
    `He says that I ate meat yesterday'\\

\exg.  n-á swee an-dao kunu\\
1SG.PST meat ?-eat yesterday\\
    `I ate meat yesterday'\\

\jf{Tony mentions that the an- morpheme in the above example is optional. He says that using it is just using a longer version. It might be worth investigating what the nature of this morpheme is, especially if it is the same as the nd͡ʒ- and possible na- morphemes found above.}

\exg.  àndò te n-á swee an-dao kunu\\
3PL say 1SG.PST meat ?-eat yesterday\\
    `They say that I ate meat yesterday'\\



\section{Sentential Possession (Daniel)}

\exg. \textipa{tSen-\`a} m-b\textipa{\`e} b\textsuperscript{w}óó\\
house-3SG.OBJ 1SG-NPST have\\
`I have a house' \ds{Comment: literally `I own a house'}

\exg. \textipa{tSen\`a} wã b\textsuperscript{w}óó\\
house-3SG.OBJ FOC? have\\
`I have a house' \label{2sgex}

\ex. \textipa{tSen-\`a} wa-i b\textsuperscript{w}óó\\
house-3SG.OBJ FOC?-2SG have\\
`You have a house'

\exg. \textipa{tSen\`e-\`a} i b\textsuperscript{w}óó\\
house-3SG.OBJ 2SG have\\
`You have a house'

\exg. \textipa{tSen-\`a} \textipa{\`a-M} b\textsuperscript{w}óó\\
house-3SG.OBJ 3-PL have\\
`They have a house' \ds{Could mean they have one house collectively, could be separate houses}

\exg. \textipa{tSene-\`a} wa Bondu b\textsuperscript{w}óó\\
house-3SG.OBJ FOC Bondu have\\
`Bondu has a house'

\exg. \textipa{tSene-\`a} Bondu b\textsuperscript{w}óó\\
house-3SG.OBJ Bondu have\\
`Bondu has a house'

\exg. úú wa Bondu b\textsuperscript{w}óó\\
dog FOC Bondu have\\
`Bondu has a dog'

\exg. úú Bondu b\textsuperscript{w}óó\\
dog Bondu have\\
`Bondu has a dog'

\exg. \textipa{tSene} \textipa{n\`i-w\'aN} Bondu b\textsuperscript{w}óó\\
house PST-FOC? Bondu have\\
`Bondu had a house' \ds{(Note no obj suffix on `house'?)}

\exg. \textipa{tSene} ni Bondu b\textsuperscript{w}óó\\
house PST Bondu have\\
`Bondu had a house'

\exg. \textipa{tSen\'e\'e} màà wa Bondu b\textsuperscript{w}óó\\
house FUT? FOC Bondu have\\
`Bondu will have a house'

\exg. ??\textipa{tSen\'e\'e} màà Bondu b\textsuperscript{w}óó\\
house FUT Bondu have\\
`Bondu will have a house'

\exg. \textipa{tSeM} féa à-m-bè Bondu b\textsuperscript{w}óó\\
house two 3-PL-NPST Bondu have\\
\glt `Bondu has two houses'

\ex. *\textipa{tSeM} féa Bondu b\textsuperscript{w}óó\\
house two Bondu have\\
\ds{Comment: `An incomplete sentence'}

\exg. Bondu mfea/ni Kai-a a-m-be \textipa{tSeeM} féa mà sonda\\
Bondu and/and Kai-A 3-PL-NPST house two FUT ??\\
`Bondu and Kai will have two houses' \label{sondapl}

\exg. Bondu ni Kai \textipa{tSeM} féa (a)-m-be maa-m b\textsuperscript{w}o\\
Bondu and Kai house two 3-PL-NPST FUT-? have\\
\glt `Bondu and Kai will have two houses'

\exg. \textipa{tSene} \textipa{tSe} Bondu ã ta-mu\\
house this Bondu 3SG.POSS FOC ????\\
Intended: `This house is Bondu's' \ds{(Two sentences offered afterwards are of a different form)}

\exg. Bondu-a \textipa{tSen\'a\'a} ne\\
Bondu-3SG house NE\\
\ds{Comment: shorter version}

\exg. \textipa{tSene} \textipa{tSe} Bondu-e \textipa{tSenaa} ne\\
house this Bondu-3SG house\\
`This house is Bondu's house'(?)

\exg. \textipa{tSene} \textipa{tSe} Bondu mfea Kai-a \textipa{aN} \textipa{faNtamu}\\
house this Bondu and Kai-A 3PL.ADJ? ????\\
`This house is Bondu and Kai's'(??)

\ds{
\begin{itemize}
\item Verb b(\textsuperscript{w})óó for possession, used at least for alienables
\begin{itemize}
\item Unclear inflection paradigm? Repeated usage of series 3 (in nonpast contexts) but subject also sometimes seems to pattern with series 1, e.g. \ref{2sgex}
\end{itemize}
\item `wa'/`wã' may be focus, emphasised as expressing stronger `ownership' and generally omittable -- except for future
\item `sonda' possessive with plurals? (different from normal `have' possessives, \ref{sondapl}
\item Possibly a genuine possessive adjective which is predicated, see agreement morpheme in last example; however, the verb in this example is unclear?
\end{itemize}
}

\section{Embedded clauses: subject control (Chun-Hung)} 

\jal{Note that if you start the paradigm and it's clearly just one of the series we have, just get a few to verify; you don't have to go through the whole paradigm}

\chs{exhaustive subject control, other verbs: forget, stop, dare} 


\chs{partial subject control, other verbs: want, prefer} 

\chs{\textbf{Simple sentences}}

\exg. T\`{e}\'{o}-t\`{e}, mb\'{e} k\^{o}s\`{o}-wã̀ s\`{o}s\^{o}-m\`{a}. \\ 
every-day 1SG-NPST get.up-WA early-MA \\
`I get up early every day.'

\exg. w\`{e}w\^{e} \\
promise \\
`promise, secret'


\chs{\textbf{Subject control}}

\chs{Note: The tones here are a bit different between when he naturally said a sentence and when he articulated word by word}

\chs{Note: Note sure what the morpheme right after `want' is as agreement or potential subjects of the embedded, but it sounds to form a phonological unit with `want' to me.}


\exg. B\`{a}nd\'{u} t\'{u}m\'{u}-\`{a}n\`{i}\`{a} k\^{o}s\`{o} s\`{o}s\^{o}-m\`{a}. \\
Bandu want-3SG get.up early-MA \\
`Bandu wants to get up early.'


\exg. D\'{e}n\`{e}-t\textipa{S}\`{e}-n\`{u} ã̀  t\'{u}m\'{u}-\`{a}n\'{i}ã̀  k\^{o}s\`{o} s\`{o}s\^{o}-m\`{a}. \\
child-DEM-PL 3PL want-3PL get.up early-MA \\
`These children want to get up early.'

\exg. \'{N} t\^{u}m\`{u}-nĩ́  k\^{o}s\`{o} s\`{o}s\^{o}-m\`{a}. \\
1SG want-1SG get.up early-MA \\
`I (sg.) want to get up early.'

\exg. \'{I} t\^{u}m\`{u}-\`{i}n\'{i} k\^{o}s\`{o} s\`{o}s\^{o}-m\`{a}. \\
2SG want-2SG get.up early-MA \\
`You (sg.) want to get up early.'

\exg. \`{A} t\'{u}m\'{u}-\`{a}n\`{i}\`{a} k\^{o}s\`{o} s\`{o}s\^{o}-m\`{a}. \\
3SG want-3SG get.up early-MA \\
`He wants to get up early.'

\chs{\textbf{Full sentence?}}

\exg. N-\'{a} f\`{o}-n-f\'{a}nd\`{i}-\`{e} m-b\'{e} k\^{o}s\`{o} s\`{o}s\^{o}-m\`{a}. \\
1SG.PST tell-1SG.POSS-REFL-E 1SG-NPST get.up early-MA \\
`I told myself to get up early.' (as a substitute of `I promised to get up early.')


\end{document}