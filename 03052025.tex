\documentclass{assets/fieldnotes}

\title{Kono (Sierra Leone)}
\author{LING3020/5020}
\date{University of Pennsylvania, Spring 2025\\03/05/2025 Syntax II}

\setcounter{secnumdepth}{4} %enable \paragraph -- for subsubsubsections

\begin{document}

\maketitle
\tableofcontents

\section{Question and answer pairs: subject, object (Giang)} \jal{at least who, what, which, IO too if you have time}\\

\jal{I'd probably start with the transitive}

\jal{remember we're using Bondu as a name that we can more easily hear any /a/ following}

\jal{we really want the answers too!  Be ready to supply him an answer (if Bondu ate meat yesterday, how would you answer this question?)}

\g{\textbf{Intransitive--unergative}}\\
\g{Control (From previous elicitation}
\exg. Bondu \'{a} d\'{i}-t\textipa{S}\`{\textipa{E}} k\`{u}n\`{u}. \\
Bondu A cry-\textit{v} yesterday \\
`Bondu cried yesterday.'

\g{Context: In a classroom of 10 students, and one of them cried yesterday. However, the teacher does not know who was the one that cried.}

\exg. ɲɔ́nà dí-t\textipa{S}\`{\textipa{E}} kùnù?\\
Who cry-\textit{v} yesterday\\
``Who cried yesterday?''

\ex. Bondu.\\

\exg. Bondu á-nà dí-t\textipa{S}\`{\textipa{E}} kùnù?\\
Bondu 3SG.PST cry-\textit{v} yesterday.\\
``It was Bondu that cried yesterday.''\\
\g{t\textipa{S}\`{\textipa{E}} might be a verbalizer in little \textit{v}.}


\exg. kàndè mí̃ á-ná dí-t\textipa{S}\`{\textipa{E}} kùnù?\\
student which 3SG.PST cry-\textit{v} yesterday\\
``Which student cried yesterday?''


\g{\textbf{Intransitive--unacusative}}\\
\g{Control (From previous elicitation}
\exg. Bòndú ná wá Bájámà sénà.\\
Bondu come \textsc{fut} Baiama tomorrow\\
``Bondu will arrive in Baiama tomorrow.'' \g{wá is nasalized}

\exg. ɲɔ́mbè ná Bájámà sénà?\\
Who come Baiama tomorrow\\
``Who will arrive in Baiama tomorrow?''

\ex. Bòndú.\\
``Bondu.''

\exg. Bòndú á-m-bè nàtéá Bájámà sénà.\\
Bondu (3-PL-NPST??) come-? Baiama tomorrow.\\
``It is Bondu who will arrive tomorrow."

\exg. Kàndé míná á-m-bè nàtéá Bájámà sénà?\\
Student 3-PL-NPST come-? Baiama tomorrow\\
Which student will arrive in Baiama tomorrow?

\exg. Fé á-m-bè nà-téá Bájámà sénà?\\
What 3-PL-NPST come-? Baiama tomorrow\\
What will arrive (in Baiama) tomorrow?%\jal{a bit odd, eh?} \g{I'm praying that a context will help}\\
\g{Fé ámbè comes from fénè.}

\g{\textbf{Transitive}}\\
\g{Control (From previous elicitation}
\exg. Bòndú á swéè dàó̃ kùnù\\
Bondu 3SG.PST meat eat yesterday \\%
    `Bondu ate meat yesterday'

\exg. ɲɔ́nà swéè dàó̃ kùnù?\\
who meat eat yesterday\\
``Who ate meat yesterday?''

\exg. Bòndú á fé(n) dàó̃ kùnù?\\
Bondu 3SG what eat yesterday\\
``What did Bondu eat yesterday?''\\
\g{fé is from fénè}

\g{Control:}
\exg. káá twá dàó̃ kùnù?\\
snake rat eat yesterday\\
The snake ate a rat (yesterday).

\exg. fénà twá dàó̃ kùnù?\\
what rat eat yesterday\\
What ate a rat yesterday?

\exg. káá fé(n) dàó̃ kùnù?\\
snake what eat yesterday\\
What did the snake eat yesterday?


\section{Relative clauses: subject, object (Wesley)}

\wml{
\begin{itemize}
    \item With RCs, there does not seem to be `true embedding' where the RC sticks with the modified NP in its argument position. Instead, I hypothesise that Anthony was using these strategies:
    %
    \begin{itemize}
        \item Topicalise the RC-modified NP, e.g. \ref{man_broke_bondu_bought_topicalised}, \ref{bondu_see_man_break_topicalised}
        \item Extrapose the RC to the right, e.g. \ref{bondu_see_man_break_extraposed}, \ref{bondu_bought_man_break_extraposed}
        \item Topicalise and cleft, e.g. \ref{bondu_buy_kai_cook_cleft}
    \end{itemize}
    %
    \item The only example without any of these strategies is with an AdjP predicate \ref{bondu_buy_meat_delicious}
    %

    \item Relative clauses are marked with \textsc{mi} in the position relativised upon, e.g.  \ref{bondu_bought_man_break_extraposed}. This seems to be related to `which' in `which student' from Giang's data, so perhaps these are relative pronoun.
    \begin{itemize}
        \item In \ref{bondu_see_man_break_extraposed}, however, we see the Series 2 marking \textit{n\'{a}} alongside \textsc{mi}. Unless \textit{n\'{a}} is resumptive here, this data could support the idea that the `pronominal forms' have some auxiliary status.
        \end{itemize}
    %
    \item With relativisation on the object position, it seems as though RCs are internally-headed, e.g. \ref{kai_cook_bondu_buy_cleft}. That is, while we might expect \textit{?swee [\mss{RC} Kai a mi-tawa]}, with the head noun \textit{swee} outside of the RC and hence above the agent \textit{Kai}, we see it below the agent Kai, presumably within the RC.
\end{itemize}


\ex. sã́\\`buy’

\exg. Bondu à sɛ̀ɛ̀ sã̂\\
Bondu \textsc{3SG.PST} sɛɛ buy\\
`Bondu bought the sɛɛ.’

\exg. mwòkàmá à sɛ̀ɛ̀ ɔ̀ té\\
man \textsc{3SG.PST} sɛɛ \textsc{3sg.obj} break\\
`The man broke the sɛɛ.’

\wml{Prompt: The man broke the sɛɛ that Bondu bought.}
\exg. {[} Bondu à sɛ̀ɛ̀ mì sã́ kúnú {]} mwòkàmá ɔ̀ té\\
{} Bondu \textsc{3SG.PST} sɛɛ \textsc{mi} buy yesterday {} man \textsc{3SG} break\\
`The sɛɛ that Bondu bought yesterday, a man broke it.’\hfill{(topicalised)}\label{man_broke_bondu_bought_topicalised}

\exg. {[} mwòkàmá sɛ̀ɛ̀ mì-ndɔ̀ té {]} Bondu ana sã́ kúnú\\
{} man sɛ̀ɛ̀ \textsc{mi}-\textsc{3SG.obj} break {} Bondu \textsc{ana} buy yesterday\\
`The sɛɛ that the man broke, it was Bondu who bought it yesterday.’\hfill{(top+ cleft)}\label{man_broke_bondu_bought_cleft}

\wml{Prompt: Bondu saw the man who broke the sɛɛ.}
\exg. Bondu\mss{j} à\mss{j} mwòkàmá\mss{k} ẽ̀ẽ̀ {[} mì na\mss{k} sɛ̀ɛ̀ ɔ̀ té {]}\\
Bondu \textsc{3SG.PST} man see {} \textsc{mi} \textsc{3sg} sɛɛ \textsc{3sg.obj} break {}\\
`Bondu saw the man who broke the sɛɛ.’\hfill{(extraposed)}\label{bondu_see_man_break_extraposed}

\exg. {[} mwòkàmá\mss{k} mì na\mss{k} sɛ̀ɛ̀ ɔ̀ té {]} Bondu à\mss{?} ẽ̀ẽ̀\\
{} man \textsc{mi} \textsc{3sg} sɛɛ \textsc{3sg.obj} break {} Bondu \textsc{3SG.PST} see\\
`The man who broke the sɛɛ, Bondu saw him.’\hfill{(topicalised)}\label{bondu_see_man_break_topicalised}\\
\wml{Here, it's not clear whether à\mss{?} is the subject or object marker.}

\wml{Prompt: Bondu bought the sɛɛ that the man broke.}

\exg. Bondu à sɛ̀ɛ̀ sã́ {[} mwòkàmá mì ndò té {]}\\
Bondu \textsc{3SG.PST} sɛɛ buy {} man \textsc{mi} \textsc{3sg.obj} break {}\\
`Bondu bought the sɛɛ that the man broke.'\hfill{(extraposed)}\label{bondu_bought_man_break_extraposed}

\noindent{\rule{\textwidth}{1pt}}

\exg. Bondu à swèé sã́\\
Bondu \textsc{3SG.PST} meat buy\\
`Bondu bought the meat.'

\exg. Kai à swèé tàwá\\
Kai \textsc{3SG.PST} meat cook\\
`Kai cooked the meat.'

\wml{Prompt: Bondu bought the meat that Kai cooked \ref{kai_cook_bondu_buy_cleft}--\ref{bondu_buy_kai_cook_cleft}}

\exg. {[} Kai à swèé mĩ̀ tàwà {]}, Bondu à-n-à sã̂\\
{} Kai \textsc{3SG.PST} meat \textsc{mi} cook {} Bondu \textsc{3-PL-PST} buy\\
`The meat that Kai cooked, it was Bondu that bought it.'\hfill{(top+cleft)}\label{kai_cook_bondu_buy_cleft}

\exg. {[} Bondu à swèé mĩ̀ sã̂ {]} Kai à-n-à tàwà.\\
{} Bondu \textsc{3SG.PST} meat \textsc{mi} buy {} Kai \textsc{3-PL-PST} cook\\
`The meat that Bondu bought, it was Kai that cooked it.'\hfill{(top+cleft)}\label{bondu_buy_kai_cook_cleft}

\exg. Bondu à swèé tʃɛ̀ à sã̂\\
Bondu \textsc{3SG} meat this \textsc{3SG.PST} buy\\
`This is the meat that Bondu bought.'

\exg. {[} Bondu à swèé mĩ̀ sã̂ {]} à-dí\\
{} Bondu \textsc{3SG.PST} meat \textsc{mi} buy {} \textsc{3SG}-sweet\\
`The meat that Bondu bought is delicious.'\label{bondu_buy_meat_delicious}

\wml{Prompt: Kai is the man who cooked the meat.}

\exg. Kai mu mwòkàmá tʃɛ̀ ànà {[} mì ná swèé tàwà {]}\\
Kai \textsc{mu} man\mss{k} this \textsc{ana} {} \textsc{mi} \textsc{3sg}\mss{k} meat cook {}\\
`Kai is the man who cooked the meat.'\hfill{(extraposed, cleft)}

\exg. Kai à-n-à swèé tàwà\\
Kai \textsc{3-PL-PST} meat cook\\
`It was Kai who cooked the meat.'\hfill{(cleft)}\\
\wml{Not sure if there's really an RC in the Kono here; RCs generally have the \textsc{mi} marking in Kono. It could be that this is a headless RC in Kono and that they lack \textsc{mi}.}

\exg. {[} mwòkàmá mì ná swèé tàwà {]} a-tomu Kai\\
{} man \textsc{mi} \textsc{3sg} meat cook {} \textsc{3sg}-name Kai\\
`The man who cooked the meat, his name is Kai.'\label{man_cook_meat_name_kai}

\noindent{\rule{\textwidth}{1pt}}



\section{Relative clauses: adjunct (Alex)} 


\subsection*{Follow-up on trans/intrans - caus}

\alex{The previous session (20250226), Tony provided the sentence in \ref{The boats are sinking - obj marker}.}

\exg.
gbéŋgbɛ̀ɛ   -n    -e     -a     tune-a \\
boat       -PL   -AUX   -OBJ   sink-A \\
`The boats are sinking.' \label{The boats are sinking - obj marker}

\alex{In this session, Tony provided another formation of this sentence, in \ref{The boats are sinking - no obj marker}. Note that \ref{The boats are sinking - no obj marker} does not contain the ``object marker'' /-a/.}

\exg.
gbéŋgbɛ̀ɛ   -m    -be    tune-a \\
boat       -PL   -AUX   sink-A \\
`The boats are sinking.' \label{The boats are sinking - no obj marker}

\alex{In slow speech, Tony segments the sentence as in \ref{The boats are sinking - slow}, with the plural (homorganic nasal for the plural remaining on `boats').}

\exg.
gbéŋgbɛ̀ɛ̃   -be    tune-a \\
boat.PL    -AUX   sink-A \\
`The boats are sinking.' \label{The boats are sinking - slow}

\exg.
mue     -m    -be    tune-a \\
person   -PL   -AUX   sink-A \\%
`The people are diving/sinking.'

\exg.
musu    -m    -be    tune-a \\
woman   -PL   -AUX   sink-A \\%
`The women are diving/sinking.' \label{The women are diving - no little v}

\alex{Note the contrast between \ref{The women are diving - no little v} and \ref{The women are diving - little v}; minimal pair with overt little v?}

\exg.
musu    -m    -be    tune   tʃ-a  \\
woman   -PL   -AUX   sink   v-A \\%
`The women are diving/sinking.' \label{The women are diving - little v}

\alex{Note that the ``object marker'' is unacceptable in \ref{The women are diving}.}

\exg.
musu    -m    -be    (*a)   tune-a \\
woman   -PL   -AUX   (*A)   sink-A \\%
`The women are diving/sinking.' \label{The women are diving}

\subsection*{Relative clauses: adjunct} 

\subsection*{the reason why...}

\paragraph*{Object Position}

\exg.
bondu   beja \\
Bondu   fall \\%
`Bondu fell.'


\exg.
n-ɔ      sõɱfã   fena-ma   bondu   beja \\
1SG-AUX.PST-OBJ  know    thing-P   bondu   fall \\%
`I know the reason why Bondu fell.'

\exg.
n-a-ɔ      sõɱfã   fena-ma   bondu   beja \\
1SG-AUX.PST -OBJ   know    thing-P   bondu   fall \\%
`I knew the reason why Bondu fell.'

\exg.
bondu   a     jaɡbasi   a     beja \\
Bondu   3SG.PST   onion     OBJ   fall \\%
`Bondu dropped the onion.'

\exg.
musu    na    jaɡbasi   a     beja \\
woman   3PL   onion     OBJ   fall \\%
`The women dropped the onion.' \label{The women dropped the onion}

\exg.
musu    mbe    jaɡbasi   a     beja-a \\
woman   3PL   onion     OBJ   fall-A \\%
`The women are dropping the onion.' \label{The women are dropping the onion}

\alex{There seems to be a clear contrast in vowel length on the verb `beja' between \ref{The women dropped the onion} (short) and \ref{The women are dropping the onion} (long), which is expected given the pattern that non-past sentences (both trans and intrans), there is a verb-final vowel /-a/.}

\exg.
n-ɔ      sõɱfã   fena-ma   bondu   a     jaɡbasi   a   beja \\
1SG-AUX.PST -OBJ   know     thing-P      bondu   3SG.PST   onion     A   fall \\%
`I know the reason why Bondu dropped the onion.'

\paragraph*{Subject Position}

\exg.
kwe   mina ma   bondu   beja   a     ɡbo    fa      i   nfani \\
reason  which P   bondu   fell   3SG   body   tired   ?   ?     \\
`The reason why Bondu fell is (because) he is tired.'

\exg.
kwe   mina ma   bondu   a     jaɡbasi   a     beja   a     ɡbo    fa      i   nfani \\
reason     which P   Bondu   3SG.PST   onion     OBJ   fall   3SG   body   tired   ?   ?     \\
`The reason why Bondu dropped the onion is because he is tired.'

\exg.
bondu   ɡbo    fa      i   nfani,   a     nama     a     jaɡbasi   a   beja \\
bondu   body   tired   ?   ?,       3SG.PST   reason   3SG   onion     A   fall \\%
`Bondu was tired; that was why he dropped the onion.'

\subsection*{the place where...}

\paragraph*{Object Position}

\exg.
n-a    kena    jɛ̃    bondu   beja   kena    mi \\
1SG.PST   place   see   Bondu   fall   place   ?  \\%
`I saw the place where Bondue fell.'

\exg.
bondu   beja   newã \\
Bondu   fall   here \\%
`Bondue fell here.'


\section{Embedded questions (Joey)}

\jf{Follow-up on last week:}

\exg. te\\
    `Say' \jf{(This appears to take ɔ-series as subject - need more variety re: subjects with this to see whether it is actually an ɔ-series subject)}\\

\exg. N-do te\\
1SG-OBJ say\\
    `I say' \\


\jf{ask:} 

\exg. tisa\\
    `ask"\\ 

\exg. tisa-t͡ʃe\\
ask-v\\
    `ask"\\ 

\jf{In the example above, the context that Tony provided suggested that the t͡ʃe suffix was used for imperatives; however, this analysis will be complicated by the use of t͡ʃe in the embedded question below that takes "who" as its subject. (64)}


\exg. n-a bondu ɔ tisa ne a swee dao faŋ kunu\\
1SG.PST Bondu 3SG.ɔSer ask whether 3SG.PST meat eat ? yesterday \\
    `I asked Bondu whether she ate meat yesterday"\\

\jf{Tony said that nea means "whether;" however, in other examples using "whether," he simply said ne. An alternative analysis of the above sentence keeping "whether" as ne can be seen below:}

\exg. n-a bondu ɔ tisa ne a swee dao faŋ kunu\\
1SG.PST  Bondu 3SG.ɔSer ask whether 3SG.PST .Ser2 meat eat ? yesterday \\
    `I asked Bondu whether she ate meat yesterday"\\


\exg. n-ɔ tisa ne bondu a swee dao faŋ kunu\\
1SG.PST -3SG.ɔSer ask whether Bondu AUX.PST. meat eat ? yesterday \\
    `I asked him/her whether Bondu ate meat yesterday" \jf{(the "a" in "na" (1SG.Ser2) seems to be elided and replaced by "ɔ" 3SG.ɔSer)}\\

\exg. na jɔ tisa ne bondu a swee dao faŋ kunu\\
1SG.PST 2SG.ɔSer ask whether Bondu AUX.PST meat eat ? yesterday \\
    `I asked you (sg.) whether Bondu ate meat yesterday"\\

\jf{It appears to be impossible to say "I asked" without an indirect object. "tisa" necessarily appears to be ditransitive.}

\jf{It is unclear to me what purpose faŋ serves. It doesn't seem to be necessary in non-question embedded CPs, per the example below from the previous week:}

\exg. kai à na-fɔ́ bondu à swee dao kunu\\
Kai 3SG ?-say Bondu 3SG.PST meat eat yesterday\\
    `He says that Bandu ate meat yesterday'\\

\jf{It is possible that faŋ may be some sort of question-marker?}

\exg. n-ɔ tisa bondu a feŋ dao kunu\\
1SG.Ser2-3SG.ɔSer ask Bondu 3SG.Ser2 what eat yesterday\\
    `I asked what Bondu ate yesterday"\\

\jf{The wh-word in embedded questions appears to remain in-situ in Kono, per the above.}

\exg. n-a tisa-t͡ʃe ɲon-á swee dao kunu\\
1SG.PST ask-v who-AUX.PST meat eat yesterday\\
    `I asked who (pl.) ate meat yesterday" \jf{(Like, if some people meat and others vegetables)}\\ 

\exg. n-ɔ tisa ɲon-á swee dao kunu\\
1SG.Ser2-3SG.ɔSer ask who-3SG.PST meet eat yesterday\\
    `I asked who (pl.) ate meat yesterday" \jf{(Like, if some people meat and others vegetables)}\\ 

\jf{There seems to be some difference between tisa and tisa-t͡ʃe that is not the same as an indicative vs. imperative distinction. Interestingly, when the -t͡ʃe is present, there is no ɔ-series indirect object present. Perhaps the -t͡ʃe suffix indicates that the verb is not ditransitive and that is why it works in the imperative context (ie: "ask it" not "ask me it.")}

\exg. ɲon-à\\
who-sg\\
    `who (sg.)" \\

\exg. ɲon-à swee dao kunu\\
who-3SG.PST meat eat yesterday\\
    `Who (sg.) ate meat yesterday?"\\

\jf{I wonder if ɲon-à is just ɲon + à (3SG.Ser2), and if ɲon-á is a shortened version of ɲon + àná (3PL.Ser2)}
    

\exg. fenama bondu a swee dao kunu\\
why Bondu 3SG.PST meat eat yesterday\\
    `Why did Bondu eat meat yesterday?"\\

\exg. na tisa-t͡ʃe fenama bondu a swee dao kunu\\
1SG.PST ask-v why Bondu 3SG.PST meat eat yesterday\\
    `I asked why Bondu ate meat yesterday" \jf{(with -t͡ʃe)}\\

\exg. n-ɔ tisa fenama bondu a swee dao kunu\\
1SG.Ser2-3SG.ɔSer ask why Bondu 3SG.PST meat eat yesterday\\
    `I asked why Bondu ate meat yesterday" \jf{(without -t͡ʃe)}\\

\exg. tima\\
    `time"\\

\exg. bondu a swee dao kunu tima-minana\\
Bondu 3SG.PST meat eat yesterday time-which\\
    `When did Bondu eat meat?"\\

\exg. bondu a swee dao tima-minana kunu\\
Bondu 3SG.PST meat eat time-which yesterday\\
    `When did Bondu eat meat?"\\

\jf{Placement of "when" can apparently come before or after "yesterday"}

\exg. na tisa-t͡ʃe bondu a swee dao tima-minana kunu\\
1SG.PST ask-v Bondu 3SG.PST meat eat time-which yesterday\\
    `I asked when Bondu ate meat yesterday"\\

    
\section{Yes/No question and answer pairs, varying focus (Daniel)}
\ds{Keeping these in past tense to avoid problems/oddity with habitual so far, but I might add in other tenses and more verbs} \jal{looks good} \\

\ds{Intransitive}

\exg. Bondu-a tombw\textipa{E} dom fa kunu?\\
Bondu-3SG dance ? FOC? yesterday\\
`Did Bondu dance yesterday?'\\
`Did \textsc{Bondu} dance yesterday?'\\
`Did Bondu \textsc{dance} yesterday?'

\ds{Thinking that \textit{fa} is a focus marker of some kind? Here predicate focus+projection?}

\exg. Àà, Bondu-a tomb\textipa{E} dom fa kunu\\
Yes Bondu-3SG dance ? FOC? yesterday\\
`Yes, Bondu danced yesterday'


\exg. A-a, Bondu-ma tombw\textipa{E} du-ni kunu\\
No Bondu-NEG dance ? ? yesterday\\
`No, Bondu did not dance yesterday'

\exg. Bondu-a na tombwe do kunu?\\
Bondu-3SG FOC/COP dance ? yesterday\\
`Was it Bondu that danced yesterday?'
\ds{Unambiguous givenness of `dance' and focus of Bondu}

\exg. \textipa{tSa}-na tombwe du kunu\\
this.one-COP? dance ? yesterday\\
`\textsc{This person} danced yesterday'

\exg. Kai-a na tombwe du kunu\\
Kai-3SG COP? dance ? yesterday\\
`\textsc{Kai} danced yesterday'

\exg. *Tombwe-a na Bondu dom-fa kunu\\
dance-3SG Bondu ? ? yesterday\\
Intended: `Was it dancing that Bondu did yesterday?'\\
\ds{Wanted to see whether possible to front the verb for clefting, but I think I constructed the example wrong. Tony's comment: this sounds like the dancing itself was somehow affecting Bondu}

\exg. Bondu-a tombwe ndu kunu?\\
Bondu-3SG dance ? yesterday\\
approx. `Was it dancing that Bondu did yesterday?'

\exg. A-a, Bondu-a \textipa{tS}en-wẽ \textipa{tS}e kunu\\
No Bondu-3SG sleep-? yesterday\\
`No, Bondu \textsc{slept} yesterday'

\exg. Bondu-a fe-ma kunu\\
Bondu-3SG thing-? yesterday\\
`What did Bondu do yesterday?'

\ds{for `fe-ma' cf. `fena(-)ma' \textit{why}}

\exg. Bondu-a tombwe ndu kunu!\\
Bondu-3SG dance ? yesterday\\
`Bondu \textsc{danced} yesterday!'

\exg. Bondu-a síí ja\textipa{N}sa\textipa{N} kunu\\
Bondu-3SG sing tll yesterday\\
`Bondu \textsc{sang} yesterday!'

\exg. Bondu-a tombe-du kunũ fã\\
Bondu-3SG dance-? yesterday ?\\
`Was it \textsc{yesterday} that Bondu danced?'

\exg. A-a, Bondu-a tombwe-du bìì-wa\\
No Bondu-3SG dance-? today-FOC(???)\\
`No, Bondu danced \textsc{today}'

\exg. A-a, Bondu-a tombwe-du kunũgo-wa/*fã\\
No, Bondu-3SG dance-? yesterday.before-FOC(???)\\
`No, Bondu danced the day before yesterday'


\section{Long distance questions (Chun-Hung)}

\jal{use ``say"; we don't have ``know'' yet, and we want to start with the easiest, likely most frequent, embedding verb. (Indeed, ``know'' would be factive and so potentially not a bridge verb even in languages that have long distance questions.) You might start with transitive; it's unlikely you're going to get all of these, so that will give you subject and object at least.}

\chs{transitive}

\exg. B\`{o}nd\'{u} \`{a} s\`{\textipa{E}}\'{\textipa{E}} sã̀. \\ 
Bondu AUX.PST s\textipa{E}\textipa{E} buy \\
Bondu bought the s\textipa{E}\textipa{E}. (data from Wesley)

\exg. ɲ\'{o}n-\`{a} s\`{\textipa{E}}\'{\textipa{E}} sã̀? \\ 
who-AUX.PST s\textipa{E}\textipa{E} buy \\
Who bought the s\textipa{E}\textipa{E}?

\exg. B\`{o}nd\'{u} \`{a} fẽ̌ sã̀? \\ 
Bondu AUX.PST what buy \\
What did Bondu buy?

\exg. K\`{a}\`{i} \'{\textipa{O}} t\'{e} B\`{o}nd\'{u} \`{a} s\`{\textipa{E}}\'{\textipa{E}} sã̀. \\ 
Kai 3SG say Bondu AUX.PST s\textipa{E}\textipa{E} buy \\
Kai said [that Bondu bought the s\textipa{E}\textipa{E}].


\exg. K\`{a}\`{i} \'{\textipa{O}} t\'{e} B\`{o}nd\'{u} \`{a} fẽ̌  sã̀? \\ 
Kai 3SG say Bondu AUX.PST what buy \\
What\mss{k} did Kai say [that Bondu bought \trace{k}]? \chs{may be an indirect question}

\exg. K\`{a}\`{i} \'{\textipa{O}} t\'{e} ɲ\'{o}n-\`{a} s\`{\textipa{E}}\'{\textipa{E}} sã̀? \\ 
Kai 3SG say who-AUX.PST s\textipa{E}\textipa{E} buy \\
Who\mss{k} did Kai say [\trace{k} bought the s\textipa{E}\textipa{E}]? \chs{may be an indirect question}

\chs{transitive: \textipa{O}-series}

\exg. M\`{o}k\`{a}m\^{a} s\`{\textipa{E}}\'{\textipa{E}} \`{\textipa{O}} t\^{e}. \\
man s\textipa{E}\textipa{E} AUX.PST break \\
The man broke the s\textipa{E}\textipa{E}. (data from Wesley)

\exg. ɲ\'{o}n-\`{a} s\`{\textipa{E}}\'{\textipa{E}} \`{\textipa{O}} t\^{e}? \\
who-AUX.PST s\textipa{E}\textipa{E} 3SG break \\
Who broke the s\textipa{E}\textipa{E}? \jal{palatal nasal, no following /i/}

\exg. M\`{o}k\`{a}m\^{a} fẽ̌  nd\`{\textipa{O}} t\^{e}? \\
man what AUX.PST break \\ 
What did the man break? \chs{nd- part could be influenced by the nasal of `what'?}


\exg. K\`{a}\`{i} \'{\textipa{O}} t\'{e} m\`{o}k\`{a}m\^{a} s\`{\textipa{E}}\'{\textipa{E}} \`{\textipa{O}} t\^{e}. \\ 
Kai 3SG say man s\textipa{E}\textipa{E} AUX.PST break \\
Kai said [that the man broke the s\textipa{E}\textipa{E}]. 

\exg. K\`{a}\`{i} \'{\textipa{O}} t\'{e} m\`{o}k\`{a}m\^{a} fẽ̌  nd\`{\textipa{O}}  t\^{e}? \\ 
Kai 3SG say man what AUX.PST break \\
What\mss{k} did Kai say [that the man broke \trace{k}]? \chs{may be an indirect question}

\exg. K\`{a}\`{i} \'{a} ɲ\'{o}n-dʒã̀ m\'{i}n-\`{a} s\`{\textipa{E}}\'{\textipa{E}} \`{\textipa{O}}  t\^{e}? \\
Kai 3SG who-JA MIN-3SG s\textipa{E}\textipa{E} AUX.PST break \\
Who\mss{k} did Kai say [\trace{k} broke the s\textipa{E}\textipa{E}]? \chs{looks different from previous one, and is similar to relativization; could be prolepsis but needs further checking}


\section{Imperatives (Jan)} \jal{you singular, you plural, transitive and intransitive, but make sure all are agentive verbs, starting with ones we know}

\jal{you wont' have time for this many; you'll need to prioritize}
\jmt{Ditransitives}

\exg. bondu a kwee tʃe da-o \\
bondu AUX.PST rice put pot-in \\
`Bondu put the rice in the pot.'

\exg. bondu kwee tʃe da-o \\
bondu rice put pot-in \\
`Bondu, put the rice in the pot.'

\exg. kaj o fea ó kwee tʃe da-o \\
kaj CONJ TWO 2PL.EXC.IMP rice put pot-in \\
`Kai and Bondu, put the rice in the pot.'


\exg. bondu o mò kwee tʃe da-o \\
bondu ? 2PL.DU.INC rice put pot-in \\
`Bondu, let's put the rice in the pot.'

\exg. kaj mó kwee tʃe da-o \\
kai 2PL rice put pot-in \\
`Kai and Bondu, let's put the rice in the pot.'


\exg. bondu gboo be kaj ma \\
bondu book give kai P \\
`Bondu, give the book to Kai.'

\exg. bondu mo gboo be kaj ma \\
bondu 2PL.DU.INC book give kai P \\
`Bondu, let's give the book to Kai.'

\exg. saa gbo be mo kaj ma \\
saa book give 2PL.INC kai P \\
`Saa and Bondu, let's give the book to Kai.'


\jmt{Transitives} \jal{start with a verb we know; look at isn't likely to be canonical transitive; you'll just have time for one}

\exg. bondu gbo a bea \\
bondu book 3SG drop \\
`Bondu, drop the book.'

\exg. bondu o fea gboo a bea \\
bondu CONJ two book 3SG drop \\
`Bondu and Kai, drop the book.'

\exg. bondu tʃene ma féé \\
bondu house P look \\
`Bondu, look at the house.'

\exg. bondu tʃene ma fé-ẽ́\\
bondu house P look-2PL.DU.INCL? \\
`Bondu, let's look at the house.'

\exg. bondu táá tʃe à \\
bondu calabash this take \\
`Bondu, take this calabash.'

\exg. bondu mò taa tʃe à \\
bondu 2PL.DU.INCL  calabash this take \\
`Bondu, let's take this calabash.'

\exg. bondu mó taa tʃe à \\
bondu 2PL.INCL calabash this take \\
`Saa and Bondu, let's take this calabash.'


\jmt{Intransitives} \jal{let's get an agentive one too }

\exg. bondu kune \\
bondu wake.up \\
`Bondu, wake up!'

\exg. bondu fea saa ó kune \\
bondu CONJ Saa 2PL.EXC.IMP wake.up \\
`Saa and Bondu, wake up!'

\exg. bondu i too ma so \\
bondu REFL? ? P listen \\
`Bondu, listen!'

\exg. bondu o fea soo ma \\
bondu CONJ two listen P \\
`Saa and Bondu, listen!'

\exg. saa o fea bondu a to ma soo \\
Saa CONJ two Bondu 3SG? ? P listen \\
`Saa and Bondu, let's listen!'

\jmt{Reflexives}\jal{you won't have time for these}

\ex. kɔ̀ \\
`wash yourself'

\exg. bondu i kɔ̀ \\
bondu REFL? wash \\
`Bondu, wash yourself!' 

\exg. bondu momo kɔ̀ \\
bondu ? wash \\
`Saa and Bondu, wash yourselves!'

\section{Modals (Mingyang)} 


\mb{\textbf{Dynamic (Ability)}}
\par
\mb{Scenario: talking about people's abilities.}
\exg. ḿ-bé kɔ̀nɔ́ kwé fɔ̀ɔ̀-wã̀.\\
    1SG-NPST Kono language speak-FUT\\
    `I can speak Kono.'

\exg. Bondu kɔ̀nɔ́ kwé fɔ̀ɔ̀-wã̀.\\
    Bondu Kono language speak-FUT\\
    `Bondu can speak Kono.'

\exg. ḿ-bé túné-t͡ʃɛ̀-wã̀.\\
    1SG-NPST swim-v-FUT\\
    `I can swim.'

\exg.Bondu túné-t͡ʃɛ̀-wã̀.\\
    1.SG swim-v-FUT\\
    `Bondu can swim.'

\mb{\textbf{Deontic}}
\par
\mb{Scenario: in a building that is required to be vacated by 10pm.}
\exg. ḿ-bé ná tè-á tá.\\
    1SG-NPST ? ? leave\\
    `I am about to leave.' (`I must leave now.')

\mb{Scenario: in a company where everyone is free to go after 5.}
\exg. mwe᷈̂ tá-wã̀ bè.\\
    1INCL-NPST leave-FUT now\\
    `We can leave now.'

\exg. é tá-wã̀ bè.\\
    2SG.NPST leave-FUT now\\
    `You can leave now.'

\exg. Bondu tá-wã̀ bè.\\
    Bondu leave-FUT now\\
    `Bondu can leave now.'

\mb{\textbf{Teleological}}
\par
\mb{Scenario: talking about going to a specific destination.}
\exg. é bɔ́sì ẽ̀ jìà\\
    2SG.NPST bus ? take\\
    `You must take the bus.'

\exg. é bɔ́sì jìà-wã\\
    2SG.NPST bus take-FUT\\
    `You can take the bus.'

\exg. ḿ-bé tá kámà?\\
    1SG-NPST leave Q\\
    `How do I go?'

\section{Quantification (Lex)} 

\wml{What's a common fruit eaten in your community/ Sierra Leone?}

\exg. dùmbî (í)\\ 
orange\\

\exg. Bondu-à dùmbí n-dàó̃ \\
Bondu AUX.PST orange eat\\
`Bondu ate an orange'

\exg. Bondu-à dùmbíí n-dàó̃ \\
Bondu AUX.PST oranges eat\\
`Bondu ate oranges'

\wml{ Anthony 1126 structure was used for any plural amount of oranges "a couple" or "some"}

\exg. Bondu-à dùmbíí n-dàó̃ \\
Bondu AUX.PST oranges eat\\
`Bondu ate some oranges/ a couple of oranges'

\exg. Bondu-má dùmbí n-dàó̃ í\\
Bondu AUX.PST did not orange eat\\
`Bondu ate no oranges' 

\wml{ Heard an additional [í] at the end of the verb for eat also with the negation morpheme [má]}

\exg. Bondu-à dùmbí sɛ́sɛ́ n-dàó̃ \\
Bondu AUX.PST orange mostly ate\\
`Bondu ate mostly oranges'

\exg. Bondu-à dùmbí (n)déɛ̀ n-dàó̃ \\
Bondu AUX.PST orange a little ate\\
`Bondu ate a little bit of (not much) orange'

\exg. Bondu-à dùmbí wá n-dàó̃ \\
Bondu AUX.PST orange a lot ate\\
`Bondu ate a lot of orange'

\wml{Maybe question if saying little is a normal way to say didn't eat much, comparison of size morpheme}

\exg. Bondu-à dùmbí dú n-dàó̃ \\
Bondu AUX.PST oranges 5 ate\\
`Bondu ate 5 oranges'

\exg.Bondu-à dùmbí ɔ́ɔ́rlɔ́ n-dàó̃ 
Bondu AUX.PST oranges 6 ate\\
`Bondu ate 6 oranges'

\exg. dùmbíí n-dí\\
Oranges are sweet\\
`Oranges are sweet'

\exg. dùmbí dí\\
Orange is sweet\\
`Orange is sweet'

\exg. dùmbí m-bɛ́ɛ́ dí\\
 orange is every sweet\\
`Every orange is sweet'

\exg. dùmbí m-bɛ́ɛ́ díwá\\
Orange sweet all \\
`All oranges are sweet'

\wml{ not sure how to fully outline the structural parts of this phrase}

\exg. dùmbí sɛ́sɛ́ dí\\
Oranges most sweet\\
`Most oranges are sweet'

\exg. Mínámà dí\\
Which ones not sweet\\
`Which ones are not sweet?'

\exg. dùmbí twɛ́tʃɛ̀ dí\\
Orange half of is sweet\\
`Half of the orange is sweet/ leftovers of the orange are sweet'





\end{document}