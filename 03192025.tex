\documentclass{assets/fieldnotes}
\usepackage{xr}
\usepackage{xr-hyper}
\externaldocument{02262025}

\title{Kono (Sierra Leone)}
\author{LING3020/5020}
\date{University of Pennsylvania, Spring 2025\\03/19/2025 Kono Story}

\setcounter{secnumdepth}{4} %enable \paragraph -- for subsubsubsections


\begin{document}

\maketitle

\tableofcontents

\section{Line 1 - Lex} 
\exg.
kaɛnɛ n-fɛ̀á fà à-n sɛ̀já à-n-á kòˈŋgʷɛ̀-ɔ̀ tʃ⁠ɛ̀nà\\
young.boy 1SG-and father 3-PL climb 3-PL-POSS their mountain-OBJ house-POS\\
`Young boy and his father climb their mountain house'

\section{Line 2 - Mingyang} 
\exg. mbɛ̀ɛ́ kɔ̀mbá fà fɔ̀ɔ̀-a-jẽ́ í má bɔ̀ bì.\\
    then Komba father tell-AUX.PST-JEN 2SG NEG go.out today\\
    `Komba's father told him `Don't go out today.''

\section{Line 3 - Giang} 
\exg. kɔ̀mbá fàà-té-à mbè bóá bámà búí-à\\
Komba heart-?-\textsc{a} then go.out outside run-?\\
"Komba was sad/vexed then he went out of the house running."

\section{Line 4 - Joey} 

\exg. mbɛ̀ɛ́ à éètɛ̀ má tù-nì-jà mbɛ̀ɛ́ kòŋɡwé à àbíjà má tù-nì-jà\\
then AUX.PST shout NEG love-NEG-2SG.POSS(?) then mountain AUX.PST echo NEG love-NEG-2SG.POSS(??)\\
    `Then he shouted `I don't love you,' then the mountain echoed `I don't love you.''\\ 

\jf{1) I had a hard time determining the length of the vowel for "then." It felt a little longer than usual but a little short for a long vowel. I was debating between what I had written and mbɛ̀. EDIT: It's likely just a prosodic effect of him dragging the word for rhetorical reasons. Actual word is mbɛ̀}\\

\jf{2) Figuring out the structure of "I don't love you" was a challenge, and there are still questions to answer. The verb appears to take Series 2 objects (based off of e tu-wa-wã-na being "You will love me. It is odd, however, for Series 2 pronouns to be objects. }\\

\exg. e tu-wa-wã-na\\
2SG.NPST love-FOC/FUT?-FOC/FUT?-1SG(?)\\
    `You will love me.'\\

\jf{3) The negation marker being má is complicated. Tony reported má to be "I," but other examples seem to show it to be negation. It's possible that it's actually 1SG.SER1-má, in which case, the transcription and gloss would look more like the following (assuming that the 1SG.SER1 pronoun (N) matches the place of articulation of the succeeding nasal. I've also included the shorter mbɛ̀ transcritpion below:}\\

\exg. mbɛ̀ à éètɛ̀ m-má tù-nì-jà mbɛ̀ kòŋɡwé à àbíjà m-má tù-nì-jà\\
then 3SG shout 1SG.SER1-NEG love-NEG-2SG then mountain 3SG echo 1SG.SER1-NEG love-NEG-2SG\\

\section{Line 5 - Alex}

\exg.
kɔ̀mbá   jé̃é̃-à   fà       tè-à,   dìì-à.   mbɛ̀    á     fɔ̀-à     fà       jɛ́  \\
Komba   go-A    father   ?-A,    cry-A.   then   3SG.PST   tell-A   father   3SG \\%
`Komba returned to his father, crying. Then he told his father,'

\ex.
\ag.
kɔ̀mbá   jé̃é̃-à   fà       tè-à,   dìì-à \\
Komba   go-A    father   ?-A,    cry-A \\%
`Komba returned to his father, crying'
\bg.
mbɛ̀    á     fɔ̀-à     fà       jɛ́  \\
then   3SG.PST   tell-A   father   P \\%
`Then he told (to) father,' \label{Then he told his father}

\alex{The final 3\Sg{} form in \ref{Then he told his father} appears to include a post-position /jɛ/. This is the same post-position we see in beneficiary ditransitives.
This is especially apparent when considering other person/number combinations in this construction, namely the 3\Pl{} pronoun as shown in \ref{told them}, where we get /(a)ndʒɛ/ -- the same form we see for 3PL beneficiaries + post-position. See document file 02262025.tex, section \ref{Ditransitives: beneficiary, etc (Mingyang)} (Ditransitives: beneficiary, etc (Mingyang)), example \ref{3PL beneficiary}, corresponding to sound file 02262025.wav, timestamp 00:48:13, for the sentence \textit{`I will cook cassava leaves for them tomorrow'}, repeated below in \ref{3PL beneficiary 2}.
}

\exg.
ḿ-bé    tàŋgùmbà         táwá   wã    \textbf{à̃ }     \textbf{dʒé}    síná     \\
1SG-NPST   cassava.leaves   cook   FUT   \textbf{3.PL}   \textbf{for}   tomorrow \\
`I will cook cassava leaves for them tomorrow.' \label{3PL beneficiary 2}

Discussion about beneficiary/recipient pronouns at 01:03:30-01:05:30 in 03192025.wav

\exg.
mbɛ̀    á     fa       fɔ̀-à     \textbf{n-dʒɛ́ }\\
then   AUX.PST   father   tell-A   \textbf{3PL-P}  \\%
`Then his father told them,' \label{told them}


\section{Line 6 - Wesley} 

\exg. mwɛ́ɛ́ kɔ̀ŋgwɛ́-tʃ-ɔ̀ mín-dɔ̀-té à-má-tù-ní-n-à\\
person mountain-this-in \textsc{rel}-\textsc{3sg.obj}-say \textsc{3sg}-\textsc{neg}-love-\textsc{neg}-\textsc{1SG}\\
`There's someone in this mountain who said he doesn't like me.'

\wml{Anthony used an existential construction (``there's...'') to translate this sentence. I wonder if kɔ̀ŋgwɛ́-tʃ-ɔ̀ functions as a non-verbal predicate denoting location and here used to express existence; that seems to have been the case for Mingyang's PP predicate data on 02192025. Then, the clause ``who said he doesn't like me'' would be a relative clause postponed to the right of the existential, as we have observed of relative clauses.}

\section{Line 7 - Jan} 

\exg. mbè á fá fwà-é jè tónó í ní-èté ń-tùmù-já \\
then AUX.PST father told-AUX.PST.OBJ 3SG go.back then 2SG-shout 1SG-love-2SG \\
`Then, his father told him go back and shout, "I love you!"'

\section{Line 8 - Daniel} 

\exg. komba yéé-à. mbe a éèté, n-túmú-jà. ko\textipa{N}gwé-à wé(é) a-bííjà n-túmú-jà.\\
Komba go-3SG then AUX.PST shout 1SG-love-2SG mountain-AUX.PST? also 3SG?-echo 1SG-love-2SG\\
`Kombe went (back). Then he shouted, ``I love you!'' The mountain also echoed, ``I love you!"'

\ds{\textit{wéé} must be linearly preverbal; not sentence-final (44:08) nor postverbal (45:41)}\\
\ds{Extra agreement on `repeat'??}

\exg. ko\textipa{N}gwé (wéé) n-twéé bííjà (*wéé)\\
mountain also 1SG-name shout also\\
`The mountain also shouted my name'

\ds{Note order of arguments}

\section{Line 9 - Chun-Hung} 

\exg. Mb\`{e} \'{a} f\'{a} f\`{o}-\'{a}-j\`{e}, d\'{u}\textipa{\textltailn}\`{a}-t\textipa{S}\`{i}-\`{o} \'{e} f\'{e}n\`{\textipa{E}} mĩ́ gb\`{e} j-\'{a} m\`{a}sõ̂. \\
then 3SG father tell-AUX.PST-3SG world-DEM-in 2SG-NPST thing REL give 2SG receive \\
`Then his father told him, in this world whatever you give, you receive (it).' 

\chs{not sure what `e' is but looks like the subject of nonpast}

\section{The love series}
\exg. m-a tu-ni-jà\\
1SG-AUX.PST  love-?-2SG\\
``I love you.''

\exg. à wé tú-mú-jà\\ 
3\textsc{SG} also love-?-2SG\\
``He also loves you.''

\exg. m-bé tu-à-j-à\\
 1SG-NPST       2SG-PST\\
``I won't love you.''

\exg. m-bé tua wá̃ jà\\
1SG-NPST        2SG-PST\\
``I will love you.''

\exg. m tu-mu-jà\\
1SG   tu-mu-2SG\\
""

\exg. à má tù-nì-jà, m fá̃ tu-mu-jà\\
3\textsc{SG} \textsc{neg} love-?-2\textsc{sg}, 1\textsc{sg} self love-?-2\textsc{SG-AUX.PST}\\
``He doesn't love you, I love you.''

\exg. má tú-ní-à. m tu-mu-i wa n-a\\
\textsc{neg} love-?-3\textsc{sg}. 1\textsc{SG-AUX.PST} love-?-? ? ?\\
I dont love him/her, I love you (it's you I love).

\exg. m tu-mu-a, m tu-mu-i wè-à\\
1SG tu-mu-3SG, 1SG tu-mu-2SG also\\
"I love her, I also love you."

\end{document}