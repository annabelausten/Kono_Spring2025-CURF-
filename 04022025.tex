z   \documentclass{assets/fieldnotes}

\title{Kono (Sierra Leone)}
\author{LING3020/5020}
\date{University of Pennsylvania, Spring 2025\\04/02/2025 Class Projects Week 2}

\setcounter{secnumdepth}{4} %enable \paragraph -- for subsubsubsections

\begin{document}

\maketitle

\maketitle
\tableofcontents

\section{Realization of nasals(inserted/production)-Lex}


     
   
\exg.  n-tʃɛ́ tʃènù\\
1SG.GEN-husband many \\
`These husbands of mine'


\exg.  n-tʃɛ́ tʃɛ́-nù\\
1SG.GEN-husband this-Pl \\
`These husbands of mine'

\ex. wà\\
`big'

\ex.tʃénàmà\\
`big'

\ex. tʃélê\\
`one'

\ex.  tʃɛ̀\\
`this'

\ex. sɔ̀ɔ̀né\\
`hole'

\exg. sɔ̀né-tʃɛ̀ \\
hole-this\\
`this hole' 

\exg. soon-dʒé\\
hole-this\\
`this hole' 

\jf{Note change of vowel and vowel length for the "shorter version"}

\exg. sɔ̀n-bá-tʃɛ̀\\
hole-big-this\\
`this big hole'

\exg. sɔ̀n-tʃénàmà-tʃɛ̀\\
hole-big-this\\
`this big hole'


\ex. tʃénè
`house'

\exg. tʃé-námà\\
house-new\\
`new house'

\exg. tʃén-tʃénàmà\\
house-big\\
`big house'

\exg. tʃéŋ-wá\\
house-big\\
`big house'

\exg. tʃém-bá\\
house-big\\
`big house'

\exg. tʃén-dòndò\\
house-one\\
`one house'

\jf{ "tʃélê" only used for counting, "dòndò" used for referring to one of something}

\ex. sénɛ̀\\
`stone'

\exg. sénɛ̀-tʃɛ̀\\
stone-this\\
`this stone' 

\exg.sén-dʒɛ̀\\
stone-this\\
`this stone' 

\jf {Anthony says "sendʒé" is a shorter version}

\exg. sén-tʃénámà\\
stone-big\\
`big stone'

\exg.sénɛ̀-tʃénámà\\
stone-big\\
`big stone'

\exg. séŋ-wá\\
stone-big\\
`big stone'

\exg. sém-bá\\
stone-big\\
`big stone'

\exg. sén-dòndwuɛɛ\\
stone-one\\
`one stone'

\ex. kònɛ̄\\
`tree'

\exg.  kònɛ̀-tʃɛ̀\\
tree-this\\
`this tree'

\exg.kòn-dʒɛ̀\\
tree-this\\
`this tree'

\exg. kòm-bá\\
tree-big\\
`big tree'

 \exg. kòŋ-wá\\
tree-big\\
`big tree'

\exg. kòn-tʃénámà\\
tree-big\\
`big tree'

\jf{Shortened version not correct here}


\ex. dòmá\\
`shirt'

\exg. dòmá-tʃɛ̀\\
shirt-this\\
`this shirt'

\jf{ "Shortened version" not correct here}

\exg. dòmá-wá\\
shirt-big\\
`big shirt'

\exg. dòmá-tʃénámà\\
shirt-big\\
`big shirt'


\exg. màà-nù\\
banana-PL\\
`bananas'

\exg. màà-tʃé-nù\\
banana-this-Pl\\
`these bananas'

\exg. màà-tʃénámà\\
banana-big\\
`big bananas'

\exg. màà-wá\\
banana-big\\
`big bananas'


\exg. n-tʃínùá\\
1SG-sleep\\
`I slept'

\ex. tʃinuɛɛ\\
`sleep'

\exg. n-tʃè tʃínùá\\
1SG.GEN-husband sleep\\
`My husband slept'

\ex. ḿ̩-bwó bwénè-mù\\
1SG-body well-?\\
`I am healthy'

\section{Pronouns/auxiliaries - Alex}

\exg.
mɛ́mɛ́-ɛ̀N     dɔ    te-a    \\
mirror-PL   OBJ   break-A \\%
`The mirrors are breaking.'

\exg.
mɛ́mɛ́-ɛ̀N     nóŋgó   te-a    \\
mirror-PL   ONGO    break-A \\%
`The mirrors are breaking.'

\exg.
mɛ́mɛ́-ɛ̀N     dóŋgó   te-a    \\
mirror-PL   ONGO    break-A \\%
`The mirrors are breaking / about to break'

\exg.
kaŋganɛ-N   oŋgo   ton-da  \\
door-PL     ONGO   close-A \\%
`The doors are closing.'

\exg.
kaŋganɛ-N   (m)bé     ton-da  \\
door-PL     AUX.NEG   close-A \\%
`The doors are not closing.'

\exg.
ambé          ton-da  \\
3PL.AUX.NEG   close-A \\%
`They are not closing.'

\exg.
ʔaʔa,   ambé          ton     te-a \\
no,     3PL.AUX.NEG   close   ?-A  \\%
`No, they are not closing.'

\exg.
mɛmɛ-N      dɔ-i       te-a    \\
mirror-PL   OBJ-NEG?   break-A \\%
`The mirrors are not breaking.'

\exg.
mɛmɛ-N      (m)bé     te-a    \\
mirror-PL   AUX.NEG   break-A \\%
`The mirrors are not breaking.'

\exg.
kaŋan-ɔ    te:    ``mbé   ton-da    waN'' \\
door-OBJ   say:   ``1SG   close-A   FUT'' \\%
`The door said, `I will close''

\exg.
kaŋan-ɔ    te:    ``mbéé      ton-da    sina''     \\
door-OBJ   say:   ``1SG.NEG   close-A   tomorrow'' \\%
`The door said, `I will not close tomorrow''

\exg.
na         ɛ     tond-a    waN   sina     \\
1SG.OBJ?   AUX   close-A   FOC   tomorrow \\%
`I will close tomorrow.' \label{I will close tomorrow}

\alex{In fast speech, I hear the subject+aux form pronounced as \textit{nɛɛ}.}

\alex{The sentence in \ref{I will not close tomorrow} is the negative version of \ref{I will close tomorrow}. Note that you cannot simply use `tonal negation' in \ref{I will close tomorrow}. Thus, it seems there is something interesting happening in \ref{I will not close tomorrow}, namely in the ``loss'' of the object marker on the pronominal subject}

\exg.
mbéé          ton-da    sina     \\
1SG.AUX.NEG   close-A   tomorrow \\%
`I will not close tomorrow.' \label{I will not close tomorrow}

\exg.
andɔ      te-a    \\
3PL.OBJ   break-A \\%
`They broke.'

\exg.
andɔ      ma    te-ni    \\
3PL.OBJ   NEG   break-NI \\%
`They did not break.' \label{They did not break1}

\exg.
am    ma    te-ni    \\
3PL   NEG   break-NI \\%
`They did not break.' \label{They did not break2}

\alex{The difference between \ref{They did not break1} and \ref{They did not break2} seems to amount to the presence/absence of the `object marker', where in \ref{They did not break1}, the object marker is present and surfaces as \textit{dɔ} on the surface subject, where as in \ref{They did not break2}, the object marker is absent, in which case, the negation form cliticizes to the surface subject.}

\alex{Still not entirely sure that the \textit{-ni} particle is. Sometimes it looks like it is tracking negation in past/perfective forms. Other times it looks like it is tracking something like a past progressive?}

\exg.
a     ma    te-ni    \\
3SG   NEG   break-NI \\%
`It did not break.' \label{It did not break}

\alex{The difference between the 3PL subject in \ref{They did not break2} and the 3SG subject in \ref{It did not break} seems to surface as something like gemination of the nasal consonant + negation.}

\exg.
andɔɔ         te-a      waN \\
3PL.OBJ.AUX   break-A   FOC \\%
`They will break.'

\exg.
ámbéé         te-a    \\
3PL.AUX.NEG   break-A \\%
`They will not break.' \label{They will not break}

\exg.
ambɔ              te-a    \\
3PL.AUX.3SG.OBJ   break-A \\%
`They are breaking it.'

\exg.
anɔŋɡɔ     te-a    \\
3PL.ONGO   break-A \\%
`They are breaking.'

\exg.
andɔŋɡɔ    te-a    \\
3PL.ONGO   break-A \\%
`They are breaking.'

\exg.
sani-N      ɔŋɡɔ   te-a    \\
bottle-PL   ONGO   break-A \\%
`The bottles are breaking.' \label{The bottles are breaking}

\alex{Need to double-check whether \ref{The bottles are breaking} is a plural or singular subject.}

\exg.
ámbéé         te-a    \\
3PL.AUX.NEG   break-A \\%
`They are not breaking.' \label{They are not breaking}

\alex{Note that the negative + non-past in \ref{They are not breaking} shows the same form as the negation + future in \ref{They will not break}.}

\exg.
taa-n         dɔ    ni    te-a      kunu,        mbe    n-donda      tʃɛna   \\
calabash-PL   OBJ   AUX   break-A   yesterday,   then   1SG-enter?   house.P \\%
`The calabashes were breaking yesterday, then/when I entered the house.'

\exg.
an    dɔ    ni    te-a      kunu,        mbe    n-donda      tʃɛna   \\
3PL   OBJ   AUX   break-A   yesterday,   then   1SG-enter?   house.P \\%
`The calabashes were breaking yesterday, then/when I entered the house.'


\section{Glides - Jan}

\subsection{Attempting to further elicit [y]/[ɥ]}

\jmt{Last week:}

\exg. jaj yy à wíí\\
Jai water A boil \\
`Jai, boil the water.'

\jmt{This week:}

\exg. jaj ʔ íí à wìì \\
Jai {} water A boil \\
`Jai, boil the water.'

\exg. ná-íì \\ 
1SG.POSS-water \\
`my water'

\jmt{Expected:}

\ex. jaj à íí\\
`Jai's water'

\jmt{Got:}

\ex. jɛ́ɛ́ íí \\
`Jai's water'

\exg. íí jáj gbòò \\
water Jai hand \\
`Jai has water.'

\ex. íí wà jàj gbòò \\
`Jai has water.'

\ex. íí (ɥ) wá\\
`big river'

\ex. * íí úú \\
`short river'

\ex. * jaj à íí úú \\
`Jai's short river'

\ex. úú \\
`dog'

\jmt{Expected:}
\ex. jaj à úú \\
`Jai's dog'

\jmt{Got:}
\ex. jɛ́ á úù \\
`Jai's dog'

\jmt{Anthony said something akin to the first example when explaining the meaning of the sentence.}

\ex. ejamba \\
`leaf'

\exg. jáj ámbá wíí \\
Jai leaf boil \\
`Jai, boil the leaf.'

\ex. gwaj \\
`(to) work'

\ex. jaj wáj wàn tʃè \\
Jai work big it? \\
`Jai works hard.'

\ex. wì \\
`blood'

\ex. jáj wí \\
`Jai's blood'

\exg. jáj wí à wíí \\
Jai boil A blood \\
`Jai, boil the blood.'

\ex. kúì \\
`leopard'

\ex. kwíí wíí \\
`leopard's blood'

\subsection{CVCV + WV vs CVV + WV}

\ex. sɛɛ \\
`instrument'

\ex. sɛ́ɛ́ ámà \\
`bad sɛɛ'

\jmt{The [j] at the beginning of `bad' is missing.}

\ex. sɛ́ɛ́ wá \\
`big sɛɛ'

\ex. sɛ́ɛ́ wɔ́ɔ́lwè \\
`six sɛɛs'

\ex. wɛwɛ \\
`whisper'

\ex. wɛ́wɛ́ ámà \\
`bad whisper'

\ex. wɛ́wɛ́ ʔ ɔ́ɔ́lwè\\
`six whispers'

\ex. kuɛɛ̀ \\
`rice'

\ex. kwɛ́ɛ́ ámà \\
`bad rice'

\ex. kúíjɛ̀ \\
`voice'

\ex. kwíjɛ́ ámà \\
`bad voice'

\subsection{Clarifying numerals}

\ex. kàw \\
`moon'

\ex. káw dúú \\ 
`five moons'

\ex. ká ɔ́ɔ́lwè \\
`six moons'

\ex. ká òɱfèà \\
`seven moons'

\ex. káw kónòntwè \\ 
`nine moons' 

\jmt{The [o] $\sim$ [wè] variation shows up again.}





\section{Long-Distance Questions - Joey}

\exg. kai-ja feɱ fɔ-je bondu a feɱ fɔ kunu\\
Kai-3SG what say-3SG bondu 3SG what say yesterday\\
`What did Kai say that Bondu said yesterday?'\\ 

\jf{Answer:}
\exg. ɔɔ bondu ɔ te... \\
? Bondu 3SG.ɔSER say...\\
`He said Bondu said...'\\ 

\jf{Unaccusative with subject "what" (context: a package arrived):}
\exg. kai ɔ te fen na kunu\\
Kai 3SG.ɔSER say what arrive yesterday\\
`What did Kai say arrived yesterday?'\\ 

\exg. kai-ja feɱ fɔ-je fen na kunu\\
Kai-3SG what say-3SG what arrive yesterday\\
`What did Kai say arrived yesterday?'\\ 

\jf{Answer:}

\exg. kai ɔ te doɱfena na kunu\\
Kai 3SG.ɔSER say food arrive yesterday\\
`Kai said that food arrived yesterday.'\\ 

\jf{Recap from last time:}

\exg. kai-ja feɱ fɔ-*(je) fen-a bondu ee kunu   \\
Kai-3SG what say-3SG what-3SG Bondu see yesterday\\
`What did Kai say saw Bondu yesterday?'\\ 

\jf{Can we use feŋ expletive with who? - this was verified, so yes.}

\exg. √kai-ja feɱ fɔ-*(je) ɲo-na bondu ee kunu   \\
Kai-3SG what say-3SG who-FOC Bondu see yesterday\\
`Who did Kai say saw Bondu yesterday?'\\ 

\jf{Reconfirm, and re-prime following structure:}

\exg. kai a feɱ fɔ-je bondu a fen sa kunu?\\
Kai 3SG what say-3SG Bondu AUX.PST what buy yesterday\\
`What did Kai say that Bondu bought yesterday?' \\ 

\jf{Can we use te with wh-expletive?}

\exg. *kai a fẽ ɔ-te-je bondu a fen sa kunu?\\
Kai 3SG what 3SG.ɔSER-say-3SG Bondu AUX.PST what buy yesterday\\
`What did Kai say that Bondu bought yesterday?' \\ 
i
\jf{Tony corrected it to:}
\exg. kai ɔ te bondu a fen sa kunu?\\
Kai 3SG.ɔSER say Bondu AUX.PST what buy yesterday\\
`What did Kai say that Bondu bought yesterday?' \\

\exg. kai a fen te-je bondu a fen sa kunu?\\
Kai 3SG what say-3SG Bondu AUX.PST what buy yesterday\\
`What did Kai say that Bondu bought yesterday?'  \jf{(Not tested, but expected ungrammaticality based on previous response.}\\ 


\jf{See different embedded pronouns (also, embedded deixis):}

\exg. kai$_{1}$ a feɱ fɔ-je a$_{1}$ fen sa kunu \\
Kai$_{1}$ 3SG what say-3SG AUX.PST$_{1}$ what buy yesterday\\
`What did Kai$_{1}$ say that he$_{1}$ bought yesterday?' \\

\jf{(This was also given, and almost feels like reverse prolepsis (with the "concerning" in the embedded clause)}\\

\exg. kai$_{1}$ a feɱ fɔ-je kaama a$_{1}$ fen sa kunu \\
Kai$_{1}$ 3SG what say-3SG concerning(?) AUX.PST$_{1}$ what buy yesterday\\
`What did Kai$_{1}$ say that he$_{1}$ bought yesterday?' \\

\jf{Answer (interestingly, seems to always default to te for answer, even when fɔ is in question):}\\

\exg. kai$_{1}$ ɔ te a$_{1}$ swee ã-sa kunu\\
Kai$_{1}$ 3SG.ɔSER say 3SG.PST$_{1}$ meat FOC(?)-buy yesterday\\
`Kai$_{1}$ said that he$_{1}$ bought meat yesterday' \\ 


\exg. kai a feɱ fɔ-je n-a fen sa kunu\\
Kai 3SG what say-3SG 1SG-AUX.PST  what buy yesterday\\
`What did Kai say that I bought yesterday?' \\

\jf{(Tony reports that the "te" version is "much simpler" and thus more common)}

\exg. kai ɔ te n-a fen sa kunu\\
Kai 3SG.ɔSER  say 1SG-AUX.PST  what buy yesterday\\
`What did Kai say that I bought yesterday?' \\



\section{Definiteness - Mingyang}
\begin{itemize}
 \item Follow-up on part-whole briding:
 \exg. n-a t͡ʃene sã. a-da kene.\\
        1SG-AUX.PST house buy 3.SG-door open\\
        `I bought a house. \textbf{The door} was open.'

 \exg. n-a t͡ʃene sã. t͡ʃɛ̀ mbe a-da ɲaan tea wã.\\
        1SG-AUX.PST house buy bought 1SG-NPST 3.SG-door fix DUR WAN\\
        `I bought a house. I need to fix \textbf{the door}.'(Lit. `...fix its door.)

 \item Descriptive anaphora
   \exg. kajne kante-a. t͡ʃɛ kajne sio ba wã.\\
    boy place.of.learning-A but boy sit back WAN\\
   `There is a boy in the classroom, but the boy is sitting in the back.'

   \jal{school might be a better translation for "place of learning", eh? or stick with place of learning.  Also,  once he's told you he doesn't have a word for classroom, you should pivot immediately away from asking for that}
   \mb{Got it, thanks!}

   \jal{Do listen again to the recording.  I have the second one as kain dʒɛ = boy this, which is a crucial difference for you. He said it seven times, the second was as you have written but it was very slow, and the rest of the times were clearly kain dʒe (and he translated it twice as "this boy" and once as "that same boy").  I also have dʒe on your other examples in this section}\mb{Please see my notes in the document for 0409}
   

   \exg. kajne kante-a. n kwa wã kajne tea kunu.\\
   boy place.of.learning-A 1.SG speak WAN boy P yesterday\\
   `There is a boy in the classroom. I talked to the boy yesterday.'

   \jal{again kain dʒɛ}

   \mb{Anthony said explicitly that te-a here means `him'; I wonder if this is further decomposed as some morpheme te plus 3.SG -a.}\\
   \jal{I think this is because of the verb `speak'.  Notice last year: } 
   \mb{Got it, thank you!}
   \exg.[(1930)] ń-koà i tea kùnù\\
1SG-speak 2SG P yesterday\\
‘I spoke to you yesterday’



   \exg. kajne n-fea dumusune-a ambe kante-a. dumusune sio kajne bema.\\
   boy and girl-A 3.PL place.of.learning-A girl sit boy near\\
   There is a boy and a girl in the classroom. The girl is sitting next to the boy.

   \jal{again: dumusun dʒɛ sio kain dʒa mbe ma   he said "this girl" and "near the boy"  }


    
\end{itemize}

\section{Relative Clauses -- Wesley}


\exg. Bondu túmú-á dè-á\\
{} love-\textsc{AUX.PST} mother-\textsc{a}\\
`Bondu loves his mother.’\\
\wml{At first I was confused about \textit{mother-\textsc{a}} being post-verbal, but then realised we've seen other examples of animate objects/internal arguments being post-verbal, e.g. the recipient of \textit{tell} in the 19 March elicitations. So perhaps this isn't so surprising, and the \textsc{-a} is the regular object marking we see generally.}

\exg. Bondu dè nàà Bajama sìná\\
{} mother come {} tomorrow\\
`Bondu’s mother is coming to Bajama tomorrow.’

\exg. Bondu túmú-á dè-á mím-bè nàà Bajama sìná \\
{} love-\textsc{AUX.PST} mother-\textsc{a} \textsc{mi-we} come {} tomorrow\\
`Bondu loves his mother, who is coming to Bajama tomorrow.’

\ex. sɔ̀nɛ́\\`to know (somebody)’

\exg. Kai a Bondu sɔ́ɱfã̀/*sɔ̀nɛ́\\
{} \textsc{AUX.PST} {} know\\
`Kai knows Bondu.’

\ex. wã̀\\`live, inhabit’

\exg. Bondu sìì-jɔ́ Bajama wã̀\\
{} sit-? {} live\\
`Bondu lives in Bajama.’\\
\wml{Tony said \textit{sìì} means sit---not sure then what \textit{-jɔ́} means.}

\exg. Bondu mín-sìì mú Bajama, Kai a sɔ́ɱfã̀\\
{} \textsc{mi}-sit \textsc{mu} {} {} \textsc{AUX.PST} know\\
`Bondu, who lives in Bajama, Kai knows him.’

\exg. Kai a Bondu sɔ́ɱfã̀, mín-sìì mú Bajama\\
{} \textsc{AUX.PST} {} know \textsc{mi}-sit \textsc{mu} {}\\
`Kai knows Bondu, who lives in Bajama.’

\exg. Bondu mú kààmwɛ́ nà\\
{} \textsc{mu} teacher \textsc{na}\\
`Bondu is a teacher.’

\exg. Kai à Bondu sɔ́ɱfã̀, mím-mú kààmwɛ́ à\\
{} \textsc{AUX.PST} {} know \textsc{mi-mu} teacher \textsc{a}\\
`Kai knows Bondu, who is a teacher.’

\exg. Bondu mím-mú kààmwɛ́ à, Kai à sɔ́ɱfã̀\\
{} \textsc{mi-mu} teacher \textsc{a} {} \textsc{AUX.PST} know\\
`Bondu, who is a teacher, Kai knows him.’

\exg. Bondu (*à)-jã̀sã̀\\
Bondu (*\textsc{3sg})-tall\\
`Bondu is tall.’



\exg. Kai à Bondu sɔ́ɱfã̀, mĩ́ jã̀sã̀\\
{} \textsc{AUX.PST} {} know \textsc{mi} tall\\
`Kai knows Bondu, who is tall.’

\exg. Bondu mín jã́sã̀, Kai à sɔ́ɱfã̀\\
{} \textsc{mi} tall \textsc{AUX.PST} {} know\\
`Kai knows Bondu, who is tall.’

\wml{Tony seems to produce mín with a nasal before jã́sã̀ in the topicalised version but mĩ́ (with no nasal stop) in the extraposed one, where he pauses. I guess this is likely due to the pause rather than anything else about the local structure of the sentences, which are the same, i.e. \textsc{mi}+jã́sã̀.}

\wml{For the following, with appositives modifying pronominal subjects, we see part of a paradigm with \textit{fɛ́nà} and \textit{wɛ́nà} forms similar to what Giang got for focus: \textit{fana, wana...}. Could these all be part of the same paradigm as \textit{ana} which we've been seeing here and there? I will follow up by testing these sentences with past tense.}

\exg. í-wɛ́nà Bondu sɔ̃̀, é nàà wã̀ sìná\\
\textsc{2sg}-\textsc{wena} Bondu know \textsc{2sg.ser3} come \textsc{wa} tomorrow\\
`You (sg.), who know Bondu, are coming tomorrow.’

\exg. ó-wɛ́nà Bondu sɔ̃̀, wé nàà wã̀ sìná\\
\textsc{2pl}-\textsc{wena} Bondu know \textsc{2pl.ser3} come \textsc{wa} tomorrow\\
`You (pl.), who know Bondu, are coming tomorrow.’

\exg. ɱ́-fɛ́nà Bondu sɔ̃̀, mbé nàà wã̀ sìná\\
\textsc{1sg}-\textsc{wena} Bondu know \textsc{1sg.ser3} come \textsc{wa} tomorrow\\
`I (sg.), who know Bondu, am coming tomorrow.’

\jal{data from very end of last time, so didn't get to pursue; would need to go back and double check tones. I glossed it as `strong', but that's subject to revision}
\exg.  ɱ́fã ni Bòndû n-à kuɛ tu kunu\\
1\textsc{sg.strong} and Bondu 1\textsc{1EXCL-AUX.PST} rice pound yesterday\\
`I and Bondu, we pounded rice yesterday.' (05/08/2023)

\exg.  Bòndû ni ɱ́fã *(nà) kuɛ tu kunu\\
Bondu and 1\textsc{sg.strong} 1\textsc{1EXCL-AUX.PST} rice pound yesterday\\
`Bondu and I, we pounded rice yesterday.' (05/08/2023)





\wml{Follow-up: try perfect/past tense with pronominal heads of appositives---do we get the wana/fana versions that Giang elicited?}

\exg. Bondu à mùsù jẽ̀ẽ̀ {[} mì-na sɛ̀ɛ̀ ɔ̀ té {]}\\
Bondu \textsc{AUX.PST} woman see {} \textsc{mi-na} sɛɛ \textsc{3sg.obj} break {}\\
`Bondu saw the \textbf{woman} who broke the sɛɛ.’

\wml{Here trying a prosodically/phonologically lighter noun than \textit{mwokama} to see if that lends itself to a RC between the head noun and verb.}
\exg. Bondu à úú jẽ̀ẽ̀ {[} mín-à sɛ̀ɛ̀ ɔ̀ té {]}\\
Bondu \textsc{AUX.PST} dog see {} \textsc{mi-na} sɛɛ \textsc{3sg.obj} break {}\\
`Bondu saw the \textbf{dog} who broke the sɛɛ.’

\exg. *Bondu à {[} úú mí-nà sɛ̀ɛ̀ ɔ̀ té {]} jẽ̀ẽ̀\\
Bondu \textsc{AUX.PST} {} dog \textsc{mi-na} sɛɛ \textsc{3sg.obj} break {} see\\
`Bondu saw the \textbf{dog} who broke the sɛɛ.’\\
\wml{Proposed this sentence to Tony and he didn’t seem to like it---he preferred the topicalised version instead.}

\exg. {[} úú mí-nà sɛ̀ɛ̀ ɔ̀ té {]} Bondu à jẽ̀ẽ̀\\
{} dog \textsc{mi-na} sɛɛ \textsc{3sg.obj} break {} Bondu \textsc{AUX.PST} see\\
`Bondu saw the \textbf{dog} who broke the sɛɛ.’

\exg. Bondu à úú jẽ̀ẽ̀ {[} mí-nà sɛ̀ɛ̀ ɔ̀ té kúnù {]}\\
Bondu \textsc{AUX.PST} dog see {} \textsc{mi-na} sɛɛ \textsc{3sg.obj} break yesterday {}\\
`Bondu saw the \textbf{dog} who [broke the sɛɛ yesterday].’

\exg. {[} úú mí-nà sɛ̀ɛ̀ ɔ̀ té kúnù {]} Bondu à jẽ̀ẽ̀\\
{} dog \textsc{mi-na} sɛɛ \textsc{3sg.obj} break yesterday {} Bondu \textsc{AUX.PST} see\\
`Bondu saw the \textbf{dog} who [broke the sɛɛ yesterday].’

\exg. {[} úú mí-nà sɛ̀ɛ̀ ɔ̀ té kúnù {]} Bondu à jẽ̀ẽ̀\\
{} dog \textsc{mi-na} sɛɛ \textsc{3sg.obj} break yesterday {} Bondu \textsc{AUX.PST} see\\
`Bondu saw the \textbf{dog} who [broke the sɛɛ yesterday].’

\exg. Bondu à úú jẽ̀ẽ̀ kúnù {[} mí-nà sɛ̀ɛ̀ ɔ̀ té {]}\\
Bondu \textsc{AUX.PST} dog see yesterday {} \textsc{mi-na} sɛɛ \textsc{3sg.obj} break {}\\
`Bondu saw the \textbf{dog} who [broke the sɛɛ] yesterday.’

\exg. {[} úú mí-nà sɛ̀ɛ̀ ɔ̀ té kúnù {]} Bondu à jẽ̀ẽ̀ kúnù\\
{} dog \textsc{mi-na} sɛɛ \textsc{3sg.obj} break yesterday {} Bondu \textsc{AUX.PST} see\\
`Yesterday, Bondu saw the \textbf{dog} who [broke the sɛɛ yesterday].’

\wml{Things to follow up on:
- Relativising on possessor
- Relativising on complement of postposition
- Relativising pronouns in past tense
- Is wena obligatory?
}



\section{Predicative possessives - Chun-Hung}


\chs{\textbf{Plan of this week}: continue getting a descriptive sketch and try to get some comparisons between possessives and locatives}

\chs{inalienable relations (others: sister, son, daughter)}
\newline

\chs{
\textbf{Constructions}: \newline
series.I-brother + wan $\rightarrow$ temporary possession?
\newline
series.I-brother + wanu $\rightarrow$ temporary possession? 
\newline
brother + wa + series.I-hand
\newline \newline
*1SG.I-want.to [1SG money ma-wa-1SG.I-hand in the future]. $\rightarrow$ possessor in hand-construction cannot be controlled PRO, probably not a grammatical subject 
\newline \newline
1SG.I-want.to [1SG money get/receive in the future]. $\rightarrow$ possessor in transitive-like construction be controlled PRO
\newline \newline
C. (2025) points that the default pattern for predicative possessives is expressed by `hand' throughout out the Mande family; only few of them have typical transitive (bivalent) constructions, which may be due to the language contact with other language families.
}


\exg. \'{N}-k\^{o} wã̀. \\
1SG.SERI-older.brother WA \\
`I have an older brother.' \chs{as I'm not alone; I have my brother with me.}

\exg. \'{N}-k\^{o} w\'{a}n\`{u}. \\
1SG.SERI-older.brother WANU \\
`I have an older brother.' \chs{as I'm not alone; I have my brother with me.}

\exg. k\'{o}\`{o} w\'{a} \textipa{N}m-gb\'{\textipa{O}}\`{\textipa{O}}. \\
older.brother WA 1SG.SER1-hand \\
`I have an older brother.' 

\exg. \'{I}-k\^{o} wã̀. \\
2SG.SERI-older.brother WA \\
`You have an older brother.'

\exg. \'{I}-k\^{o} w\'{a}n\`{u}. \\
2SG.SERI-older.brother WANU \\
`You have an older brother.' 

\exg. k\'{o}\`{o} w\'{e} \'{i}-gb\'{\textipa{O}}\`{\textipa{O}}. \\
older.brother WA 2SG.SER1-hand \\
`You have an older brother.' 

\exg. \`{A}-k\^{o} wã̀. \\
3SG.SERI-older.brother WA \\
`He has an older brother.'

\exg. \`{A}-k\^{o} w\'{a}n\`{u}. \\
3SG.SERI-older.brother WANU \\
`He has an older brother.' 

\exg. k\'{o}\`{o} w\'{a} \`{a}-gb\'{\textipa{O}}\`{\textipa{O}}. \\
older.brother WA 3SG.SER1-hand \\
`He has an older brother.' 

\exg. B\`{o}nd\'{u}-k\^{o} wã̀. \\
Bondu-older.brother WA \\
`Bondu has an older brother.'

\exg. B\`{o}nd\'{u}-k\^{o} w\'{a}n\`{u}. \\
Bondu-older.brother WANU \\
`Bondu has an older brother.' 

\exg. k\'{o}\`{o} w\'{a}  B\`{o}nd\'{u}-gb\'{\textipa{O}}\`{\textipa{O}}. \\
older.brother WA Bondu-hand \\
`Bondu has an older brother.' 


\exg. \'{N}-k\^{o} mb\'{e} wã̀. \\
1SG.SERI-older.brother PL WA \\
`I have older brothers.' \chs{as I'm not alone; I have my brothers with me.}

\exg. \'{N}-k\^{o} mb\'{e} w\'{a}n\`{u}. \\
1SG.SERI-older.brother PL WANU \\
`I have older brothers.' \chs{as I'm not alone; I have my brothers with me.}

\exg. k\'{o}\`{o} mb\'{e} w\'{a} \textipa{N}m-gb\'{\textipa{O}}\`{\textipa{O}}. \\
older.brother PL WA 1SG.SER1-hand \\
`I have older brothers.' 

\chs{Controlled PRO}

\exg. \'{N}-t\'{u}-m\`{u} nĩ́ k\^{o}s\`{o}(-wã̀) s\`{o}s\^{o}m\`{a} s\'{i}n\'{a}. \\
1SG.SER1-like-MU 1SG get.up-WA early tomorrow \\
`I want [to get up early tomorrow].' \chs{regular subject targeted}

\exg. Mb\'{e} k\'{o}pw\`{e} m\'{a}s\`{o}nd-a-wã́ j\`{a}j.  \\
1SG.SER3 money get-A-WA next.year \\
`I will have money next year.'

\exg. \'{N}-t\'{u}-m\`{u} nĩ́ k\'{o}pw\`{e} m\'{a}sõ̀ j\`{a}j.  \\
1SG.SER1-like-MU 1SG money get next.year \\
`I want [to have money next year].' \chs{possessor targeted}

\exg. \'{A}-t\'{u}-m\`{u} \'{a}n\`{i} k\'{o}pw\`{e} m\'{a}sõ̀ j\`{a}j.  \\
3SG.SER1-like-MU 3SG money get next.year \\
`He wants [to have money next year].' \chs{possessor targeted}

\exg. K\'{o}pw\`{e} m\'{a}-wã́ \textipa{N}-gb\'{\textipa{O}}\`{\textipa{O}} j\`{a}j.  \\
money FUT-WA 1SG.SERI-hand next.year \\
`I will have money next year.'

\exg. *\'{N}-t\'{u}-m\`{u} nĩ́  k\'{o}pw\`{e} m\'{a}-wã́ \textipa{N}-gb\'{\textipa{O}}\`{\textipa{O}} j\`{a}j.  \\
1SG.SER1-like-MU 1SG money FUT-WA 1SG.SERI-hand next.year \\
`I want to have money next year.' \chs{possessor targeted; cannot be controlled PRO}

\ex. Bondu wants [to have money in the future]. \chs{possessor targeted}

\ex. I regret [breaking the calabash yesterday]. \chs{regular subject targeted}

\ex. I regret [not having a car before]. \chs{possessor targeted}

\ex. Bondu regrets [not having a car before]. \chs{possessor targeted}

\ex. I want [to be in the house today]. \chs{subject of locatives targeted}

\ex. The snake wants [to be on the stone today]. \chs{subject of locatives targeted} 


\chs{Topicalization}

\exg. t\textipa{S}\'{\textipa{E}}n\`{e} w\'{a}-k\`{a}n\'{i}nd\textipa{Z}\`{e}-b\'{o}\`{o}. \\
house WA-student-hand \\
`The student has a house.'

\exg. k\`{a}n\'{i}nd\textipa{Z}\`{e}, t\textipa{S}\'{\textipa{E}}n\`{e} w\'{a}-\'{a}-b\'{o}\`{o}. \\
student house WA-3SG.I-hand \\
`The student, he has a house.' \chs{possessor targeted}

\ex. The dogs is in the house. 

\ex. The house, the dog is in it. \chs{object of postposition targeted}

\ex. The snake is behind Bondu. 

\ex. Bondu, the snake is behind. 

\chs{Focalization}

\ex. It's Bondu that has a house (not me). \chs{possessor targeted}

\ex. It's Bondu that has an older brother (not me). \chs{possessor targeted}

\ex. It's a house that Bondu has (not a car). \chs{possessum targeted}

\ex. It's a lion that Bondu has (not a dog). \chs{possessum targeted}

\ex. It's the house that the dog is in. \chs{object of postposition targeted}

\ex. It's the stone that the snake is on. \chs{object of postposition targeted}

\ex. It's the dog that is in the house. \chs{subject of locatives targeted}

\ex. It's the snake that is on the stone. \chs{subject of locatives targeted} 

\chs{Relativization}

\ex. The man that has a house cried. \chs{possessor targeted}

\ex. The boy that has an older brother. \chs{possessor targeted}

\ex. The house that the man has is beautiful. \chs{possessum targeted}

\ex. The dog that the boy has is barking. \chs{possessum targeted}

\ex. The house that the dog is in is beautiful. \chs{object of postposition targeted}

\ex. The snake that the stone is on is scary. \chs{object of postposition targeted}

\ex. The dog that is in the house is barking. \chs{subject of locatives targeted}

\ex. The snake that is on the stone is scary. \chs{subject of locatives targeted}

\chs{others}

\ex. I have the key (Context: You are staying at a hotel with your friend. Your friend is wondering
where the key to the room has gone. You say:).

\ex. I have that book.

\ex. I have your book. 

\ex. Kai has a big hand. [body part]

\ex. Bondu has black eyes. [body part]

\ex. Kai has a lot of strength. [attribute]

\ex. The house has windows. [part-whole]

\ex. The s\textipa{E}\textipa{E} has stings. [part-whole]

\section{Wh/focus - Giang}
\g{Probe for wã́}
\exg. fé m-be ma ne (*wã́) sena?\\
what \textsc{foc-aux} happen here ? tomorrow\\
What will happen tomorrow?

\exg. fe m-ma (ne) (*wã́) kùnù?\\
what \textsc{foc}-happen? here ? yesterday\\
What happened yesterday?

\exg. fe mbe ma téà?\\
what ? happen now\\
What is happening now?

\g{Testing wã́ with past tense sentences}
%past tense, unerg
\exg. ɲɔ́-n-à dí-t\textipa{S}\`{\textipa{E}} (*wã́) kùnù?\\
Who-\textsc{foc-aux} cry-\textit{v} ? yesterday\\
``Who cried yesterday?''

\ex. Bondu.

\ex. *Bondu a-na.

\exg. Bondu án-à dí-t\textipa{S}\`{\textipa{E}} (*wã́) kùnù\\
Bondu \textsc{FOC-AUX} cry-\textit{v} ? yesterday.\\
``It was Bondu that cried yesterday.''


\g{Test for all persons and number. See if there is variation in a-na.}

\exg. ɱ fana dí-t\textipa{S}\`{\textipa{E}} kùnù\\
1sg f-ana cry-v yesterday\\
``It was me who cried yesterday.''

\exg. mfá ánà dí-t\textipa{S}\`{\textipa{E}} kùnù\\
1pl-fa ana cry-v yesterday\\
``It was us who cried yesterday.''

\exg. i wánà dí-t\textipa{S}\`{\textipa{E}} kùnù\\
2sg w-ana cry-v yesterday\\
``It was you(sg) who cried yesterday.''

\exg. o wán-à dí-t\textipa{S}\`{\textipa{E}} kùnù\\
2pl \textsc{foc-3sg.aux} cry-v yesterday\\
``It was you(pl) who cried yesterday.''


\exg. á ánà dí-t\textipa{S}\`{\textipa{E}} kùnù\\
3sg w-ana cry-v yesterday\\
``It was him who cried yesterday.''

\exg. am f-ana dí-t\textipa{S}\`{\textipa{E}} kùnù\\
3pl f-ana cry-v yesterday\\
``It was them who cried yesterday.''

\exg. Bondu \'{a} d\'{i}-t\textipa{S}\`{\textipa{E}} (wã́) k\`{u}n\`{u}. \\
Bondu A cry-\textit{v} ? yesterday \\
`Bondu cried yesterday.'


\exg. Bòndú á swéè dàó̃ wã́ kùnù\\
Bondu AUX.PST meat eat ? yesterday \\%
    `Bondu ate meat yesterday'

\exg. ɲɔ́nà swéè dàó̃ (*wã́) kùnù?\\
who meat eat ? yesterday\\
``Who ate meat yesterday?''

\ex. Bondu.

\ex. *Bondu a-na.

\exg. Bòndú án-à swéè dàó̃ (*wã́) kùnù\\
Bondu \textsc{foc-AUX.PST} meat eat ? yesterday \\%
    `Bondu ate meat yesterday'\\
    
\g{Can say this out of the blue, but it's more of an answer to a question.}

\exg. Bòndú á fé(n) dàó̃ (*wã́) kùnù?\\
Bondu \textsc{AUX.PST} what eat ? yesterday\\
``What did Bondu eat yesterday?''

\g{Get possible answers for the previous questions. Check for adding wã́.}

\ex. swéè

\exg. Bondu a swee ɛn dao kunu.\\
Bondu AUX.PST meat ? eat yesterday\\
``Bondu ate meat yeserday.''

\exg. Bondu a swee ɛn dao (*wã́) kunu.\\
Bondu AUX.PST meat ? eat ? yesterday\\
``Bondu ate meat yeserday.''

\exg. Bondu a swee dao (wã́) kunu.\\
Bondu AUX.PST meat  eat ? yesterday\\
``It was that Bondu ate meat yeserday.''



\section{Wh/focus - Daniel}

\ds{Several previous data points have wã in non-future contexts, but most are progressive/habitual/modal, one or two examples with wã in past(?) -- reprobe those\\
Start working towards distinguishing wã from fã (mostly in questions, see 03052025)\\
Verum focus and whether predicate/sentential focus in general is somehow different from verum}


\exg. Bondu-a tombw\textipa{E} dom fa kunu?\\
Bondu-3SG dance ? FOC? yesterday\\
`Did Bondu dance yesterday?'\\
`Did \textsc{Bondu} dance yesterday?'\\
`Did Bondu \textsc{dance} yesterday?'\\
\ds{Comment: best if you were previously certain that Bondu is going to dance, i.e. some kind of verum}

\exg. A-a, Bondu tombwe do wã\\
no Bondu dance \textit{v} FOC\\
`No, Bondu didn't dance'

\exg. Àà, Bondu-a tombwe dom-fã/*wã kunu\\
yes Bondu-3SG dance \textit{v}-\textit{fã} yesterday\\
`Yes, Bondu did dance yesterday'

\exg. Kunu (fã) nì Bondu-a tombwe do\\
yesterday \textit{fã} ? Bondu-3SG dance \textit{v}\\
`Was it yesterday that Bondu danced?'
\ds{fã is optional; is nì the same as negation, i.e. some kind of high negation/biased question?}

\exg. Bondu-a tombwe du banũ fã\\
Bondu-3SG dance \textit{v} last.year \textit{fã}\\
`Bondu danced last year'

\exg. Bondu-a tombwe du kunũ fã\\
Bondu-3SG dance \textit{v} last.year \textit{fã}\\
`Bondu danced last year'
\ds{Both in response to a previou assertion/to the question above, weirder out of the blue}

\exg. Bondu-a tombe-du banũ fã?\\
Bondu-3SG dance-? yesterday ?\\
`Was it \textsc{last year} that Bondu danced?'\\

\exg. A-a, kùnù nì/*fã/wã\\
no yesterday ?/fã/wã\\
`No, it was yesterday'

\exg. A-a, Bondu-a tombwe-du bìì(-wã)/*fã\\
No Bondu-3SG dance-? today-FOC(???)\\
`No, Bondu danced \textsc{today}'
\ds{This can only be in the past, neither present nor future readings available)

\exg. Bondu tombwe-ndo da-wã bìì\\
Bondu dance-\textit{v} ? \textit{wã} today\\
`Bondu will dance today'
\ds{Usable out of the blue}



\ds{Verum -- use tia-tia from last time. Each of these with wã/fã}

\exg. Tià-tià Bondu-a tombwe dom-fa\\
true-true Bondu-3SG dance \textit{v}-fã\\
\glt `Did Bondu really dance? (lit.: is it true that Bondu danced?)

\exg. Àà, Bondu-a tombwe\\
yes Bondu-3SG dance\\
`Yes, Bondu danced'

\exg. Àà, tombwe dom *(fã)\\
yes dance \textit{v} \textit{fã}\\
`Yes, he danced'
\ds{fã obligatory in responses, this has argument drop albeit not full ellipsis (see if possible `Àà, dom fã'?)}

\end{document}