\documentclass{assets/fieldnotes}

\title{Kono (Sierra Leone)}
\author{LING3020/5020}
\date{University of Pennsylvania, Spring 2025\\04/09/2025 Class Projects Week 3}

\setcounter{secnumdepth}{4} %enable \paragraph -- for subsubsubsections

\begin{document}

\maketitle

\maketitle
\tableofcontents

\section{Joey}

\jf{See if different matrix pronouns change the -je suffix (confirms that it's different pronouns + je preposition):}

\exg. kàì á fèɱ fɔ̀-ì-jé bòndú á fèn sàŋ kùnù?\\
Kai 3SG what say-2SG-to Bondu AUX.PST what buy yesterday\\
`What did Kai tell you (sg.) that Bondu bought yesterday?'  

\exg. kai a feɱ fɔ-n-d͡ʒe bondu a fen saŋ kunu?\\
Kai 3SG what say-1SG.to Bondu 3SG what buy yesterday\\
`What did Kai tell me that Bondu bought yesterday?' \jf{(Context: I forgot, and you were there and heard the conversation, so please remind me)}

\jal{This is a good example for Lex}

\exg. kai a feɱ fɔ-a-je bondu a fen saŋ kunu?\\
Kai 3SG what say-3SG.to Bondu 3SG what buy yesterday\\
`What did Kai tell him that Bondu bought yesterday?' \}

\jf{If yes, Answer:}

\exg. ɔ̀ té á swèè-àn sã̀\\
3SG.OBJ say 3SG.PST meat-FOC buy\\
`Kai said that Bondu bought meat \jf{(notably, not `Kai told you that bondu bought meat')}' 

\jf{Reconfirm, and prime alternate translation offered:}

\exg. kàì á fèɱ fɔ̀-ì-jé bòndú á fèn sàŋ kùnù?\\ 
Kai 3SG what say-3SG Bondu AUX.PST what buy yesterday\\
`What did Kai say regarding what Bondu bought yesterday?'  

\jf{Can you have a matrix and embedded question?}
\exg. Tisa\\
`Ask' \\ 

\jf{1) Doesn't seem to be a lot of optionality with different kinds of CP embedding V's. 2) The ɔ object seems to be agreeing with a listener, not the CP. 3) There seems to be a proleptic-ish translation (although the "concerning" is embedded instead of matrix), but there is no clear proleptic object here.}\\

\exg. kai a jɔ tisa bondu a fen saŋ kunu\\
Kai 3SG 2SG.ɔSER ask Bondu AUX.PST what buy yesterday\\
`What did Kai ask you concerning what Bondu bought yesterday?'  \jf{Translation feels proleptic, but there doesn't appear to be a proleptic object}\\

\exg. kai a tisa-t͡ʃe kaama bondu a fen saŋ kunu\\
Kai 3SG ask-t͡ʃe concerning Bondu AUX.PST what buy yesterday\\
`What did Kai ask concerning what Bondu bought yesterday?' \\


 \jf{The previous examples above suggest that the ɔ-SER object isn't obj agreement with embedded CP. It refers to the person who was told. 2SG in the first example, no one in the second (interesting that the second one does not require this object)}
 
\jf{If the above is good, this may suggest prolepsis?}\\

\jf{Other verbs that can take embedded questions?}\\

\ex. inat͡ʃii \\
`Think' \\ 

\ex. m-wajɔ\\
`It seems to me' \jf{(not remembered by Tony. Possibly mispronounced by me)}\\ 


\exg. kai a inat͡ʃii bondu a fen saŋ kunu\\
Kai 3SG think Bondu AUX.PST what buy yesterday\\
`What does Kai think Bondu bought yesterday?' \

\exg. *kai a fẽ inat͡ʃii bondu a fen saŋ kunu\\
Kai 3SG what think Bondu AUX.PST what buy yesterday\\
`What does Kai think Bondu bought yesterday?' \ \jf{(Can't put the wh-expletive-type thing here. Suggests that perhaps this isn't prolepsis? Proleptic objects have no limitation as to what kind of CP-comp-taking verb they occur with, I thought. Tony said something like "`inat͡ʃii' means `think.' You have to put one on only one of the sides.")}


\exg. kai inat͡ʃii-ɔ kaama bondu a fen saŋ kunu\\
Kai think-3SG-ɔSER concerning Bondu AUX.PST what buy yesterday\\
`What did Kai think concerning what Bondu bought yesterday?' \jf{(This is the assumed translation that I have, Tony didn't actually give a translation. The problem with this, though, is that there's no matrix "what.")}\\

\exg. bondu a fen saŋ kunu Kai ni inat͡ʃii wa ama\\
Bondu AUX.PST what buy yesterday Kai NI think WA AMA\\
`What Bondu bought yesterday is what kai was thinking about.'\\

\exg. bondu a fena-min saŋ kunu Kai ni inat͡ʃii wa ama\\
Bondu AUX.PST thing-which buy yesterday Kai NI think WA AMA\\
`Which thing Bondu bought yesterday is what kai was thinking about.' \jf{(When said slowly, the WA sounds more like ɔ)}\\


\jf{Can I get the critical structure above with either of these? TEST:}


\jf{Reconfirm, and re-prime following structure (suggests something interesting about the argument structure of fɔ, in that it doesn't directly take an object):}

\exg. kai a feɱ fɔ-je bondu a fen sa kunu?\\
Kai 3SG what say-3SG Bondu AUX.PST what buy yesterday\\
`What did Kai say that Bondu bought yesterday?' \\ 

\jf{Can fɔ take ɔ-series object agreement with CP? (Apparently not CP object agreement after all)}

\exg. *kai a feɱ ɔ fɔ-je bondu a fen sa kunu?\\
Kai 3SG what 3SG.ɔSER say-3SG Bondu 3SG what buy yesterday\\
`What did Kai say that Bondu bought yesterday?' \\ 

\exg. *kai ɔ feɱ fɔ-je bondu a fen sa kunu?\\
Kai 3SG.ɔSER what say-3SG Bondu 3SG what buy yesterday\\
`What did Kai say that Bondu bought yesterday?' \\ 

\exg. *kai a feɱ ɔ fɔ bondu a fen sa kunu?\\
Kai 3SG what 3SG.ɔSER say Bondu 3SG what buy yesterday\\
`What did Kai say that Bondu bought yesterday?' \\ 
 

\exg. ɔfɔ \\
`backbite/criticize' \\ 

\section{Jan}

\subsection{[ja] vs [jaa]}

\ex. já \\
`eye'

\exg. ń-dʒà \\
1SG.POSS-eye \\
`my eye'

\jal{good e.g. for Lex}

\exg. í jà \\
2SG.POSS-eye \\
`your eye'

\exg. ó jà \\
2PL.POSS-eye \\
`you all's eyes'

\exg. í já jàwɛ̀ɛ̀ nè \\
2SG eye red COP \\
`Your eye is red.' 

\jmt{Context. Bondu somehow found an eye. She was scared and put it on the table.}

\exg. bòndú á éjá sá tébù mà  \\
bondu AUX.PST eye put table P \\
`Bondu put the eye on the table.' 

\exg. bóndú tʃéé á jà sá tébù mà \\
bondu husband AUX.PST eye put table P \\
`Bondu's husband put the eye on the table.'

\exg. bóndú tʃéé $\varnothing$ éjá sá tébù mà \\
bondu husband 2SG.IMP eye put table P \\
`Bondu's husband, put the eye on the table!'

\exg. jɛ̀j á já sá tébù mà \\
jaj AUX.PST eye put table P \\
`Jai put the eye on the table.'

\ex. jàà \\
`(to) prank'

\exg. jɛ́j á bóndú jàà \\
Jai AUX.PST bondu prank \\
`Jai pranked Bondu.'

\jmt{Next week: Jai pranked you.}

\exg. sáá bóndú jà \\
Saa bondu prank \\
`Saa pranked Bondu.'

\exg. bóndú á sáá jàà \\
bondu AUX.PST Saa prank \\
`Bondu pranked Saa.'

\exg. bóndù tʃéé á sáá jàà \\
bondu husband AUX.PST Saa prank \\
`Bondu's husband told the joke.'

\ex. jáá \\
`peanut'

\jmt{Context. Bondu now finds a peanut, which for some reason Fea wanted to eat.}

\exg. bóndú à ejáá sá tébù mà \\
bondu AUX.PST peanut put table P \\
`Bondu put the peanut on the table.'


\subsection{[i]-syllable onset}

\jmt{Structure: }

\ex. $x$ í jéé mɛ̀m̀ɛ̀ ò kunu \\
`$x$ saw you in the mirror yesterday.'

\exg. bóndú é jẽ́ẽ́ mɛ́mɛ́ ó kùnù \\
bondu AUX.PST.2SG see mirror P yesterday \\
`Bondu saw you in the mirror yesterday.'

\jmt{anthony notes [a] + ̃[i]+ [jẽ ] here  but says [e]}

\exg. jɛ́j $\varnothing$ jẽ̀ mɛ̀mɛ̀ ó kùnù \\
jai 1SG.2SG see mirror P yesterday \\
`Jai saw you in the mirror yesterday.'

\exg. kúnú nìí jɛ́j é jẽ̀ mɛ̀mɛ̀-ò \\
yesterday {it was?} jaj 1SG.2SG see mirror-P \\
`Yesterday, Jai saw you in the mirror.'

\jmt{The `mirror' + `P' at the end was also realized as [memjo]}

\exg. kúnú níì sá í jè mɛ̀mɛ̀ ò \\
yesterday {it was} Saa 2SG see mirror P \\
`Saa saw you in the mirror yesterday.'

\exg. fá í jè mɛ̀mɛ̀ ò kùnù \\
father 2SG see mirror P yesterday \\
`Father saw you in the mirror yesterday.'

\exg. kúnú nìì fá é jè mɛ̀mɛ̀ ò \\
yesterday {it was} father 2SG see mirror P \\
`Yesterday, father saw you in the mirror.'

\exg. bóndú tʃéé í jè mɛ̀mɛ̀ ò kùnù \\ % check duration
bondu husband 2SG see mirror P yesterday \\
`Bondu's husband saw you in the mirror yesterday.'

\exg. kúnú nì bòndú tʃé í jè mɛ̀mɛ̀ ò̜̜ \\
yesterday {it was} bondu husband 2SG see mirror P \\
`Yesterday, Bondu's husband saw you in the mirror.'

\exg. í bóndjè dúú àmbè \\
2SG finger five AMBE \\
`You have five fingers.'

\exg. fén dúú àmbì gbóó \\
thing five AMBE.2SG.POSS hand \\
`I have five things.' 

\jmt{I'd like to know if this [àmbì] is different because of inalienable possession.}

\exg. dúú àmbì gbóó \\
five AMBE.2SG.POSS hand \\
`I have five.'

\jmt{Not elicited:}

\jmt{Structure:} \jal{did you talk to Jianjing about this structure? these initial DPs are going to be intonationally dislocated from the rest of the sentence, (hence the comma), so we don't expect any interaction between the DP and the verb}


\section{Mingyang}

\begin{itemize}
    \item Follow up on last time: 
    \exg. kajne kante-a. n kwa wã kajne tea kunu.\\
   boy place.of.learning-A 1.SG speak WAN boy P yesterday\\
   `There is a boy in the place of learning. I talked to \textbf{the boy} yesterday.'
%\jal{Do listen again to the recording.  I have the second one as kain dʒe = boy this, which is a crucial difference for you. He said it seven times, the second was as you have written but it was very slow, and the rest of the times were clearly kain dʒe (and he translated it twice as "this boy" and once as "that same boy").  I also have dʒe on your other examples from last time in the anaphoric section}
%\mb{Thank you! It is very likely that we can have a demonstrative there! I was hesitant to put kajn dʒɛ `this boy' because 1) I haven't found other examples with ne+t͡ʃɛ becoming ndʒɛ, like `this house' t͡ʃene t͡ʃɛ; 2) we only get kajn dʒɛ in fast speech, so I thought the difference was phonetic within the word kajne; 3) it is less surprising that demonstratives are compatible with such contexts, so I was detecting instances where the definite is clearly realized as a bare noun (and we indeed only got one case with slow speech for this sentence); 4) when I asked Anthony whether we can say dumusune sio kaine bema `the girl sits next to the boy', he said ndʒɛ and ne are both acceptable. But on the other hand, we never hear kajn dʒɛ in `there is a boy...'. More things to ask Anthony today!} \jal{we have seen final e of ne dropped, and n + t͡ʃɛ becoming ndʒɛ is exactly expected}
%\\
%\mb{About the status of `te-a': Can we say the sentence without `te-a'? Can we say the sentence with only `-a'? } \jal{see my notes on your elicitaiton from last class} \mb{Got it, thanks!}

    \exg. kajne n-fea dumusune-a ambe kante-a. n kwa wã kajne tea kunu.\\
    boy and girl-A 3.PL place.of.learning-A 1.SG speak WAN boy P yesterday\\
   `There is a boy and a girl in the place of learning. I talked to \textbf{the boy} yesterday.'

    \ex. kamwe\\
        `teacher'

    \ex. kàndè\\
        `student'
    
   \exg. kamwe kante-a. n kwa kunu kamwe tea.\\
        teacher place.of.learning-A 1.SG speak yesterday teacher P\\
   There is a teacher in the place of learning. I talked to the teacher yesterday.

   \jal{you elicited this as two separate sentences, so we don't know yet if it's a natural discourse}

   \exg. kamwe kante-a. kamwe (t͡ʃɛ) si-o gba.\\
   teacher place.of.learning-A teacher DEM sit-O back\\ 
   There is a teacher in the place of learning. The teacher sits in the back.

\jal{again, we can't tell yet if this is a natural discourse as he didn't say it together as an utterance. (He did offer pointing to the teacher as a way to use it without t͡ʃɛ )}
   
    
   \exg. kamwe n-fea kande-a ambe kante-a. n kwa kunu kamwe (t͡ʃɛ) tea.\\
   teacher and student-A 3.PL place.of.learning-A 1.SG speak yesterday teacher DEM P\\
   There is a teacher and a student in the place of learning. I talked to the teacher yesterday.

   \jal{again, he didn't say it together}

   \exg. kamwe n-fea kande-a ambe kante-a. t͡ʃe kamwe (t͡ʃɛ) si-o kande-a (t͡ʃɛ) bema.\\
   teacher and student-A 3.PL place.of.learning-A but teacher DEM sit-O student-A DEM near\\
   There is a teacher and a student in the place of learning. The teacher sits near the student.

   \jal{for this one, when he said it together, he used t͡ʃe; you asked if it was possible without t͡ʃe and he said it was a complete sentence, repeated just the second sentence without the ``but", and said then you're just saying `a teacher is sitting next to a student'}

   \jal{so far we have data to show that the demonstrative is natural for anaphoric definites; we have suggestive comments that the bare noun isn't, but it would need to be more clearly tested.}

  \item President sentences
   \exg. sandʒa mansa wuu. jaj mansa *(t͡ʃɛ) jansã da wã.\\
   this.year chief short next.year chief DEM tall DA WAN\\
   This year, the chief$_i$ is short. Next year, \textbf{the chief$_i$} will be tall.

   \jal{you've got this annotated as impossible without. I don't think that's what you mean. But notice what he said was ``you have to be able to identify him. If the future chief is not around, then you cannot use t͡ʃɛ"  Remember early on he suggested that something marked as t͡ʃɛ had to be visible.  I'd love it if you could figure t͡ʃɛ out!}
    
   cf.:
    \exg. sandʒa mansa wuu. jaj manse jansã da wã.\\
    this.year chief short next.year chief tall DA WAN\\
    This year, the chief$_i$ is short. Next year, the chief$_{i/j}$ will be tall.

    \alex{I am wondering whether the /e/ in `manse' is the non-past auxiliary we see in future/non-past clauses? not sure if we get these kinds of auxiliaries in non-verbal predication, though...}


\end{itemize}

\section{Relative Clauses---Wesley}


\wml{Testing for restrictive use of relative clauses}

\exg. Bondu g͡bòò kàándà Sahr à mĩ́ sã̀\\
{} book read {} \textsc{AUX.PST} \textsc{mi} buy\\
`Bondu is reading the book that Sahr bought.’

\exg. Bondu g͡bòò kàándà Kai à mìm-bè à-mà\\
{} book read {} \textsc{AUX.PST} \textsc{mi}-give \textsc{3sg}-to\\
`Bondu is reading the book that Kai gave to her.’

\jal{I wasn't completely sure he'd internalized the context. It would be better if you had it in what you ask him to say; e.g.  `Bondu has two books. She is reading the one that Saa bought.'  But later you got very clear comments from him reporting a restrictive interpretation }

\exg. Bondu à mùsù-tʃɛ̀ sɔ́ɱfã̀ mín-à sɛ̀ɛ̀ sã̂\\
{} \textsc{AUX.PST} woman-\textsc{det} know \textsc{mi}-\textsc{3sg} sɛɛ bought\\
`Bondu knows the woman who bought a see.’

\exg. Bondu à mùsù-tʃɛ̀ sɔ́ɱfã̀ mín-à g͡bòò sã̂\\
{} \textsc{AUX.PST} woman-\textsc{det} know \textsc{mi}-\textsc{3sg} book bought\\
`Bondu knows the woman who bought a book.’

\wml{Turning back to non-restrictive relative clauses, manipulating tense in matrix and embedded clauses}

\exg. Bondu túmú-á dè-á mím-bè nàà Baiama sìná \\
{} love-\textsc{AUX.PST} mother-\textsc{a} \textsc{mi-we} come {} tomorrow\\
`Bondu loves his mother, who is coming to Baiama tomorrow.’\hfill{(old)}

\exg. Bondu túmú-á dè-á mí nàà Baiama kúnù\\
{} love-\textsc{AUX.PST} mother-\textsc{a} \textsc{mi-we} come {} tomorrow\\
`Bondu loves his mother, who came to Baiama yesterday.’\hfill{(old)}

\exg. Bondu mí nàà kúnù Baiama ɛ̀ jɛ̀ɛ̀ wã̀ sìná \\
{} \textsc{mi} come yesterday Baiama \textsc{3sg.ser3} return \textsc{wa} tomorrow\\
`Bondu, who came to Baiama yesterday, will leave tomorrow.’

\exg. í-wá nàà kúnù Baiama é jɛ̀ɛ̀ wã̀ sìná\\
\textsc{2sg-wa} come yesterday Baiama \textsc{3sg.ser3} return \textsc{wa} tomorrow\\
`You (sg.), who came to Baiama yesterday, will leave tomorrow.’

\exg. í-wɛ́nà Bondu sɔ̃̀, é nàà wã̀ sìná\\
\textsc{2sg}-\textsc{wena} Bondu know \textsc{2sg.ser3} come \textsc{wa} tomorrow\\
`You (sg.), who know Bondu, are coming tomorrow.’\hfill{(old)}

\exg. í-wɛ́-mbè nàà sìná, *(ja) Bondu sɔ́ɱfã̀\\
\textsc{2sg-wɛ-2sg.ser2} come yesterday \textsc{2sg.ser2} {} know\\
`You (sg.), who will come tomorrow, know Bondu.’

\exg. ó-wɛ́ nàà kúnù, wé jɛ̀ɛ̀ wã̀ sínà \\
\textsc{2pl-wɛ} come yesterday \textsc{2pl.ser2} return \textsc{wa} tomorrow\\
`You (pl.), who came yesterday, will leave tomorrow.’

\exg. àɱ-fɛ́ nàà kúnù, àmbè jɛ̀ɛ̀ wã̀ sínà \\
\textsc{3pl-wɛ} come yesterday \textsc{3pl.ser2} return \textsc{wa} tomorrow\\
`They, who came yesterday, will leave tomorrow.’

\exg. ḿ̩-bwɛ́ nàà sínà\\
\textsc{1sg.poss}-friend come tomorrow\\
`My friend is coming tomorrow.’

\exg. ḿ̩-bwɛ́ mù kààmwɛ́ ànà\\
\textsc{1sg.poss}-friend \textsc{mu} teacher \textsc{ana}\\
`My friend is a teacher.’

\exg. ḿ̩-bwɛ́ mí-mù kààmwɛ́ à ɛ̀ nàà sínà\\
\textsc{1sg.poss}-friend \textsc{mi-mu} teacher \textsc{a} \textsc{3sg.ser3} come tomorrow\\
`My friend, who is a teacher, is coming tomorrow.’\hfill{(restrictive)}

\exg. ḿ̩-bwɛ́ mí-mù kààmwɛ́ à à nàà-wã̀ kúnù\\
\textsc{1sg.poss}-friend \textsc{mi-mu} teacher \textsc{a} \textsc{3sg.ser1} come-\textsc{wa} tomorrow\\
`My friend, who is a teacher, came yesterday.’

\exg. ḿ̩-bwɛ́ nàà kúnù mí-mù kààmwɛ́\\
\textsc{1sg.poss}-friend come yesterday \textsc{mi-mu} teacher\\
`My friend came yesterday who is a teacher.’

\exg. ḿ̩-bwɛ́-nù mî-mù kààmwɛ́ à(nà) *(àm)bè nàà sínà\\
\textsc{1sg.poss}-friend-\textsc{pl} \textsc{mi-mu} teacher \textsc{a} \textsc{3pl.ser3} come-\textsc{wa} tomorrow\\
`My friends, who are teachers, are coming tomorrow.’

\exg. ḿ̩-bwɛ́-nù mî-mù kààmwɛ́ à(nà) àn-náà kúnù\\
\textsc{1sg.poss}-friend-\textsc{pl} \textsc{mi-mu} teacher \textsc{a} \textsc{3pl.ser3} come-\textsc{wa} tomorrow\\
`My friends, who are teachers, came yesterday.’

\exg. ḿ̩-bwɛ́ náà kúnù mî-mù kààmwɛ́ ànà\\
\textsc{1sg.poss}-friend come yesterday \textsc{mi-mu} teacher \textsc{a-foc}\\
`My friends came yesterday who are teachers.’


\section{Chun-Hung}

\chs{\textbf{A. Predicative possessives, PP predication, NP predication} --- to argue that predicative possessives pattern with PP predication (in line with Creissels (2025) who described `hand' as an adposition) not NP predication} \newline

\chs{\textbf{Possessives}: Possessum + (wa) + possessor-hand} 

\chs{\textbf{PP predication}: Subject + location-postposition} 

\chs{\textbf{NP predication}: Subject + (mu) + possessor-noun} \newline

\chs{1. Presence/absence of MU (in the present tense) --- predicative possessives and PP predication can't co-occur with the copular \textit{mu}, while NP predication can.}

\jal{-note that `hand' is boo (I'm more confident of the onset than the vowel, but both the formants and the dissertation support /o/; dog has a long vowel}

\jal{-Is /wa/ actually /a/ with a preceding /u/? For PP predication we'd expect /ɛ̀/ or nothing, although all the PP predicates we tried were stage-level. It could also be our post-verbal w\~{a}, although it's not nasal}

\exg. W\'{u} (*m\`{u}) w\`{a} B\`{o}nd\'{u}-gb\'{\textipa{O}}\`{\textipa{O}}. \\ 
dog (*COP) WA Bondu-hand \\
`Bondu has a dog.' \chs{predicative possessives}

\exg. W\'{u} B\`{o}nd\'{u}-gb\'{\textipa{O}}\`{\textipa{O}}. \\ 
dog Bondu-hand \\
`Bondu has a dog.' \chs{predicative possessives}

\exg. W\'{u} (*m\`{u}) B\`{o}nd\'{u}-t\'{e}\`{a}. \\ 
dog (*COP) Bondu-with \\
`The dog is with Bondu.' \chs{PP predication}

\exg. W\'{u} (m\`{u}) B\`{o}nd\'{u}-\`{a}-s\'{o}f\'{e}n(\`{e})-ǎn(\`{e})-\`{a}. \\
dog (COP) Bondu-A-pet-ANE-A \\
`The dog is Bondu's pet.' \chs{NP predication}

\exg. B\`{o}nd\'{u}-\`{a}-s\'{o}f\'{e}n(\`{e})-ǎn\`{e} w\'{u}-w\`{a}. \\
Bondu-A-pet-ANE dog-A \\
`Bondu's pet is a dog.' \chs{NP predication}

\chs{The final \textit{-(w)a} morpheme seems to mark NP serving as predicates.}

\exg. W\'{u} m\'{e}nd\'{e}? \\
dog where \\
`Where is the dog?'

\chs{2. Plural subjects (in the present tense) --- predicative possessives and PP predication occur with the morpheme \textit{mbe}, while NP predication has some tonal changes on the subject.}

\exg. W\'{u} \`{m}b\'{e} w\`{a} B\`{o}nd\'{u}-gb\'{\textipa{O}}\`{\textipa{O}}. \\ 
dog PL WA Bondu-hand \\
`Bondu has dogs.' \chs{predicative possessives}

\exg. W\'{u} \`{m}b\'{e} B\`{o}nd\'{u}-t\'{e}\`{a}. \\ 
dog PL Bondu-with \\
`The dogs are with Bondu.' \chs{PP predication}

\exg. Wǔ\'{u} m\`{u} B\`{o}nd\'{u}-\`{a}-s\'{o}f\'{e}n(\`{e})-ǎn(\`{e})-\`{a}. \\
dog COP Bondu-A-pet-ANE-A \\
`The dogs are Bondu's pet.' \chs{NP predication}

\exg. *W\'{u} \`{m}b\'{e} m\`{u} B\`{o}nd\'{u}-\`{a}-s\'{o}f\'{e}n(\`{e})-ǎn(\`{e})-\`{a}. \\
dog PL COP Bondu-A-pet-ANE-A \\
`The dogs are Bondu's pet.' \chs{NP predication}

\chs{3. topicalization on possessors/locations --- no difference; all three can be topicalized}

\exg. B\`{o}nd\'{u}, w\'{u} w\'{a} \'{a}-gb\'{\textipa{O}}\`{\textipa{O}}. \\ 
Bondu dog WA 3SG.SER1-hand \\
`Bondu, she has a dog.' \chs{predicative possessives}

\exg. B\`{o}nd\'{u}, w\'{u} w\`{a} ã̀-gb\'{\textipa{O}}\`{\textipa{O}}. \\ 
Bondu dog WA 3PL.SER1-hand \\
`Bondu, they have a dog.' \chs{predicative possessives; Bondu interpreted as a part of `they'}

\exg. B\`{o}nd\'{u}, w\'{u} \`{a}-t\'{e}\`{a}. \\ 
Bondu dog 3SG.SER1-with \\
`Bondu, the dog is with her.' \chs{PP predication}

\exg. B\`{o}nd\'{u}, w\'{u} m\`{u} \`{a}-s\'{o}f\'{e}n(\`{e})-ǎn(\`{e})-\`{a}. \\
Bondu dog COP 3SG.SER2-pet-ANE-A \\
`Bondu, the dog is her pet.' \chs{NP predication}

\chs{4. \textit{wh}-formation on possessors/locations --- no difference}

\exg. W\'{u} \textipa{\textltailn}\^{o}-gb\'{\textipa{O}}\`{\textipa{O}}? \\
dog who-hand \\
`Who has a dog?' \chs{predicative possessives; the morpheme \textipa{wa} may not be used here but needs double checking}

\jal{I had: \textipa{\textltailn}\^{o}m b\'{o}\`{o}}

\exg. W\'{u} \textipa{\textltailn}\^{o}-t\'{e}\`{a}? \\
dog who-with \\
Who is the dog with? \chs{PP predication}

\exg. W\'{u} m\`{u} \textipa{\textltailn}\'{o}n-\`{a}-s\'{o}f\'{e}n(\`{e})-\`{a}? \\
dog COP who-A-pet-A \\
Whose pet is the dog? \chs{NP predication; \textit{-ane} doesn't show up}

\exg. \textipa{\textltailn}\'{o}n-\`{a}-s\'{o}f\'{e}n\`{e} m\`{u} wǔ-w\`{a}? \\
who-A-pet COP dog-A \\
Whose pet is the dog? \chs{NP predication; \textit{-ane} doesn't show up}

\chs{For the word `pet', the final \textit{-a} shows up when NP serves as nominal predication, the morpheme \textit{-ane} doesn't show up with interrogatives or in the control domain as followed.}

\exg. W\'{u} \`{m}b\'{e} \textipa{\textltailn}\^{o}-gb\'{\textipa{O}}\`{\textipa{O}}? \\
dog PL who-hand \\
`Who has dogs?'

\exg. W\`{u}-w\^{e} \textipa{\textltailn}\^{o}-gb\'{\textipa{O}}\`{\textipa{O}}? \\
dog-big who-hand \\
`Who has a big dog?' \chs{`big' has some vowel change.}

\chs{Others to try with: \textit{wh}-formation on possessums/subjects, vowel change for subjects?} \newline

\chs{\textbf{Status of grammatical subjects}}

\chs{1. Possessums can be controlled PRO?}

\ex. I have a dog. 

\ex. I will have a dog.

\exg. W\'{u}-t\textipa{S}\`{\textipa{E}} t\'{u}-m\'{u} [CP \`{a}n\`{i} m\'{a} (\'{m})f\'{a}\textipa{N}-gb\'{\textipa{O}}\`{\textipa{O}}]. \\
dog-this like-MU [ 3SG FUT MFANG-hand] \\
`The dog would like me to be his owner.' \newline
`The dog wants [to be owned by me].' \chs{don't know what \textit{mfang} is.}

\jal{it seems to be the strong form of the 1sg pronoun; on the notes from 04/09/2025 I added two sentences from last year (132, 133). Also notice that Daniel got it during this elicitation when focusing ``to ME", and Giang also did when ``I'' was a narrow focus answer to a question;  again, I had this as ɱ́f\^{a}m-b\'{o}\`{o}}

\exg. W\'{u}-t\textipa{S}\`{\textipa{E}} t\'{u}-m\'{u} [CP \`{a}n\`{i} m\'{a} n\'{a}-s\'{o}f\'{e}n\`{e}-\`{a}]. \\
dog-this like-MU [ 3SG FUT 1SG.SER2-pet-A] \\
`The dog would like be my pet.' \newline
`The dog wants [to be owned by me].'

\chs{2. Possessors cannot be controlled PRO?}

\exg. K\'{o}pw\`{e} m\'{a}-wã́ \textipa{N}-gb\'{\textipa{O}}\`{\textipa{O}} j\`{a}j.  \\
money FUT-WA 1SG.SERI-hand next.year \\
`I will have money next year.'

\exg. *\'{N}-t\'{u}-m\`{u} [CP nĩ́  k\'{o}pw\`{e} m\'{a}-wã́ \textipa{N}-gb\'{\textipa{O}}\`{\textipa{O}} j\`{a}j].  \\
1SG.SER1-like-MU [ 1SG money FUT-WA 1SG.SERI-hand next.year] \\
`I want to have money next year.' \chs{possessor targeted; cannot be controlled PRO}

\exg. Mb\'{e} k\'{o}pw\`{e} m\'{a}s\`{o}nd-a-wã́ j\`{a}j.  \\
1SG.SER3 money get-A-WA next.year \\
`I will have money next year.'

\exg. \'{N}-t\'{u}-m\`{u} nĩ́ k\'{o}pw\`{e} m\'{a}sõ̀ j\`{a}j.  \\
1SG.SER1-like-MU 1SG money get next.year \\
`I want [to have money next year].' \chs{possessor targeted}

\ex. I want [to have a son next year].

\chs{\textbf{C-commanding relations between possessors and possessums}} \newline

\chs{1. Condition C}

\ex. Bondu\textit{\scriptsize{i}} hugged her\textit{\scriptsize{i}} mother.

\ex. Her\textit{\scriptsize{i}} mother hugged Bondu\textit{\scriptsize{i}}. 

\ex. Bondu's mother hugged her.

\ex. She\textit{\scriptsize{i}} hugged Bondu\textit{\scriptsize{i}}'s mother.

\ex. Bondu\textit{\scriptsize{i}} has her\textit{\scriptsize{i}} keys.

\ex. She\textit{\scriptsize{i}} has Bondu\textit{\scriptsize{i}}'s keys. (coindexation in English is ungrammatical, but conindexation in Kono may be grammatical if the possessum is higher than the possessor)



\chs{2. Variable binding}

\ex. Every student hugged his (own) mother.

\ex. His mother hugged every student. (meaning Every student's mother hugged him.)

\ex. Every student has his (own) schoolbag.

\ex. His mother has every student (with her). (meaning Every student's mother has him with her.) 

\chs{3. Reciprocals/Reflexives (if possible)}

\ex. I have you. (in the context of `I'm not alone, and I have you with me').

\ex. The students hugged each other.

\ex. The students have each other. (in the context of `They can support each other.')

\ex. I have myself. (???) 


\section{Daniel}
\exg. kunũ fa-ni Bondu-a tombwe do\\
yesterday \textit{fa}-\textit{ni} Bondu-3SG dance \textit{v}\\
`It was yesterday that Bondu danced'
\ds{Some interesting stacking of the two focus markers here?}

\exg. Bondu-a tombwe du so d\textipa{Z}o\\
Bondu-3SG dance \textit{v} ? ?\\
`When did Bondu dance?'

\exg. Kunu-ni Bondu-a tombwe do\\
yesterday-\textit{ni} Bondu-3SG dance \textit{v}\\
`Was it yesterday that Bondu danced?'

\exg. Banũ ni Bondu-a tombwe do\\
last.year \textit{ni} Bondu-3SG dance \textit{v}\\
`Was it last year that Bondu danced?'

\exg. Ban-fã ni Bondu-a tombwe do\\
last.year-\textit{fã} \textit{ni} Bondu-3SG dance \textit{v}\\
`Was it last year that Bondu danced?'

\exg. Bìì ni/*wã Bondu-a tombwe do\\
today \textit{ni}/\textit{wã} Bondu-3SG dance \textit{v}\\
`Was it today that Bondu danced?'

\exg. Bondu-a tat\textipa{S}e-t\textipa{S}e Baiama-wã\\
Bondu-3SG walk-? Baiama-wã\\
`Bondu walked to Baiama'

\exg. Baiama-ni/*na Bondu-a tat\textipa{S}e t\textipa{S}e\\
Baimaia-\textit{ni}/\textipa{na} Bondu-3SG walk ?\\
`It was to Baiama that Bondu walked'
\ds{It looks like across the board for adjuncts \textit{ni} is some kind of focus marker, maybe a cleft? Not compatible with the argument narrow focus/cleft marker}

\exg. Bondu-a tat\textipa{S}e-t\textipa{S}e mfã téa?\\
Bondu-3SG walk-? 1SG-fã towards\\
`Was it to me that Bondu walked?'

\exg. mfã téa nì Bondu-a tat\textipa{S}e t\textipa{S}e\\
1SG-fã towards \textit{ni}/\textipa{na} Bondu-3SG walk ?\\
`It was to me that Bondu walked'

\exg. Bondu-a tat\textipa{S}e-t\textipa{S}e iwã téa\\
Bondu-3SG walk-? 2SG-wã towards\\
`Was it to you that Bondu walked?'

\exg. iwã téa nì Bondu-a tat\textipa{S}e t\textipa{S}e\\
2SG-fã towards \textit{ni}/\textipa{na} Bondu-3SG walk ?\\
`It was to you that Bondu walked'
\ds{w/f alternation adjacent to the nasal?}

\exg. Bondu-a bwìì-t\textipa{S}e taia-taia-na?\\
Bondu-3SG ran-? fast-fast\\
`Did Bondu run quickly?'

\exg. Taia-taia-ni Bondu-a bwìì-t\textipa{S}e\\
fast-fast-\textipa{ni} Bondu-3SG ran-?\\
`It was quickly that Bondu ran'

\exg. Tiá-tiá Bondu-a tombwe dõ fã?\\
true-true Bondu-3SG dance \textit{v} \textit{fã}\\
`Did Bondu really dance?'

\exg. Tiá-tiá Bondu-a tombwe dõ\\
true-true Bondu-3SG dance \textit{v}\\

\exg. A-a, Bondu níí-ma tombwe do-ni\\
no Bondu ?-NEG dance \textit{v}-?\\
`No, Bondu did not (actually) dance'

\exg. Bondu-a tombwe dõ-fã tiá-tiá?\\
Bondu-3SG dance \textit{v}-\textit{fä} true-true\\
`Did Bondu really dance?'

\exg. Bondu-a tiá-tiá tombwe dõ-fã\\
Bondu-3SG true-true dance \textit{v}-\textit{fã}\\
*`Did Bondu really dance?' (sentential reading)\\
ok `Did Bondu dance a true dance?' (verbal reading?)


\exg. Àà, tià-tià (*wã)\\
yes true-true wã\\
`Yes, he did' (lit: Yes, really)


\section{Giang}
 


\exg. ɲɔ́mbè dí-tʃa a sénà?\\
Who cry-v a tomorrow\\
Who will cry tomorrow?

\ex. Bondu

\exg. Bondu ambe di-tʃe a sena\\
Bondu ? cry-v a tomorrow\\
``Bondu will cry tomorrow.'' (Answer form)

\g{Testing for exhaustiveness:}

\exg. Bondu ambe di-tʃe a sena Saa we di-tʃe wa sena.\\
Bondu \textsc{foc} cry-v a tomorrow Saa also cry-v FOC tomorrow.\\
``It was Bondu who will cry tomorrow, and Saa will also cry.''

\exg. Bondu di-tʃa (w\'ã) sena Saa we di-tʃe wa sena\\
Bondu cry-v (\textsc{foc}) tomorrow Sa also cry-v \textsc{foc} tomorrow\\
Bondu will cry tomorrow and Saa will also cry tomorrow


\jal{great clear result on this test!}

\exg. ɲɔ́-n-à dí-t\textipa{S}\`{\textipa{E}} (*wã́) kùnù?\\
Who-\textsc{foc-aux} cry-\textit{v} ? yesterday\\
``Who cried yesterday?''

\g{Testing for exhaustiveness:}
\exg. Bondu án-à dí-t\textipa{S}\`{\textipa{E}} (*wã́) kùnù tʃe Saa we di-tʃe wa kunu\\
Bondu \textsc{FOC-AUX} cry-\textit{v} ? yesterday and Saa also cry-v \textsc{foc} yesterday.\\
``It was Bondu that cried yesterday.''

\exg. Bondu \'{a} d\'{i}-t\textipa{S}\`{\textipa{E}} (wã́) k\`{u}n\`{u} tʃe Saa we di-tʃe wa kunu\\
Bondu A cry-\textit{v} ? yesterday and Saa also cry-v \textsc{foc} yesterday\\
`Bondu cried yesterday and Saa also cried yesterday.'

\exg. Bondu ana di-tʃe kunu tʃe Saa (*ana) we (*ana) di-tʃe wa kunu\\
Bondu \textsc{foc} cry-v yesyerday and Saa (*\textit{foc}) also (*\textit{foc}) cry-v (*\textit{foc}) yesterday.\\
`Bondu cried yesterday and Saa also cried yesterday.'

\g{Person-Number paradigm for answers to "Who will cry tomorrow?"}
\exg. ɱ fambe di-tʃa sena\\
1\textsc{sg} \textsc{foc} cry-v tomorrow\\
I will cry tomorrow. 

\exg. mo ambe di-tʃa sena\\
1\textsc{pl} \textsc{foc} cry-v tomorrow\\
We will cry tomorrow (incl/excl).

\exg. i wambe di-tʃa sena\\
2\textsc{sg} \textsc{foc} cry-v tomorrow\\
You(sg) will cry tomorrow.

\exg. o ambe di-tʃa sena\\
2\textsc{pl} \textsc{foc} cry-v tomorrow\\
You(pl) will cry tomorrow.

\exg. a wambe di-tʃa sena\\
3\textsc{sg} \textsc{foc} cry-v tomorrow\\
He will cry tomorrow.

\exg. am fambe di-tʃa sena\\
3\textsc{pl} \textsc{foc} cry-v tomorrow\\
They will cry tomorrow

\g{o-series verbs: "asked" and "broke" \& object paradigm}

\exg. Bondu a Saa o tisa kunu.\\
Bondu AUX.PST Saa o ask yesterday.\\
``Bondu asked Saa yesterday''

\exg. Bondu a ɲondo tisa kunu\\
Bondu AUX.PST who-o ask yesterday\\
``Who did Bondu ask yesterday?''

\section{Alex}

\exg.
kaŋana   a            tond-a  \\
door     OBJ(.AUX?)   close-A \\%
`The door closed.'

\exg.
kaŋanɛ-n   a            tond-a  \\
door-PL    OBJ(.AUX?)   close-A \\%
`The doors closed.'

\exg.
ɔɔ        te-a    \\
3SG.OBJ   break-A \\%
`It broke.' \label{It broke}

\exg.
mbé       kaŋana   a     tond-a    waN \\
1SG.AUX   door     OBJ   close-A   FOC \\%
`I will close the door.'

\exg.
mbé       kaŋana   a     tond-a  \\
1SG.AUX   door     OBJ   close-A \\%
`I am closing the door.'

\exg.
mbé       kaŋanɛ-n   a     tond-a    waN \\
1SG.AUX   door-PL    OBJ   close-A   FOC \\%
`I will close the doors.'

\exg.
kaŋanɛ   a     tond-a    waN \\
door     OBJ   close-A   FOC \\%
`The door will close.'

\exg.
kaŋanɛ-n   ɛ     a     tond-a    waN \\
door-PL    AUX   OBJ   close-A   FOC \\%
`The doors will close.' \label{The doors will close}

\alex{In \ref{The doors will close}, it sounds like the non-past auxiliary /ɛ/ intervenes between the surface subject and what I am glossing as the object marker /a/.}

\exg.
ɔɔ              te-a      waN \\
3SG.OBJ?.AUX?   break-A   FOC \\%
`It will break.' \label{It will break}

\alex{Regarding the length of the 3SG vowel, I am not sure if there is a length contrast between \ref{It will break} and \ref{It broke}.}

\exg.
andɔɔ         te-a      waN \\
3PL.OBJ.AUX   break-A   FOC \\%
`They will break.'

\exg.
n-ɔŋɡɔ     kaŋana   a     ton-da  \\
1SG-ONGO   door     OBJ   close-A \\%
`I am closing the door.'

\exg.
n-ɔŋɡɔ     kaŋanɛ-n   a     ton-da  \\
1SG-ONGO   door-PL    OBJ   close-A \\%
`I am closing the doors.'

\exg.
an-ɔŋɡɔ    kaŋana   a     ton-da  \\
3PL-ONGO   door     OBJ   close-A \\%
`They are closing the doors.'

\exg.
an-ɔŋɡɔ    kaŋanɛ-n   a     ton-da  \\
3PL-ONGO   door-PL    OBJ   close-A \\%
`They are closing the doors.'

\exg.
mba           ton-da  \\
1SG.AUX.3SG   close-A \\%
`I am closing it.'

\exg.
mba       an    a     ton-da  \\
1SG.AUX   3PL   OBJ   close-A \\%
`I am closing them.'

\exg.
n-ɔŋɡɔ     a         ton-da  \\
1SG-ONGO   3SG.OBJ   close-A \\%
`I am closing it.' \label{I am closing it}

\alex{I don't hear a long vowel for the 3SG object in \ref{I am closing it}, thus unsure if it indeed occurs with the /a/ ``object marker''. However, it becomes a little more clear/apparent when we have a 3PL object pronoun, as in \ref{I am closing them}.}

\exg.
n-ɔŋɡɔ     an    a     ton-da  \\
1SG-ONGO   3PL   OBJ   close-A \\%
`I am closing them.' \label{I am closing them}

\exg.
an-ɔŋɡɔ    a         ton-da  \\
3PL-ONGO   3SG.OBJ   close-A \\%
`They are closing it.'

\exg.
an-ɔŋɡɔ    an    a     ton-da  \\
3PL-ONGO   3PL   OBJ   close-A \\%
`They are closing them.'

\exg.
kaŋanɛ   ɔŋɡɔ   ton-da  \\
door     ONGO   close-A \\%
`The door is closing.'

\exg.
kaŋganɛ-n   ɔŋɡɔ   ton-da  \\
door-PL     ONGO   close-A \\%
`The doors are closing.'

\exg.
a     ɛ     ton-da  \\
3SG   AUX   close-A \\%
`It is closing.'

\exg.
ɛ     ŋɡɔ    ton-da  \\
3SG   ONGO   close-A \\%
`It is closing.'

\exg.
an    ɔŋɡɔ   ton-da  \\
3PL   ONGO   close-A \\%
`They are closing.'

\exg.
n     ɔŋɡɔ   ton-da  \\
1SG   ONGO   close-A \\%
`I am closing.' (Context: door is talking)

\exg.
nɛ        ɛ     ton-da  \\
1SG.OBJ   AUX   close-A \\%
`I am closing.' \label{I am closing}

\alex{I believe that \ref{I am closing} provides evidence for the object marker `traveling' with the theme argument, as it appears to intervene between the subject and the AUX. In cases where we see nothing intervene between the homorganic nasal for 1SG and the non-past AUX (w)e, we get something like bilabial assimilation. Yet, we do not see this in \ref{I am closing}, possibly suggesting that there is intervening material blocking the local assimilation environment.}

\exg.
n     ɔŋɡɔ   sani     ɔ     te-a    \\
1SG   ONGO   bottle   OBJ   break-A \\%
`I am breaking the bottle.'

\exg.
mbɛ        anɔŋɡɔ   sani     ɔ     te-a    \\
1SG.also   ?.ONGO   bottle   OBJ   break-A \\%
`I also am breaking the bottle.' \label{I also am breaking the bottle}

\alex{Initially Tony produced \textit{mbɛ ɔŋɡɔ...}, then corrected himself with the form in \ref{I also am breaking the bottle}. I am not entirely sure about the segmentation as I have glossed it. I could be possible that the `also' morpheme is underlying somethign like /(w)eN/, with a floating nasal, which gets resyllabified on ONGO.}

\exg.
sani-n      ɔŋɡɔ   te-a    \\
bottle-PL   ONGO   break-A \\%
`The bottles are breaking.'

\exg.
taa        mbɛ    ndɔ       te-a    \\
calabash   also   PL?.OBJ   break-A \\%
`The calabashes are also breaking.' \label{The calabashes are also breaking}

\alex{I am not entirely convinced this is the correct form in \ref{The calabashes are also breaking}.}

\exg.
taa        mbɛ    anɔŋɡɔ      te-a    \\
calabash   also   3PL?.ONGO   break-A \\%
`The calabashes are also breaking.'

\exg.
n-ɔŋɡɔ     andɔ      te-a    \\
1SG-ONGO   3PL.OBJ   break-A \\%
`I am breaking them.'

\exg.
an-ɔŋɡɔ    andɔ      te-a    \\
3PL-ONGO   3PL.OBJ   break-A \\%
`They are breaking them.'

\exg.
ɛ     ŋɡɔ    te-a    \\
3SG   ONGO   break-A \\%
`It is breaking.'

\exg.
sani     ɔŋɡɔ   te-a    \\
bottle   ONGO   break-A \\%
`The bottle is breaking.'

\exg.
n-i       kaŋana   a     ton-da    kunu,        mbɛ    sa   ana  \\
1SG-AUX   door     OBJ   close-A   yesterday,   then   Sa   came \\%
`I was closing the door, then Sa came.'

\exg.
kaŋanɛ   ɛ     ton-da,    mbɛ    sa   ana  \\
door     AUX   close-A,   then   Sa   came \\%
`The door was about to close, then Sa came.'

\exg.
kaŋanɛ   a     ni    ton     te-a,   mbɛ    sa   ana  \\
door     OBJ   AUX   close   ?-A,    then   Sa   came \\%
`The door was closing, then Sa came.'

\exg.
a     ni    ton     te-a   waN,   mbɛ    sa   ana  \\
3SG   AUX   close   ?-A    FOC,   then   Sa   came \\%
`It was closing, then Sa came'

\exg.
a     ni    ton     te-a,   mbɛ    sa   ana  \\
3SG   AUX   close   ?-A,    then   Sa   came \\%
`It was closing, then Sa came'


\section{Nasal Realization - Lex}


\exg. séŋ-wá\\
stone-big\\
`big stone'

\exg. sém-bá\\
stone-big\\
`big stone'

\ex. jáwâ\\
`red'

\exg. séŋ-jáwâ\\
stone-red\\
` red stone'

\ex. séŋ-jáwâ tʃénámà\\
stone-red big\\
` big red stone'

\ex. séŋ-jáwâ wá\\
stone-red big\\
` big red stone'

\ex. kàmànɛ̀\\
`wing'

\exg. kàmànɛ̀-wá\\
 wing-big\\
 `big wing'

 \exg. kàmàn-tʃénámà\\
 wing-big\\
 `big wing'

\exg. kàmàm-bá\\
 wing-big\\
 `big wing'

\ex. wiinɛ\\
`deer'

\exg. wiin-tʃénámà\\
deer-big\\
`big deer'

\exg. wiiŋ-wá\\
deer-big\\
`big deer'

\exg. wiim-bá\\
deer-big\\
`big deer'

\ex. kènè\\
` wooden instrument'

\exg. kènè-wá\\
wooden instrument-big\\
`big kene'

\exg. kèm-bá\\
wooden instrument-big\\
`big kene'

\exg. kèn-tʃénámà\\
wooden instrument-big\\
`big kene'

%\ex.the big kene fell 

\ex. bani\\
`drum musical instrument'

\ex. baní-wá\\
baní-big\\
`big drum musical instrument'

\ex. baní-tʃénámà\\
baní-big\\
`big drum musical instrument'

\jf{banímbá not said here, wanted to question bambá}


\ex. bɛɲɛ
` bench, stool'

\exg. bɛŋ-tʃénámà\\
bench-big\\
`big bench'

\exg. bɛnɛ-wá\\
bench-big\\
`big bench'

\exg. bɛŋ-wá\\
bench-big\\
`big bench'

\exg. bɛm-bá\\
bench-big\\
`big bench'

\ex.ɲɛː\\
`fish'

\ex. ɲɛ-wá\\
fish-big\\
`big fish'

\ex. ɲɛ-tʃénámà\\
fish-big\\
`big fish'

\ex. ɲɛ-mbá\\
fish-macho\\
`macho fish'

\jf{Anthony begins distinguishing between 'mba' as a 'macho, masculine' marker here}

\ex.ŋàɲá
`gravel'

\exg.ŋàɲá-wá\\
gravel-big\\
`big gravel'

\exg.ŋàɲá-tʃénámà\\
gravel-big\\
`big gravel'

\exg.ɲám-bá\\
riches-big\\
`big riches, really rich'

\ex.ɲánɛ
`rich'

\jf{like big gravel rocks}

\ex. kámá\\
`elephant'

\ex. kámá-wá\\
elephant-big\\
`big elephant'

\ex. kámá-tʃénámà\\
elephant-big\\
`big elephant'

\ex. kámá-bá\\
elephant-macho\\
`macho elephant'

\ex. kám-bá\\
learning-big\\
`big learning'

\ex.kanɛ
`learning'

\ex. sámâ\\
`shoes'

\jf{ Regional, In Sando Kono}

\exg. sámâ-wá\\
shoes-big\\
`big shoes'

\exg. sámâ-mbá\\
shoes-macho\\
`macho shoes'

\exg. sámâ-tʃénámà\\
shoes-big\\
`big shoes'

\end{document}