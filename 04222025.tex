\documentclass{assets/fieldnotes}
\usepackage{wasysym}

\title{Kono (Sierra Leone)}
\author{LING3020/5020}
\date{University of Pennsylvania, Spring 2025\\04/22/2025 Extra Sessions}

\setcounter{secnumdepth}{4} %enable \paragraph -- for subsubsubsections

\begin{document}

\maketitle

\maketitle
\tableofcontents

\section{Mingyang}
\ex. sene-sa-mwe\\
    `farmer'

\ex. sene-t͡ʃe-mwe\\
    `farmer'

\exg. sene-sa-mwe wuu a bɔɔ. sene-sa-mwe t͡ʃɛ sajboa wuu a.\\
        farmer dog A have farmer DEM play.with dog A\\
        `A farmer has a dog. The farmer plays with the dog.'

\exg. sene-sa-mwe mbɛɛ wuu ambã bɔɔ. (ã) sene-sa-mwe t͡ʃɛ-nu ambe sajboa \#(ana) wuu a/na.\\
    farmer all dog 3.PL? have ANA farmer DEM-PL 3.PL play.with 3.PL dog A\\
    `Every farmer who has a dog plays with the dog.' (Lit. `All farmers have a dog ($\forall > \exists$). These farmers play with their dog.')

\mb{The possessive pronoun can co-occur with the exophoric use of the demonstrative.}

\ex. mansa\\
    `chief'

\ex. (Pointing to a chief)\\
    $\checked$ \textit{mansa t͡ʃɛ};\\
    $\times$ \textit{ã mansa}

\ex. (Pointing to two chiefs)\\
    $\checked$ \textit{mansa t͡ʃɛ-a n-fea mansa t͡ʃɛ-a} or \textit{mansa t͡ʃɛ ni mansa t͡ʃɛ};\\
    $\times$ \textit{ã mansa t͡ʃɛ-a n-fea ã mansa t͡ʃɛ-a}

Homogeneity (involving anti-uniqueness):
\exg. mansa (dondo-e) wuu, mansa \#(dondo-e) jansan.\\
    chief one-E short chief one-E tall\\
    A chief is short and a chief is tall.

(Expected; awaiting confirmation)
\exg. (\#ã) mansa t͡ʃɛ wuu, (\#ã) mansa t͡ʃɛ jansan.\\
    ANA chief DEM short ANA chief DEM tall\\
    (Pointing to a chief) This chief is short; (pointing to another chief) this chief is tall.

\exg. Bondu kongo min-do; \#(ana) mansa jansan.\\
    Bondu village REL-DO 3.PL chief tall\\
    `Bondu lives in a village. The chief is tall.' (Lit. `The village that Bondu lives in - their chief is tall.')

\mb{The demonstrative is not compatible with `the chief' in this case.}
    
Alternatively:
\exg. Bondu kongo min-do; \#(Bondu na kongo) mansa jansan.\\
    Bondu village REL-DO Bondu NA village chief tall\\
    `Bondu lives in a village. The chief is tall.' (Lit. `The village that Bondu lives in - the chief of Bondu's village is tall.')

\exg. Bondu kongo min-do; mwe minbe bondu-a kongo antumu \#(ana) mansa.\\
    Bondu village REL-DO people in Bondu-A village like 3.PL chief\\
    `Bondu lives in a village. People like the chief.' (Lit. `The village that Bondu lives in - the people in Bondu's village like their chief.')

Alternatively:
\exg. Bondu kongo min-do; mwe minbe no antumu \#(ana) mansa.\\
    Bondu village REL-DO people in there like 3.PL chief\\
    `Bondu lives in a village. People like the chief.' (Lit. `The village that Bondu lives in - the people there like their chief.')

\section{Chun-Hung}

\chs{\textbf{A. C-commanding relations between possessors and possessums} --- to test whether possessums are structurally higher than possessors as PP predication, or possessors (with delayed saturation) are structurally higher than possessums as proposed in Myler (2016)} \newline

\chs{1. Reflexives/Reciprocals: Possessums can bind reflexive or reciprocal possessors.}

\exg. \'{N}-d\'{o}nd\`{o} b\'{e}m\`{a}, \'{e} w\'{a} \`{m}-b\'{o}\`{o}. \\
1SG.SER1-one BEMA. 2SG.SER3 WA 1SG.SER1-hand \\
`It's not just me; I have you.' (in the context of `I'm not alone, and I have you with me').' \chs{the pronoun for the possessum shows that predicative possessives pattern with PP predication not NP predication}

\exg. \`{O}-d\'{o}nd\`{o} b\'{e}m\`{u}, mb\`{e} w\'{a} \textipa{M}f-\'{a}nd\`{i}-b\'{o}\`{o}. \\
O-one BEMU 1SG.SER3 WA 1SG-self-hand \\
`Even though it's just me, I have myself.' 

\exg. K\`{a}nd\'{i}\`{e} n\^{a} \textipa{\textltailn}ṍ-dʒ\`{e}\`{e}. \\
student 3PL.SER1 each.other-see \\
`The students saw each other.' \chs{The 3PL.SER1 is affected by phonology?}

\exg. B\`{o}nd\'{u} \`{a} \textipa{\textltailn}ṍ-dʒ\`{e}\`{e}. \\
Bondu 3SG.SER1 NO-see \\
`Who did Bondu see?' \\
*`Bondu saw each other.' \chs{Tony didn't explicitly say no, but was confused by this meaning.}

\exg. K\`{a}nd\'{i}\`{e} mb-\^{a} \textipa{\textltailn}ṍ-b\'{o}\`{o}. \\
student PL-A each.other-hand \\
The students have each other. (in the context of `They can support each other.') \chs{the part -mb may be from the plural marking for possessums as in PP predication, but I am not sure of the morpheme a-}

\ex. `Bondu's older brother has her'

\chs{2. Condition C}

\exg. B\`{o}nd\'{u}\textsubscript{i} (\`{a}) \`{a}\textsubscript{i}-d\`{e} jẽ̀ẽ̀. \\
Bondu (3SG.SER1) AUX.PST.SER1-mother see \\
`Bondu\textsubscript{i} saw her\textsubscript{i} mother'. \chs{the brackets indicate that the morpheme should be there but indistinguishable from the surrounding}

\exg. \`{A}\textsubscript{?i/k}-d\'{e} \`{a} B\`{o}nd\'{u}\textsubscript{i} jẽ̀ẽ̀. \\
3SG.SER1-mother AUX.PST.SER1 Bondu see \\
`Her\textsubscript{?i/k} mother saw Bondu\textsubscript{i}'. \chs{Tony seems to like the pronoun to refer to a third person.}

\exg. B\`{o}nd\'{u}\textsubscript{i}-d\`{e} (\`{a}) \`{a}\textsubscript{i/k} jẽ̀ẽ̀. \\
Bondu-mother (3SG.SER1) AUX.PST.SER1 see \\
`Bondu\textsubscript{i}'s mother saw her\textsubscript{i/k}.'

\exg. B\`{o}nd\'{u}\textsubscript{i}-d\`{e} \`{a} B\`{o}nd\'{u}\textsubscript{i} jẽ̀ẽ̀. \\
Bondu-mother AUX.PST.SER1 Bondu see \\
`Bondu\textsubscript{i}'s mother saw Bondu\textsubscript{i}.' 

\exg. \`{A}\textsubscript{*i/k} (a) B\`{o}nd\'{u}\textsubscript{i}-d\`{e} jẽ̀ẽ̀. \\
3SG.SER1 (AUX.PST.SER1) Bondu-mother see \\
`She\textsubscript{*i/k} saw Bondu\textsubscript{i}'s mother.' \chs{Condition C violation is present}

\exg. B\`{o}nd\'{u}\textsubscript{i} \`{a}\textsubscript{i}-k\'{o}-t\textipa{S}\`{\textipa{E}}n\`{a}m\'{a}-b\'{o}\`{o}. \\
Bondu AUX.PST.SER1-brother-big-hand \\
`Her\textsubscript{i} older brother has Bondu\textsubscript{i}.' = `Bondu\textsubscript{i}'s older brother has her\textsubscript{i}.' \chs{Unlike literal English translation, no Condition C violation is present in Kono, which suggests the possessum is higher.}

\jf{not sure I heard above in the recording}


\exg. \`{A}\textsubscript{*i/k}-k\'{i} w\`{a} B\`{o}nd\'{u}\textsubscript{i}-b\'{o}\`{o}. \\
3SG.SER2-key WA Bondu-hand \\
`Bondu\textsubscript{i} has her\textsubscript{*i/k} keys.' \chs{Tony may not like R-expressions are weakly bound, similar to transitive clause above.}

\exg. B\`{o}nd\'{u}\textsubscript{i}-\`{a}-k\'{i} (w\'{a}) \`{a}\textsubscript{i}-b\'{o}\`{o}. \\
Bondu-A-key (WA) 3SG.SER1-hand \\
`She\textsubscript{i} has Bondu\textsubscript{i}'s keys.' 
= `Bondu has her keys.' \chs{the wa- part has an emphasis like `Bondu indeed has her (own) key.'} \chs{When the wa- is added, the third singular possessor can hardly be distinguished as the whole sounds a high tone instead high + low to me.}

\exg. B\`{o}nd\'{u}\textsubscript{i}-k\'{o}-t\textipa{S}\`{\textipa{E}}n\`{\textipa{E}}m\'{e} w\'{a} \`{a}\textsubscript{i}-b\'{o}\`{o}. \\
Bondu-older.brother-big WA 3SG.SER1-hand \\
`She\textsubscript{i} has Bondu\textsubscript{i}'s older brother.' = `Bondu has an/her older brother.'

\exg. \`{A}\textsubscript{i}-k\'{i} \`{a}-w\'{a}nd\`{i}\textsubscript{i}-b\'{o}\`{o}. \\
3SG.SER2-key 3SG.SER1-self-hand \\
`He/She\textsubscript{i} has his/her\textsubscript{i} own key.' \chs{The reflexive can be weakly c-commanded by the antecedent.}

\exg. B\`{o}nd\'{u}-\`{a}\textsubscript{i}-k\'{i} \`{a}-w\'{a}nd\`{i}\textsubscript{i}-b\'{o}\`{o}. \\
Bondu-A-key 3SG.SER1-self-hand \\
`Bondu\textsubscript{i} has her\textsubscript{i} own key.'

\section{Wesley}

\wml{Relativising on possessor (sentential possession)}

\exg. tʃénè wá mwòkàmá bòò\\
house \textsc{wa} man hand\\
`The man has a house.’\hfill{(04-22-25, 33:17)}

\exg. Bondu à mwòkàmà sɔ́ɱfã̀ tʃénè mím-bòò\\
{} \textsc{AUX.PST} man know house \textsc{mi}-hand\\
`Bondu knows the man who has a house.’\hfill{(04-22-25, 33:24)}\\
\wml{Can we put \textit{wa} in there?}

\exg. Bondu à mwòkàmà sɔ́ɱfã̀ tʃénè ní mím-bòò\\
{} \textsc{AUX.PST} man know house \textsc{ni} \textsc{mi}-hand\\
`Bondu knows the man who had a house.’\hfill{(04-22-25, 33:48)}

\wml{Relativising on possessor in possessive DP}

\exg. káínè dè nàà Baiama.\\
boy mother come {}\\
`The boy’s mother is coming to Baiama.’\hfill{(04-22-25, 34:12)}

\exg. Bondu à kàìnè sɔ́ɱfã̀ mín-dè nàà Baiama.\\
{} \textsc{AUX.PST} boy know \textsc{mi}-mother come {}\\
`Bondu knows the boy whose mother is coming to Baiama.’\hfill{(04-22-25, 34:25)}

\exg. Bondu à kàìnè sɔ́ɱfã̀ mín-dè náà nàà Baiama.\\
{} \textsc{AUX.PST} boy know \textsc{mi}-mother \textsc{naa} come {}\\
`Bondu knows the boy whose mother came to Baiama.’\hfill{(04-22-25, 34:45)}

\exg. Kai à mùsù à tàá ò té\\
{} \textsc{AUX.PST} woman \textsc{3sg.poss} calabash \textsc{3sg.obj} break\\
`Kai broke the woman’s calabash.’\hfill{(04-22-25, 35:06)}

\exg. Bondu à mùsù sɔ́ɱfã̀ Kai à mín-à tàá ò té\\
{} \textsc{AUX.PST} woman know Kai \textsc{AUX.PST} \textsc{mi-3sg.poss} calabash \textsc{3sg.obj} break\\
`Bondu knows the woman whose calabash Kai broke.’\hfill{(04-22-25, 35:22)}

\exg. mùsù mín-à tàá ò té-à, Bondu à sɔ́ɱfã̀\\
woman \textsc{mi-3sg.poss} calabash \textsc{3sg.obj} break, Bondu \textsc{3sg} know\\
\wml{Prompt: `The woman whose calabash Kai broke, Bondu knows her'. This seems to have been a mistranslation---I will need to check with Tony what this sentence means, if it is grammatical. If grammatical, I believe it means, `The woman who broke the calabash, Kai knows her.'}\hfill{(04-22-25, 35:51)}

\exg. Kai à mùsù mín-à tàá ò té, Bondu à sɔ́ɱfã̀\\
{} \textsc{AUX.PST} woman\mss{k} \textsc{mi-AUX.PST} calabash \textsc{3sg.obj} break {} \textsc{3sg}\mss{k} know\\
`The woman\mss{k} whose calabash Kai broke, Bondu knows her\mss{k}.’\hfill{(04-22-25, 36:13)}

\wml{Relativising on indirect object (=complement of P)}

\exg. Bondu à sɛ̀ɛ̀ bè mwòkàmá-mà\\
{} \textsc{AUX.PST} sɛɛ give man-to\\
`Bondu gave a sɛɛ to a man.’\hfill{(04-22-25, 36:44)}\\
\wml{Prompt: `Bondu gave the man a sɛɛ.’}

\exg. Kai à mwòkàmà sɔ́ɱfã̀ Bondu à sɛ̀ɛ̀ bè mím-mà\\
{} \textsc{AUX.PST} man know {} \textsc{AUX.PST} sɛɛ give \textsc{mi}-to\\
`Kai knows the man to whom Bondu gave a sɛɛ.’\hfill{(04-22-25, 37:03)}

\exg. Bondu à sɛ̀ɛ̀ bè mwòkàmà mím-mà, Kai à sɔ́ɱfã̀\\
{} \textsc{AUX.PST} sɛɛ give man\mss{k} \textsc{mi}-to {} \textsc{AUX.PST}\mss{k} know\\
`The man\mss{k} to whom Bondu gave a sɛɛ, Kai knows him\mss{k}.’\hfill{(04-22-25, 37:13)}

\wml{Relativising on complement of P}

\exg. jàg͡básì dàà-ò\\
onion pot-in\\
`The onion is in the pot.’\hfill{(04-22-25, 37:38)}

\exg. Bondu à dàà ò té, jàg͡básì mínd-ò\\
{} \textsc{AUX.PST} pot \textsc{3sg.obj} break onion \textsc{mi}-in\\
`Bondu breaks the pot that the onion is in. ’\hfill{(04-22-25, 38:17)}

\exg. jàg͡básì dàà mínd-ò, Bondu ò té\\
onion pot \textsc{mi}-in {} \textsc{3sg.obj} break\\
`The pot\mss{k} that the onion is in, Bondu breaks it\mss{k}.’\hfill{(04-22-25, 38:36)}

\exg. kàà ɛ̀ sɛ̀nɛ́-kɔ̀\\
snake \textsc{3sg} stone-under\\
`The snake is under the stone.’\hfill{(04-22-25, 39:18)}

\exg. kàà ɛ̀ sɛ̀nò-kɔ̀\\
snake \textsc{3sg} stone-under\\
`The snake is under the stone.’\hfill{(04-22-25, 39:20)}\\
\wml{Not sure if the sɛ̀nò here is a production error or a legitimate alternation.}

\exg. Kai à sɛ̀nɛ́ ò té, kàà ɛ̀ míŋ-kɔ̀\\
{} \textsc{AUX.PST} stone \textsc{3sg.obj} break snake \textsc{3sg} \textsc{mi}-under\\
`Kai broke the stone that the snake is under.’\hfill{(04-22-25, 39:32)}

\exg. kàà ɛ̀ sɛ̀nɛ́ míŋ-kɔ̀, Kai ò té\\
snake \textsc{3sg} stone \textsc{mi}-under {} \textsc{3sg.obj} break \\
`The stone\mss{k} that the snake is under, Kai broke it\mss{k}.’\hfill{(04-22-25, 39:41)}

\exg. Kai à sɛ̀nɛ́ ò té, kàà à nì míŋ-kɔ̀\\
{} \textsc{AUX.PST} stone \textsc{3sg.obj} break snake \textsc{3sg} \textsc{ni} \textsc{mi}-under\\
`Kai broke the stone that the snake was under.’\hfill{(04-22-25, 39:57)}

\exg. Kai à sɛ̀nɛ́ ò té, kàà à nì míŋ-kɔ̀\\
{} \textsc{AUX.PST} stone \textsc{3sg.obj} break snake \textsc{3sg} \textsc{ni} \textsc{mi}-under\\
`Kai broke the stone that the snake was under.’\hfill{(04-22-25, 39:57)}

\exg. kàà à nì sɛ̀nɛ́ míŋ-kɔ̀, Kai ò té\\
snake \textsc{3sg} \textsc{ni} stone \textsc{mi}-under {} \textsc{3sg.obj} break\\
`The stone\mss{k} that the snake was under, Kai broke it\mss{k}.’\hfill{(04-22-25, 40:04)}

\wml{Checking referential indexing on RCs when fronted or extraposed}

\exg. Bondu mĩ́-jã́sã̀, Kai à sɔ́ɱfã̀\\
{} \textsc{mi}-tall {} \textsc{AUX.PST} know\\
`Bondu\mss{k}, who\mss{k} is tall, Kai knows him\mss{k}.’\hfill{(04-22-25, 40:37)}

\exg. Bondu mĩ́-jã́sã̀, à Kai sɔ́ɱfã̀\\
{} \textsc{mi}-tall \textsc{3sg} {} know\\
`Bondu\mss{k}, who\mss{k} is tall, he\mss{k} knows Kai.’\hfill{(04-22-25, 40:43)}

\exg. Bondu à Kai sɔ́ɱfã̀, mĩ́-jã́sã̀\\
{} \textsc{3sg} {} know \textsc{mi}-tall\\
`Bondu\mss{j} knows Kai\mss{k}, who\mss{k/*j} is tall.’\hfill{(04-22-25, 40:43)}

\exg. Bondu mĩ́-úú, à Kai sɔ́ɱfã̀, mĩ́-jã́sã̀\\
{} \textsc{mi}-short \textsc{3gs} {} know \textsc{mi}-tall\\
`Bondu\mss{j}, who\mss{j} is short, knows Kai\mss{k}, who\mss{k} is tall.’\hfill{(04-22-25, 41:44)}

 \exg. ? Kai mĩ́-jã́sã̀, Bondu mĩ́-úú, à sɔ́ɱfã̀\\
{} \textsc{mi}-tall {} \textsc{mi}-short \textsc{3sg} know\\
`Kai\mss{k} who is short, Bondu\mss{j} who is tall, he\mss{k/j} knows him\mss{j/k}.’\hfill{(04-22-25, 42:23)}\\
\wml{Tony said that you could say it but that it’s not clear. I’m not sure if this sentence is grammatical but ambiguous or outright ungrammatical.}


\exg.

\ag. Q: Bondu g͡bòò mín-à àŋ kààntèà?\\
{} {} book \textsc{mi-a} \textsc{foc}-read\\
`Which book is Bondu reading?\hfill{(04-22-25, 43:27)}

\bg. A1: Kai à g͡bòò mím-bè Bondu mà, Bondu à kàándà\\
{} {} \textsc{AUX.PST} book \textsc{mi}-give {} to {} \textsc{3sg} read\\
`The book that Kai gave to Bondu, Bondu is reading it.’\hfill{(04-22-25, 43:56)}

\cg. A2: Bondu g͡bòò míŋ-kààntèà, Kai ànà bè à-mà\\
{} {} book \textsc{mi}-read {} \textsc{3sg.foc} give \textsc{AUX.PST}-to\\
`The book that Bondu is reading,\\it was Kai that gave it to him.’\hfill{(04-22-25, 44:14)}

\dg. A3: Kai à g͡bòò mím-bè à-mà\\
{} {} \textsc{AUX.PST} book \textsc{mi}-give \textsc{3sg}-to\\
`The book that Kai gave him.’\hfill{(04-22-25, 44:38)}

\eg. A4: Kai *(à) mím-bè à-mà\\
{} {} \textsc{AUX.PST} \textsc{mi}-give \textsc{3sg}-to\\
`The one that Kai gave him.’\hfill{(04-22-25, 44:38)}

\end{document}