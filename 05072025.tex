\documentclass{assets/fieldnotes}

\title{Kono (Sierra Leone)}
\author{LING3020/5020}
\date{University of Pennsylvania, Spring 2025\\05/07/2025 Class Projects Week 7}

\setcounter{secnumdepth}{4} %enable \paragraph -- for subsubsubsections

\begin{document}

\maketitle

\maketitle
\tableofcontents


\section{Joey}

\subsection{fo + ami, proper long distance, no scope marking}

\exg. kàì á fɔ̀-ì-jé bòndú á fèn sàŋ kùnù.\\
Kai AUX.PST say-2SG-to Bondu AUX.PST what buy yesterday\\
`Kai told you what Bondu bought yesterday' \jf{(NOT A QUESTION - Per Tony, could be a response to a question)} \\

\exg. kàì á fɔ̀-ì-jé bòndú á fènè sàŋ kùnù.\\
Kai AUX.PST say-2SG-to Bondu AUX.PST something buy yesterday\\
`Kai told you that Bondu bought something yesterday' \jf{(Unlike with "ami" you can have both fen and fene embedded here.)} \\

\exg.  Kàì á fèn àmí bòndú á fèn sàŋ kùnù\\
Kai AUX.PST what hear Bondu AUX.PST what buy yesterday\\
`What did Kai hear that Bondu bought yesterday?' \jf{(Tony corrected me - no "m" at end of "ami" - YES QUESTION)}\\

\jf{If above is okay, and is a question, check answer:}

\exg. Kàì á àmí bòndú á sùéé-àn sã̀  kùnù\\
Kai AUX.PST hear Bondu AUX.PST meat-FOC buy yesterday\\
`Kai heard that Bondu bought meat yesterday'  \\

\exg.  Kàì á àmí bòndú á fènè sã̀  kùnù\\
Kai AUX.PST hear Bondu AUX.PST something buy yesterday\\
`Kai heard that Bondu bought something yesterday.' \\

\exg.  Kàì á àmí bòndú á fèn sã̀  kùnù\\
Kai AUX.PST hear Bondu AUX.PST what buy yesterday\\
`?Kai heard regarding what Bondu bought yesterday.' \jf{This was corrected to the above example, but seemed to maybe work on a marginal or infelicitous level.}\\


\subsection{Long-distance questions with think and ask:}

\subsubsection{Reconfirm that the following are questions:}

\exg. kàì á j-ɔ̀ tìsá bòndú á fèn sàŋ kùnù\\
Kai AUX.PST 2SG-ɔSER ask Bondu AUX.PST what buy yesterday\\
`Kai asked you what Bondu bought yesterday.' \jf{(Statement, not Q)}\\  


\exg. kàì á j-ɔ̀ tìsá káàmá bòndú á fèn sàŋ kùnù\\
Kai AUX.PST 2SG-ɔSER ask concerning Bondu AUX.PST what buy yesterday\\
`Kai asked you concerning what Bondu bought yesterday.' \\

\exg. kàì á tìsá-t͡ʃé káàmá bòndú á fèn sàŋ kùnù\\
Kai AUX.PST ask-t͡ʃe concerning Bondu AUX.PST what buy yesterday\\
`Kai asked concerning what Bondu bought yesterday.' \\

\jf{NB: In the above to examples, -tse got removed after tisa with the presence of the object-marked j-o.}

\jf{If this is a question, answer}

\subsubsection{Re-elicit long-distance for these two:}

\exg. j-á ínát͡ʃìí bòndú á fèn sàŋ kùnù \\
2SG-PST think Bondu AUX.PST what buy yesterday\\\
`What did you think that Bondu bought yesterday?' \jf{PROPER LD}\\  


\jf{answer:}

\exg. n-á n-d͡ʒínát͡ʃìí bòndú á kùè-àn sàŋ kùnù\\
1SG-AUX.PST 1SG-think bondu AUX.PST rice-FOC buy yesterday\\
`I thought through it. It was rice that Bondu bought yesterday.' \jf{(Interestingly here, we have a two sentence answer)}\\  
`From my thinking, it was rice that Bondu bought yesterday.' \jf{(This is an alternate translation offered - this translation inspired the possessive reading of the second "n")}\\ 

\exg. *j-á fèn ínát͡ʃìí bòndú á fèn sàŋ kùnù \\
2SG-PST what think Bondu AUX.PST what buy yesterday\\\
`What did you think that Bondu bought yesterday?' \jf{Scope marking not possible here}\\  

\exg. j-á ínát͡ʃìí fèn-á \\
2SG-PST think what-about\\
`What did you think about?' \\  

\exg. tìmá-mìn-ànà-mú j-á ínát͡ʃìí bòndú á sùèè sàŋ \\
time-which-ANA-COP 2SG-PST think Bondu AUX.PST meat buy\\
`When did you think that Bondu bought meat?'\\  

\jf{answer:}

\exg. bòndú á sùèè sàŋ kùnù-fã̀\\
Bondu AUX.PST meat buy yesterday-FOC\\
`It was yesteday that Bondu bought meat.'\\  

\exg. bòndú á sùèè sàŋ bánù-fã̀\\
Bondu AUX.PST meat buy last.year-FOC\\
`It was last year that Bondu bought meat.'\\  

\exg. bòndú á sùèè sàŋ dì-wã́\
Bondu AUX.PST meat buy today-FOC\\
`It was today that Bondu bought meat.'\\  

\exg. tìmá-mìn-ànà-mú j-á ínát͡ʃìí bòndú á sùèè sàŋ kùnù\\
time-which-ANA-COP 2SG-PST think Bondu AUX.PST meat buy yesterday\\
`When did you think that Bondu bought meat yesterday?' \jf{(Tony notes that in this instance, the question as to when pertains to the act of thinking, not the act of buying meat. He says it's the same for the previous quesiton as well, but that puts the answers he offered in a complicated situation, as they don't appear to be answering that question.)}\\ 

\exg. j-á ínát͡ʃìí bòndú á sùèè sàŋ tìmá-mìn-ànà-mú kùnù\\
2SG-PST think Bondu AUX.PST meat buy time-which-ANA-COP yesterday\\
`When did you think that Bondu bought meat yesterday?'  \jf{(This asks when regarding the act of buying meat)}\\  

\exg. j-á ínát͡ʃìí bòndú á sùèè sàŋ tìmá-mìn-ànà-(mù)\\
2SG-PST think Bondu AUX.PST meat buy time-which-ANA-(COP)\\
`When did you think that Bondu bought meat?' \jf{(This asks when regarding the act of buying meat)} \\  

\exg. j-á ínát͡ʃìí tìmá-mìn-ànà-(mù) bòndú á sùèè sàŋ\\
2SG-PST think time-which-ANA-(COP) Bondu AUX.PST meat buy\\
`When did you think that Bondu bought meat ?'  \jf{(This asks when regarding the act of buying meat)}\\  
 



\section{Mingyang}
\subsection{More on anaphoric ã}
\begin{itemize}
    \item Can ã be used with negation?
    \ex. `There is a teacher in the place of learning. Bondu doesn't know the teacher.'
\jf{Needs transcription}
    \item Can ã be used with a plural NP?
    \ex. `There are several teachers in the place of learning. The teachers are sitting in the back.'
\jf{Needs transcription}
    \item Can ã be used with a relative clause?\\
    Two situations - 1) Bondu has prior knowledge of the cassava leaves in the pot and 2) Bondu has no idea that there are cassava leaves in the pot.
    \ex. (Adapted from Arkoh 2011)\\
    `Bondu, bring the cassava leaves that are in the pot.'
\jf{Needs transcription}
    \item Can ã be used with epithet definites?
    \ex. (Adapted from Ahn \& Schwarz 2023)\\
    `Bondu came to my house. That idiot broke my pot.'
\jf{Needs transcription}    
\end{itemize}

\subsection{Situation-based co-variation}
   \ex. (Adapted from Jenks \& Konate 2022)\\
    \textit{Context: It is well known that chiefs are mean and grumpy people.}\\
    `In all the villages, the people don’t like the chief.'
\jf{Needs transcription}
    cf.:
    \ex. (Adapted from Jenks \& Konate 2022)\\
    \textit{Context: There is a chief, namely Bondu, who is mean to everybody; nobody from any village likes her.}\\
    `In all the villages, the people don’t like the chief.'
\jf{Needs transcription & addition of tested phrases}
\section{Chun-Hung}

\chs{\textbf{A. Variable binding}}

\exg. K\`{a}nd\'{i}\`{e} \`{m}-m\'{a} B\`{o}nd\'{u} jẽ́ẽ́-n\'{i}. \\
student M-NEG Bondu see-NEG \\
`Students did not see Bondu.'

\exg. K\`{a}nd\'{i}\`{e} k\`{a}ṍkã̀  m\'{a} B\`{o}nd\'{u} jẽ́ẽ́-n\'{i}. \\
student all.category NEG Bondu see-NEG \\
`All categories of students did not see Bondu.' \chs{The word k\`{a}ṍkã̀  means `all categories', such as from the high school, college ... in this case}

\exg. B\`{o}nd\'{u} \`{a} k\`{a}nd\'{i}\`{e}-gb\'{e} jẽ́ẽ́. \\
Bondu AUX.PST student-all see \\
`Bondu saw all students.'

\exg. B\`{o}nd\'{u} \`{a} k\`{a}nd\'{i}\`{e} k\`{a}ṍkã̀  jẽ̀ẽ̀-fã̀. \\
Bondu AUX.PST student all.category see-FAN \\
`Bondu saw all categories of students.'


\chs{It appears that no clear variable binding is found, as the negation comes from the clausal negator and the pronoun is plural, which may be confused with coreference.} \newline

\exg. K\`{a}nd\'{i}\`{e} (k\`{a}ṍkã̀)  m\'{a} d\`{e}-jẽ́ẽ́-n\'{i}. \\
student (all.category) NEG mother-see-NEG \\
`No student\textit{\scriptsize{i}} saw (his\textit{\scriptsize{i}}) mother.' \chs{I am not so sure about this sentence, because Tony seemed to take  k\`{a}ṍkã̀  away when he was explaining the difference with the plural one.}

\exg. K\`{a}nd\'{i}\`{e} k\`{a}ṍkã̀ m(a) -ã̀n-d\`{e} dʒẽ́ẽ́-n\'{i}. \\
student all.category NEG -3PL.SER1-mother see-NEG \\
`All categories of students\textit{\scriptsize{i}} did not see their\textit{\scriptsize{i}} mother.' \chs{Maybe this is not a good example for variable binding as the negation is the clausal negator instead of the negative quantifier, and thus it has some chance of being interpreted as co-reference?} 

\exg. \`{A}-d\`{e} m\'{a} k\`{a}nd\'{i}\`{e} k\`{a}ṍkã̀  dʒẽ́ẽ́-n\'{i}. \\
3SG.SER1-mother NEG student all.category see-NEG \\
`His\textit{\scriptsize{k}} mother saw no student\textit{\scriptsize{i}}.'

\exg. Ã̀-d\^{e} m\'{a} k\`{a}nd\'{i}\`{e} k\`{a}ṍkã̀  dʒẽ́ẽ́-n\'{i}. \\
3PL.SER1-mother NEG student all.category see-NEG \\
`Their\textit{\scriptsize{i}} mothers saw no student\textit{\scriptsize{i}}.'

\chs{\textbf{B. Some checking}} \newline

\chs{1. reflexive baseline --- may be logophors}

\exg. B\`{o}nd\'{u} \`{a}-w\`{a}nd\`{i} jẽ́ẽ́ m\`{e}m\`{e}-\`{o}. \\
Bondu AUX.PST-self see mirror-in \\
`Bondu saw himself in the mirror.'

\exg. *B\`{o}nd\'{u} \`{a} \textipa{M}f-\'{a}nd\`{i} jẽ́ẽ́ m\`{e}m\`{e}-\`{o}. \\
Bondu AUX.PST 1SG.SER1-self see mirror-in \\
`Bondu saw myself in the mirror.'  

\exg. \`{A}-w\`{a}nd\`{i}\textsubscript{*i/k} \'{a} B\`{o}nd\'{u}\textsubscript{i} jẽ́ẽ́ m\`{e}m\`{e}-\`{o}. \\
3SG.self 3SG.PST Bondu see mirror-in \\
`He\textsubscript{*i/k} saw Bondu\textsubscript{i} in the mirror.' \chs{The morpheme can survive the unbound position with a reference to a third person, which appears not like reflexives.}

\chs{2. reciprocal baseline}

\exg. K\`{a}nd\'{i}\`{e}-n- \^{a}-\textipa{\textltailn}ṍ-dʒ\`{e}\`{e}. \\
student-PL 3PL.SER1-each.other-see \\
`The students saw each other.'

\exg. *K\`{a}nd\'{i}\`{e}-n\`{u} ã̌-\textipa{\textltailn}ṍ-dʒ\`{e}\`{e}. \\
student-PL 3PL.SER1-each.other-see \\
`The students saw each other.' 

\exg. *B\`{o}nd\'{u} ã̀-\textipa{\textltailn}ṍ-dʒ\`{e}\`{e}. \\
Bondu 3PL.SER1-each.other-see \\
`Bondu saw each other.' 

\exg. B\`{o}nd\'{u}-n- \^{a}-\textipa{\textltailn}ṍ-dʒ\`{e}\`{e} (m\`{e}m\`{e}-\`{o}). \\
Bondu-PL- 3PL.SER1-each.other-see (mirror-in)\\
`Bondu and somebody else saw each other (in the mirror).' \chs{the plural marking seems like associatives}


\exg. *ã̀-\textipa{\textltailn}ṍ  ã̀ k\`{a}nd\'{i}\`{e}-n dʒ\`{e}\`{e}. \\
3PL.SER1-each.other 3PL.SER2 student-PL see \\
`The students saw each other.' 

\exg. K\`{a}nd\'{i}\`{e} mb\'{e} w\'{a} ã̀-\textipa{\textltailn}ṍ-b\'{o}\`{o}. \\
student PL WA 3PL.SER1-each.other-hand \\
The students have each other. (in the context of `They can support each other.')

\exg. *ã̀-\textipa{\textltailn}ṍ mb\'{e} w\'{a} k\`{a}nd\'{i}\`{e}-b\'{o}\`{o}. \\
3PL.SER1-each.other PL WA student-hand \\
`The students have each other.'

\exg. \textipa{\textltailn}ṍn- \`{a} k\`{a}nd\'{i}\`{e} jẽ́ẽ́ . \\
who 3SG.SER2 student see \\
`Who saw the student?'

\chs{3. Condition C baseline}

\exg. \`{A}\textsubscript{*i/k}-d\'{e} \'{a} B\`{o}nd\'{u}\textsubscript{i} jẽ̀ẽ̀. \\
3SG.SER1-mother 3SG.SER2 Bondu see \\
`Her\textsubscript{*i/k} mother saw Bondu\textsubscript{i}'. \chs{Tony doesn't like R-expressions to be weakly bound.}

\exg. B\`{o}nd\'{u}\textsubscript{i}-d\'{e} \'{a}\textsubscript{i/k} jẽ̀ẽ̀. \\
Bondu-mother 3SG.SER1 see \\
`Bondu\textsubscript{i}'s mother saw her\textsubscript{i/k}'.

\section{Alex}

\exg.
ɔ̀             té-à    \\
3SG.OBJ.PST   break-A \\%
`It broke.'


\exg.
ɔ́ɔ́             tè-á      wàN \\
3SG.OBJ.NPST   break-A   FOC \\%
`It will break.'

\exg.
ɔ̀         ní    tè      tè-á      wàN,   mbè    bòndù   nà   \\
3SG.OBJ   AUX   break   break-A   FOC,   then   Bondu   came \\%
`It was breaking, then Bondu came.'

\exg.
ɔ̀         ní    tè-à,      mbè    bondu   na   \\
3SG.OBJ   AUX   break-A,   then   Bondu   came \\%
`It was breaking, then Bondu came.'

\exg.
àn    dɔ̀        té-à    \\
3PL   OBJ.PST   break-A \\%
`They broke.'

\exg.
àn    dɔ́ɔ́        tè-à      wáN \\
3PL   OBJ.NPST   break-A   FOC \\%
`They will break.'

\exg.
àn    dɔ̀    ní    tè      tè-à      wáN,   mbè    bòndù   nà   \\
3PL   OBJ   AUX   break   break-A   FOC,   then   Bondu   came \\%
`They were breaking, then Bondu came.'

\exg.
an    dɔ    ni    te-a      waN,   mbe    bondu   na   \\
3PL   OBJ   AUX   break-A   FOC,   then   Bondu   came \\%
`They were breaking, then Bondu came.'

\exg.
an    dɔ    ni    te-a,      mbe    bondu   na   \\
3PL   OBJ   AUX   break-A,   then   Bondu   came \\%
`They were breaking, then Bondu came.'

\exg.
à             tón-dà  \\
3SG.OBJ.PST   close-A \\%
`It closed.'

\exg.
àá             ton-da    waN \\
3SG.OBJ.NPST   close-A   FOC \\%
`It will close.'

\exg.
a         ni    ton-da,    mbe    bondu   na   \\
3SG.OBJ   AUX   close-A,   then   Bondu   came \\%
`It was closing, then Bondu came.'

\exg.
an    a         ton-da  \\
3PL   OBJ.PST   close-A \\%
`They closed.'

\exg.
an    ɛɛ         ton-da    waN \\
3PL   OBJ.NPST   close-A   FOC \\%
`They will close.' \label{26890}

\alex{\ref{26890} is good evidence for an intervening morpheme between the subject pronoun and the non-past auxiliary. If there was nothing intervening, we would expect [3PL-NPST] = [an-(w)e] = [ambe]. Instead, we get [an-ɛɛ].}

\exg.
an    ni    ton-da,    mbe    bondu   na   \\
3PL   AUX   close-A,   then   bondu   came \\%
`They were closing, then Bondu came.' \label{17800}

\alex{\ref{17800} is interesting, since there does not seem to be any overt intervener between the subject pronoun and the `ni' auxiliary. Not sure what to say about this, especially in light of my observation/comment for \ref{26890}.}

\exg.
a             kune-a \\
3SG.OBJ.PST   melt-A \\%
`It melted.'

\exg.
an    a         kune-a \\
3PL   OBJ.PST   melt-A \\%
`They melted.'

\exg.
*an   dɔ   kune-a \\
{}    {}   {}     \\%
Int. `They melted.' \label{42408}

\alex{\ref{42408} helps demonstrate that the ɔ object marker is highly dependent on the lexical verb. Possibly more evidence for a ``quirky case'' kind of analysis?}

\exg.
aa             kune-a   waN \\
3SG.OBJ.NPST   melt-A   FOC \\%
`It will melt.'

\exg.
anɛɛ           kune-a   waN \\
3PL.OBJ.NPST   melt-A   FOC \\%
`They will melt.'

\exg.
a     ni    kune-a,   mbe    bondu   na   \\
3SG   AUX   melt-A,   then   Bondu   came \\%
`It was melting, then Bondu came.'

\exg.
an    a     ni    kune-a,   mbe    bondu   na   \\
3PL   OBJ   AUX   melt-A,   then   Bondu   came \\%
`They were melting, then Bondu came.' \label{31232}

\alex{\ref{31232} is exactly the kind of example I've been looking for, where the a- object marker clearly occurs before the auxiliary. This example presents an interesting contrast with \ref{17800}, where we don't see the intervening object marker. Not sure what to make of this variability yet. Nonetheless, \ref{31232} is good evidence for something.}




\section{Giang}

\g{This is a question that we have elicited in a previous session:}

\exg. B\textipa{\`ond\'u} \textipa{\textltailn\'ond\`o} \textipa{t\'is\'a} \textipa{\`a} \textipa{s\`en\`a}?\\
Bondu who ask \textsc{fut?} tomorrow?\\
``Who will Bondu ask tomorrow?''

\g{The following answer I've tried to elicit but Tony could only give me one answer (and it was unexpected), so I wanna try again:}
\ex. Bondu will ask them tomorrow.\\
\g{For this answer, please get a form with wa, a form without, and a form with some kind of ando. }

\g{This is a question that we have elicited in a previous session:}
\exg. B\textipa{\`ond\'u} \`a \textipa{f\'{\~e}}-\textipa{n\`a} \textipa{t\'{\~o}} \textipa{k\`un\`u}?\\
Bondu \textsc{aux} what-\textsc{foc} close yesterday\\
``What did Bondu close yesterday?''

\g{In answer to this question, how would you answer:...}
\g{For each answer, please get a form with wa,a form without wa, and preferable a form with some ando}

\g{Plural, dems}
\ex. Bondu closed the doors yesterday.
\jf{ needs transcription}

\ex. Bondu closed them yesterday.
\jf{Needs transcription }

\ex. Bondu closed these doors yesterday.
\jf{ needs transcription}

\ex. It was doors that Bondu closed yesterday. 
\jf{Needs transcription}
\g{This is a question that we elicited from a previous session}

\exg. B\textipa{\`ond\'u} a \textipa{f\'e\~{\'e}} \textipa{d\`ot\'e} \textipa{k\`un\`u}\\
Bondu \textsc{\textsc{3sg}} what break yesterday\\
What did Bondu break yesterday?

\g{In answer to the previous question, how would you answer....\\
For each one, please try to get an answer with the focus form ando, one with wa and one with neither.}

\ex. Bondu broke the sees yesterday.
\jf{ Needs transcription}

\exg. Bondu a séé ... phrase *phrase that could work

\ex. Bondu broke these sees.
\jf{ needs transcription}
\ex. Bondu broke them yesterday.
\jf{Needs transcription}

\jf{It was these that broke yesterday}

\ex. cassava leaf, and the meat that Kai bought yesterday, Kai cooked 
{needs transcription}

\section{Wesley}


\ex. Kai asked me, "who cooked the meat?"
{ Needs transcription, all below}
\g{ the above, is just stating a fact in general time}

\ex. Kai asked me who cooked the meat that Bondu bought

\exg. proposes a potential phrase for given answer 

\ex. Kai asked me whether Jai cooked the meat 

\ex. Kai asked me whether Jai cooked the meat that Bondu bought

\exg. proposes and clarifies aspects of past sentences

\ex. Kai bought a book for himself 

\ex. Kai is reading the book that he bought for himself 

\ex. The book that kai bought for himself, Kai is reading it

\exg. bí \\
`today'

\ex. Bondu cooked the meat that Kai bought and the cassava leaves 

\ex. Kai bought the meat and the cassava leaves 

\ex. The meat that kai bought, and the cassava leaves, kai cooked

\section{Daniel}
\ds{Past tense copula}
\exg. Bondu nì bwíí be-mᴡ\textipa{\'E\'E}nà / doktor\\
Bondu COP.PST medicine give-person {} doctor\\
`Bondu was a doctor'

\exg. Bondu mù bwíí be-mᴡ\textipa{\'E\'E}nà / doktor\\
Bondu COP.PRS medicine give-person {} doctor\\
`Bondu is a doctor'

\exg. úú t\textipa{S}e nì kowã\\
dog this COP.PST big\\
`this dog is big'

\exg. Bondu-à sw\textipa{\`E\`E} bè Kai-má kùnù\\
Bondu-AUX.PST meat give Kai-to yesterday\\
`Bondu gave the meat to Kai yesterday'

\exg. Bondu-àN-a sw\textipa{\`E\`E} bè Kai-má kùnù\\
Bondu-FOC-AUX.PST meat give Kai-to yesterday\\
`It was Bondu that gave the meat to Kai yesterday'

\exg. *Bondu nì sw\textipa{\`E\`E} bè Kai-má kùnù\\
Bondu COP.PST meat give Kai-to yesterday\\
`It was Bondu that gave the meat to Kai yesterday'

\exg. Bondu nì mìná sw\textipa{\`E\`E} bè Kai-má kùnù\\
Bondu COP.PST REL.SBJ meat give Kai-to yesterday\\
`It was Bondu that gave the meat to Kai yesterday'(?)

\exg. Kai nì Bondu-à sw\textipa{\`E\`E} bè-à-mà kùnù\\
Kai COP.PST Bondu-3SG meat give-AUX-PST to yesterday\\
`It was to Kai that Bondu gave the meat yesterday'

\exg. Kai mù\\
Kai COP.PRS\\
`It is to Kai (that Bondu gave the meat yesterday')

\exg. *Kai mù Bondu-à sw\textipa{\`E\`E} bè-à-mà kùnù\\
Kai COP.PRS Bondu-3SG meat give-AUX-PST-to yesterday\\
`It is to Kai that Bondu gave the meat yesterday'

\exg. Bondu-à sw\textipa{\`E\`E}-aN bè Kai-má kunu\\
Bondu-AUX-PST meat-FOC give Kai-to yesterday\\
`It was meat that Bondu gave to Kai yesterday'

\exg. sw\textipa{\`E\`E} nì Bondu-à bè Kai-má kunu\\
meat COP.PST Bondu-AUX-PST give Kai-to yesterday\\
`It was meat that Bondu gave to Kai yesterday'

\exg. sw\textipa{\`E\`E} mù Bondu-à bè Kai-má kunu\\
meat COP.PRS Bondu-AUX.PST give Kai-to yesterday\\
`It was meat that Bondu gave to Kai yesterday'

\exg. kùnù nì Bondu-à sw\textipa{\`E\`E} bè Kai-má\\
yesterday COP.PST Bondu-AUX.PST meat give Kai-to\\
`It was yesterday that Bondu gave meat to Kai'

\exg. Bondu-à \textipa{f\~{\'{e}}} wái t\textipa{S}è kùnù\\
Bondu what work do yesterday\\
`What (work) did Bondu do yesterday?'

\exg. Bondu-a sw\textipa{\`E\`E}-àN bè Kai-má\\
Bondu-3SG meat-FOC give Kai-to\\
`It was giving meat to Kai that Bondu did yesterday'

\exg. sw\textipa{\`E\`E} nì Bondu-à bè Kai-má\\
meat COP.PST Bondu-3SG give Kai-to\\
`It was giving meat to Kai that Bondu did yesterday' (?)

\exg. Bondu-*(aN-bè) sw\textipa{\`E\`E} bè-á (wã) Kai-mà sìná\\
Bondu-FOC-? meat give-FUT \textit{wã} Kai-to tomorrow\\
`It will be Bondu that will give meat to Kai tomorrow'

\exg. Bondu sw\textipa{\`E\`E} bé-à Kai-àà(N)-má sìná\\
Bondu meat give-FUT Kai-FOC-to tomorrow\\
`It is to Kai that Bondu will give meat tomorrow'

\exg. *Kai-aN-be Bondu-a sw\textipa{\`E\`E} bé-à sìná\\
Kai-FOC-? Bondu-3SG meat give-FUT tomorrow\\
Intended: `It is to Kai that Bondu will give meat tomorrow'

\exg. *Kai mù Bondu-a sw\textipa{\`E\`E} bé-à sìná\\
Kai COP.PRS Bondu-3SG meat give-FUT tomorrow\\
Intended: `It is to Kai that Bondu will give meat tomorrow'

\exg. Bondu sw\textipa{\`E\`E}-aN bé-à Kai-mà siná\\
Bondu meat-FOC give-FUT Kai-to tomorrow\\
`It is meat that Bondu will give to Kai tomorrow'

\exg. sina mù Bondu sw\textipa{\`E\`E}-aN bé-à Kai-mà\\
tomorrow COP.PRS Bondu meat-FOC give-FUT Kai-to\\
`It is tomorrow that Bondu will give meat to Kai'

\exg. *sina-aN-be Bondu sw\textipa{\`E\`E}-aN bé-à Kai-mà\\
tomorrow-FOC-? Bondu meat-FOC give-FUT Kai-to\\
`It is tomorrow that Bondu will give meat to Kai'

\exg. Bondu sw\textipa{\`E\`E}-aN bé-à Kai-mà siná\\
Bondu meat-FOC give-FUT Kai-to tomorrow\\
`It is giving meat to Kai that Bondu will do tomorrow'

\exg. Bondu-àN-à \textipa{s\`E\`E} gbásì kùnù. Kai-wéé-à \textipa{s\`E\`E} gbasi wã kùnù.\\
Bondu-FOC-AUX.PST \textipa{s\`E\`E} play yesterday Kai-also-AUX.PST \textipa{s\`E\`E} play \textit{wã} tomorrow\\
`Bondu played the \textipa{s\`E\`E} yesterday. Kai also played the \textipa{s\`E\`E} yesterday.'

\exg. *Bondu-àN-à \textipa{s\`E\`E} gbásì kùnù. Kai-àN-à \textipa{s\`E\`E} gbasi wã kùnù.\\
Bondu-FOC-AUX.PST \textipa{s\`E\`E} play tomorrow Kai-FOC-AUX.PST \textipa{s\`E\`E} play \textit{wã} tomorrow\\
`Bondu played the \textipa{s\`E\`E} yesterday. Kai also played the \textipa{s\`E\`E} yesterday.'

\exg. Àà, kùnù nì Bondu \textipa{s\`E\`E} gbásì.\\
yes yesterday COP.PST Bondu \textipa{s\`E\`E} play\\
`Yes, it was yesterday that Kai played the \textipa{s\`E\`E}'

\exg. kùn\textipa{\~{\`u}}go nì Bondu-à textipa{s\`E\`E} gbásì. kùnù-*(mbéé) Bondu-à s\textipa{\`E\`E} gbasi wã.\\
three.days.ago COP.PST Bondu-AUX.PST \textipa{s\`E\`E} play yesterday-also Bondu-AUX.PST \textipa{s\`E\`E} play \textit{wã}\\
`It was three days ago that Bondu played the \textipa{s\`E\`E}. But it was \textit{also} yesterday that Bondu played the \textipa{s\`E\`E}.'

\end{document}