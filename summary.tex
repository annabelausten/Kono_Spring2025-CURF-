% \documentclass[12pt]{article}
\documentclass[12pt]{assets/fieldnotes}
\usepackage[utf8]{inputenc}
\usepackage[margin=1in]{geometry}

\usepackage{fontspec}
    \setmainfont{Charis SIL}

\usepackage{enumerate}
\usepackage{textcomp}
\usepackage{tipa, vowel}
\usepackage{amssymb}
\usepackage{booktabs}


\usepackage[dvipsnames]{xcolor}
\usepackage{leipzig}

\newcommand{\lp}[1]{\textcolor{blue}{#1}}
\newcommand{\g}[1]{\textcolor{JungleGreen}{#1}}
\newcommand{\jal}[1]{\textcolor{Fuchsia}{#1}}
\newcommand{\jk}[1]{\textcolor{red}{#1}}
\newcommand{\csy}[1]{\textcolor{PineGreen}{#1}}
\newcommand{\gm}[1]{\textcolor{brown}{#1}}
\newcommand{\ah}[1]{\textcolor{red!65!black}{#1}}
\newcommand{\x}[1]{\textcolor{black!35!white}{#1}}
\newcommand{\mb}[1]{\textcolor{Dandelion}{#1}}
\newcommand{\jmt}[1]{\textcolor{Gray}{#1}}

\newcommand{\ts}{\textsc}

% For subscripts and superscripts 
\newcommand{\spspt}[1]{\ensuremath{^{\textrm{\scriptsize#1}}}}
\newcommand{\sbspt}[1]{\ensuremath{_{\textrm{\scriptsize#1}}}}

\usepackage{setspace}
\usepackage{hyperref}
\setcounter{tocdepth}{1}

\usepackage{linguex}
\usepackage{typgloss}

\title{Kono (Sierra Leone) Summary}
\author{LING3020/5020}
\date{University of Pennsylvania, Spring 2025}

\setcounter{secnumdepth}{4} %enable \paragraph -- for subsubsubsections

\begin{document}

\maketitle

\section{Phoneme Inventory}

\subsection{Vowels}
\begin{center}
{\Large
\begin{vowel}
 \putcvowel[l]{i}{1}
 \putcvowel[l]{e}{2}
 \putcvowel[l]{ɛ}{3}
 \putcvowel[l]{a}{4}
 \putcvowel[r]{ɔ}{6}
 \putcvowel[r]{o}{7}
 \putcvowel[r]{u}{8}
\end{vowel}
}
\end{center}

Vowels differ in length and tones

\begin{table}[htb!]
    \centering
    \begin{tabular}{c|c}
        tʃé & call \\
        tʃɛ̀ & this \\
        tʃɛ́ & pour \\
        tʃii & egg \\
        tʃéè & whopping cough \\
        tʃɛ́ɛ̀ & war \\
        tʃɛ́ɛ́ & husband \\
    \end{tabular}
    \caption{Minimal pairs for front vowels}
    \label{tab:my_label}
\end{table}

\begin{table}[]
    \centering
    \begin{tabular}{c|c}
        tá & go \\
        tâ & fire \\
        tàà & crawl \\
        tàâ & calabash \\
    \end{tabular}
    \caption{Minimal pairs for a}
    \label{tab:my_label}
\end{table}
    
\newpage 
\section{Pronouns/Agreement}

NOTE: don't correct the tones/vowels on any data yet; this is just for glossing

\begin{table}[htb!]
    \centering
    \begin{tabular}{ccccc}
        \toprule
        & Singular & \multicolumn{3}{c}{Plural} \\
        \toprule 
        & & 1, 2 & 1, 2, 3 & 1, 3 \\  
        \cmidrule(r){3-3}
        \cmidrule(r){4-4} 
        \cmidrule(r){5-5}
        1 & Ń & mɔ̀ & mɔ́ & Ǹ \\
        & 1SG & 1INCL.DU & 1INCL & 1EXCL\\
        \midrule
        2 & í & \multicolumn{3}{c}{wó} \\
         & 2SG & \multicolumn{3}{c}{2PL} \\
        \midrule
        3 & à & \multicolumn{3}{c}{\`{\~{a}} / \`{a}-N}\\
          & 3SG & \multicolumn{3}{c}{3PL / 3-PL} \\
        \bottomrule
    \end{tabular}
    \caption{just the pronouns}
    \label{tab:fut}
\end{table}

\begin{table}[htb!]
    \centering
    \begin{tabular}{ccccc}
        \toprule
        & Singular & \multicolumn{3}{c}{Plural} \\
        \toprule 
        & & 1, 2 & 1, 2, 3 & 1, 3 \\  
        \cmidrule(r){3-3}
        \cmidrule(r){4-4} 
        \cmidrule(r){5-5}
        1 & n-á & mɔ̀-á & mɔ́-á & ná-à \\
        & 1SG.PST & 1INCL.DU-PST & 1INCL-PST & 1EXCL-PST\\
        \midrule
        2 & j-á & \multicolumn{3}{c}{w-á} \\
          & 2SG-PST & \multicolumn{3}{c}{2PL-PST} \\
        \midrule
        3 & à & \multicolumn{3}{c}{à-n-á}\\
        & 3SG.PST / PST & \multicolumn{3}{c}{3-PL-PST} \\
        \bottomrule
    \end{tabular}
    \caption{pronouns with past auxiliary}
    \label{tab:fut}
\end{table}
\begin{table}[htb!]
    \centering
    \begin{tabular}{ccccc}
        \toprule
        & Singular & \multicolumn{3}{c}{Plural} \\
        \toprule 
        & & 1, 2 & 1, 2, 3 & 1, 3 \\  
        \cmidrule(r){3-3}
        \cmidrule(r){4-4} 
        \cmidrule(r){5-5}
        1 & n-á & mɔ̀-á & mɔ́-á & ná-à \\
        & 1SG-POSS & 1INCL.DU-POSS & 1INCL-POSS & 1EXCL-POSS\\
        \midrule
        2 & j-á & \multicolumn{3}{c}{w-á} \\
          & 2SG-POSS & \multicolumn{3}{c}{2PL-POSS} \\
        \midrule
        3 & à & \multicolumn{3}{c}{à-n-á}\\
        & 3SG.POSS & \multicolumn{3}{c}{3-PL-POSS} \\
        \bottomrule
    \end{tabular}
    \caption{Alienable possession}
    \label{tab:fut}
\end{table}

\begin{table}[htb!]
    \centering
    \begin{tabular}{ccccc}
        \toprule
        & Singular & \multicolumn{3}{c}{Plural} \\
        \toprule 
        & & 1, 2 & 1, 2, 3 & 1, 3 \\  
        \cmidrule(r){3-3}
        \cmidrule(r){4-4} 
        \cmidrule(r){5-5}
        1 & ḿ-bé & mw-ɛ̃̀ & mw-ɛ̃̂ & m-bè \\
        & 1SG-NPST & 1INCL.DU-NPST & 1INCL-NPST & 1EXCL-NPST\\
        \midrule
        2 & é & \multicolumn{3}{c}{w-é} \\
        & 2SG.NPST & \multicolumn{3}{c}{2PL-NPST} \\
        \midrule
        3 & ɛ̀ & \multicolumn{3}{c}{à-m-bè}\\
         & 3SG.NPST & \multicolumn{3}{c}{3-PL-NPST} \\
        \bottomrule
    \end{tabular}
    \caption{Pronouns with nonpast auxiliary \jal{Jianjing - what's the vowel for 3sg with DP subject? You were going to check these, I think the 1incl should have a vowel instead of the w?}}
    \label{tab:fut}
\end{table}

\begin{table}[htb!]
    \centering
    \begin{tabular}{ccccc}
        \toprule
        & Singular & \multicolumn{3}{c}{Plural} \\
        \toprule 
        & & 1, 2 & 1, 2, 3 & 1, 3 \\  
        \cmidrule(r){3-3}
        \cmidrule(r){4-4} 
        \cmidrule(r){5-5}
        1 & Ń-dó & mɔ́ & mɔ̀ & Ǹ-dó \\
        & 1SG-OBJ & 1INCL.DU.OBJ & 1INCL.OBJ & 1EXCL-OBJ\\
        \midrule
        2 & j-ɔ́ & \multicolumn{3}{c}{wó} \\
         & 2SG-OBJ & \multicolumn{3}{c}{2PL.OBJ} \\
        \midrule
        3 & ɔ́ & \multicolumn{3}{c}{à-n-dò}\\
        & 3SG.OBJ / OBJ & \multicolumn{3}{c}{3-PL-OBJ} \\
        \bottomrule
    \end{tabular}
    \caption{Pronouns plus object markers, not involving the subject}
    \label{tab:fut}
\end{table} 

\begin{table}[htb!]
\centering
    \begin{tabular}{cc}
    \toprule
    a & ɔ \\
    OBJ & OBJ\\
     \bottomrule
    \end{tabular}
    \caption{object markers without pronouns}
\end{table} 

\begin{table}[htb!]
    \centering
    \begin{tabular}{ccccc}
        \toprule
        & Singular & \multicolumn{3}{c}{Plural} \\
        \toprule 
        & & 1, 2 & 1, 2, 3 & 1, 3 \\  
        \cmidrule(r){3-3}
        \cmidrule(r){4-4} 
        \cmidrule(r){5-5}
        1 & ń-dó & m-ɔ́ & m-ɔ̀ & \`{n}-dó \\
        & 1SG-PST.3SG.OBJ & 1INCL.DU-PST.3SG.OBJ & 1INCL-PST.3SG.OBJ & 1EXCL-PST.3SG.OBJ\\
        \midrule
        2 & j-ɔ́ & \multicolumn{3}{c}{w-ó} \\
         & 2SG-PST.3SG.OBJ & \multicolumn{3}{c}{2PL-PST.3SG.OBJ} \\
        \midrule
        3 & ɔ́ & \multicolumn{3}{c}{à-n-dò}\\
        & 3SG.PST.3SG.OBJ /  & \multicolumn{3}{c}{3-PL-PST.3SG.OBJ} \\
        & PST.3SG.OBJ \\
        \bottomrule
    \end{tabular}
    \caption{subject pronouns plus past auxiliary plus 3sg object plus object marker, Jianjing some of these have long vowels?}
    \label{tab:fut}
\end{table} 


\end{document}

\begin{table}[htb!]
    \centering
    \begin{tabular}{ccccc}
        \toprule
        & Singular & \multicolumn{3}{c}{Plural} \\
        \toprule 
        & & 1, 2 & 1, 2, 3 & 1, 3 \\  
        \cmidrule(r){3-3}
        \cmidrule(r){4-4} 
        \cmidrule(r){5-5}
        1 & Ń & mɔ́ & mɔ̀ & Ǹ \\
        \midrule
        2 & í & \multicolumn{3}{c}{wó} \\
        \midrule
        3 & à & \multicolumn{3}{c}{àŋ}\\
        \bottomrule
    \end{tabular}
    \caption{Object pronouns}
    
    \label{tab:fut}
\end{table}
\begin{table}[htb!]
    \centering
    \begin{tabular}{ccccc}
        \toprule
        & Singular & \multicolumn{3}{c}{Plural} \\
        \toprule 
        & & 1, 2 & 1, 2, 3 & 1, 3 \\  
        \cmidrule(r){3-3}
        \cmidrule(r){4-4} 
        \cmidrule(r){5-5}
        1 & Ń & Ǹ + length & mɔ́ & Ǹ \\
        \midrule
        2 & í & \multicolumn{3}{c}{wó} \\
        \midrule
        3 & à & \multicolumn{3}{c}{àŋ}\\
        \bottomrule
    \end{tabular}
    \caption{Past tense and adjectival predicates}
    \label{tab:fut}
\end{table}

\begin{table}[htb!]
    \centering
    \begin{tabular}{ccccc}
        \toprule
        & Singular & \multicolumn{3}{c}{Plural} \\
        \toprule 
        & & 1, 2 & 1, 2, 3 & 1, 3 \\  
        \cmidrule(r){3-3}
        \cmidrule(r){4-4} 
        \cmidrule(r){5-5}
        1 & Ń & mɔ̀ & mɔ́ & Ǹ \\
        \midrule
        2 & í & \multicolumn{3}{c}{wó} \\
        \midrule
        3 & à & \multicolumn{3}{c}{àŋ}\\
        \bottomrule
    \end{tabular}
    \caption{inalienable possession}
    \label{tab:fut}
\end{table}

\begin{table}[htb!]
    \centering
    \begin{tabular}{ccccc}
        \toprule
        & Singular & \multicolumn{3}{c}{Plural} \\
        \toprule 
        & & 1, 2 & 1, 2, 3 & 1, 3 \\  
        \cmidrule(r){3-3}
        \cmidrule(r){4-4} 
        \cmidrule(r){5-5}
        1 & ná & mɔ̀á & mɔ́á & náà \\
        \midrule
        2 & já & \multicolumn{3}{c}{wá} \\
        \midrule
        3 & à & \multicolumn{3}{c}{àná}\\
        \bottomrule
    \end{tabular}
    \caption{Alienable possession}
    \label{tab:fut}
\end{table}

\begin{table}[htb!]
    \centering
    \begin{tabular}{ccccc}
        \toprule
        & Singular & \multicolumn{3}{c}{Plural} \\
        \toprule 
        & & 1, 2 & 1, 2, 3 & 1, 3 \\  
        \cmidrule(r){3-3}
        \cmidrule(r){4-4} 
        \cmidrule(r){5-5}
        1 & ḿbé & mwɛ̃̀ & mwɛ̃̂ & mbè \\
        \midrule
        2 & é & \multicolumn{3}{c}{wé} \\
        \midrule
        3 & ɛ̀ & \multicolumn{3}{c}{àmbè}\\
        \bottomrule
    \end{tabular}
    \caption{Nonpast}
    \label{tab:fut}
\end{table}

\newpage 
\subsection{Object pronouns}

\begin{table}[htb!]
    \centering
    \begin{tabular}{ccccc}
        \toprule
        & Singular & \multicolumn{3}{c}{Plural} \\
        \toprule 
        & & 1, 2 & 1, 2, 3 & 1, 3 \\  
        \cmidrule(r){3-3}
        \cmidrule(r){4-4} 
        \cmidrule(r){5-5}
        1 & Ń & mɔ́ & mɔ̀ & Ǹ \\
        \midrule
        2 & í & \multicolumn{3}{c}{wó} \\
        \midrule
        3 & à & \multicolumn{3}{c}{àŋ}\\
        \bottomrule
    \end{tabular}
    \caption{Object pronouns}
    
    \label{tab:fut}
\end{table}

\begin{table}[htb!]
    \centering
    \begin{tabular}{ccccc}
        \toprule
        & Singular & \multicolumn{3}{c}{Plural} \\
        \toprule 
        & & 1, 2 & 1, 2, 3 & 1, 3 \\  
        \cmidrule(r){3-3}
        \cmidrule(r){4-4} 
        \cmidrule(r){5-5}
        1 & Ńdó & mɔ́ & mɔ̀ & Ǹdó \\
        \midrule
        2 & jɔ́ & \multicolumn{3}{c}{wó} \\
        \midrule
        3 & ɔ́ & \multicolumn{3}{c}{àndò}\\
        \bottomrule
    \end{tabular}
    \caption{Object pronouns: lexical}
    \label{tab:fut}
\end{table} 



\end{document}